\documentclass[12pt,letterpaper]{article}

\font\bfit = cmbxti10 scaled \magstephalf

\usepackage{amsmath}
\usepackage{latexsym}
\usepackage{graphicx}
\usepackage{amssymb}
\usepackage{ulem}
%\usepackage{psfig}
\usepackage{fancyheadings}
 
\pretolerance=10000
\textwidth=6.7in
\textheight=8.9in
\topmargin=-0.6in
\headheight=.15in
\hoffset = -0.2in
\headsep=.35in
\oddsidemargin=0in
\evensidemargin=0in
\parindent=2em
\parskip=0.2ex

\newcommand{\nhi}{$N_{\rm HI}$}
\newcommand{\mnhii}{N_{{\rm HI},i}}
\newcommand{\mnhi}{N_{\rm HI}}
\def\mfnhi{f(\mnhi)}
\def\fnhi{$\mfnhi$}
\def\mfnorm{<F>_{\rm norm}}
\def\fnorm{$\mfnorm$}
\def\mlmfp{\lambda_{\rm mfp}^{912}}
\def\lmfp{$\mlmfp$}
\def\mtll{\tau_{\rm eff}^{\rm LL}}
\def\tll{$\mtll$} 
%%%%%
\def\rexp{{\rm e}}
\def\katzt{\ket{\alpha, t_0=0; t}}
\def\katt{\ket{\alpha, t_0; t}}
\def\usc{\mathcal{U}}
\def\tsc{\mathcal{T}}
\special{papersize=8.5in,11in}
\newcommand{\nid}{\noindent}
\newcommand{\ew}{W_\lambda}
\def\aap{A \& A}
\def\aj{AJ}
\def\apj{ApJ}
\def\apss{Ap\&SS}
\def\apjl{ApJL}
\def\apjs{ApJS}
\def\apjsupp{ApJS}
\def\araa{ARAA}
\def\mnras{MNRAS}
\def\nat{Nature}
\def\pasp{PASP}
\def\prd{PhRvD}
\def\hangpara{\par\hangindent 25pt\noindent}
\def\Lya{Ly$\alpha$}
\def\lya{Ly$\alpha$}
\def\Lyb{Ly$\beta$ }
\def\lyb{Ly$\beta$ }
\def\Lyg{Ly$\gamma$ }
\def\Lyd{Ly$\delta$ }
\def\etal{et al. }
\def\kms{km~s$^{-1}$ }
\def\mkms{{\rm \; km\,s\,^{-1}}}
\def\bmkms{{\rm\bf \; km\,s\,^{-1}}}
\def\L{$\lambda$}
\def\s-1{s$^{-1}$}
\def\Hz-1{Hz$^{-1}$}
\def\pohf{\rm p_\ohf}
\def\prhf{\rm p_\rhf}
\def\ohf{\frac{1}{2}}
\def\nohf{\frac{-1}{2}}
\def\rhf{\frac{3}{2}}
\def\sohf{\rm s_\ohf}
\def\esp{\Delta E_{2s, 2p}}
\def\chal{\frac{\Delta \alpha}{\alpha}}
\def\ylab{y_{\rm lab}}
\def\yaY{y \; {\rm and} \; Y}
\def\dadz{\frac{1}{\alpha} |\frac{{\rm d} \alpha}{{\rm d} z}|}
\def\abdadz{\frac{1}{\alpha} \left|\frac{{\rm d} \alpha}{{\rm d} z}\right|}
\def\abdadt{\frac{1}{\alpha} \left|\frac{{\rm d} \alpha}{{\rm d} t}\right|}
\def\sci#1{{\; \times \; 10^{#1}}}
\def\yold{y_{\rm old}}
\def\und#1{{\rm \underline{#1}}}
\def\dlmb{\Delta \lambda}
\def\abchal{| \chal |}
\def\lbar{\bar\lambda}
\def\elc{\rm e^-}
\def\qnn#1{\eqno {\rm {#1}}}
\def\l1l2{\lambda_2 \; {\rm and} \; \lambda_1}
\def\id{\indent}
\def\wid{\id\id}
\def\rid{\id\id\id}
\def\uid{\id\id\id\id}
\def\imp{\, \Rightarrow \>}
\def\di{\partial}
\def\ba{{\bf a}}
\def\bx{{\bf x}}
\def\bv{{\bf v}}
\def\bu{{\bf u}}
\def\bq{{\bf q}}
\def\br{{\bf r}}
\def\bs{{\bf s}}
\def\bp{{\bf p}}
\def\bj{{\bf j}}
\def\bk{{\bf k}}
\def\bm{{\bf m}}
\def\bn{{\bf n}}
\def\bd{{\bf d}}
\def\bg{{\bf g}}
\def\by{{\bf y}}
\def\bB{{\bf B}}
\def\bE{{\bf E}}
\def\bD{{\bf D}}
\def\bH{{\bf H}}
\def\bA{{\bf A}}
\def\bI{{\bf I}}
\def\bJ{{\bf J}}
\def\bK{{\bf K}}
\def\bL{{\bf L}}
\def\bS{{\bf S}}
\def\bF{{\bf F}}
\def\bN{{\bf N}}
\def\bP{{\bf P}}
\def\bR{{\bf R}}
\def\bU{{\bf U}}
\def\bM{{\bf M}}
\def\bY{{\bf Y}}
\def\bOm{{\bf \Omega}}
\def\bom{{\bf \omega}}
\def\bell{{\bf \ell}}
\def\di{\partial}
\def\ddi#1#2{\frac{\di #1}{\di #2}}
\def\gmm{\sqrt{1 - \frac{v^2}{c^2}}}
\def\eng{\varepsilon}
\def\dd#1#2{\frac{d #1}{d #2}}
\def\dt#1{\frac{d #1}{dt}}
\def\ddt{\frac{d}{d t}}
\def\ddit{\frac{\di}{\di t}}
\def\arr{\to \quad}
\def\smm{\sum\limits}
\def\intl{\int\limits}
\def\ointl{\oint\limits}
\def\liml{\lim\limits}
\def\prodl{\prod\limits}
\def\grd{{\bf \nabla}}
\def\crl{{\bf \grd} \times}
\def\dv{{\bf \grd} \cdot}
\def\ec{\frac{e}{c} \,}
\def\oc{\frac{1}{c}}
\def\xht{{\; \rm \hat x}}
\def\yht{{\; \rm \hat y}}
\def\zht{{\; \rm \hat z}}
\def\rht{{\; \rm \hat r}}
\def\pht{{\; \rm \hat \phi}}
\def\iht{{\; \rm \hat i}}
\def\jht{{\; \rm \hat j}}
\def\kht{{\; \rm \hat k}}
\def\theht{{\; \rm \hat \theta}}
\def\nht{{\; \rm \hat n}}
\def\oht{{\; \rm \hat 1}}
\def\tht{{\; \rm \hat 2}}
\def\vht{{\; \rm \hat v}}
\def\thht{{\; \rm \hat 3}}
\def\Rht{{\; \rm \hat R}}
\def\cyc{\Omega_c}
\def\dl{{d {\bf \ell}}}
\def\da{{\rm d {\bf a}}}
\def\ds{{\rm d {\bf s}}}
\def\smt{{\smll \rm T}}
\def\opi#1{\frac{1}{#1 \pi}}
\def\strs{\underline {\underline {\rm T}}}
\def\dtr{\, d^3 r}
\def\dftr{d^3 r \,}
\def\dtp{\, d^3 p}
\def\dftp{d^3 p \,}
\def\abrr{|\br - \br'|}
\def\rr{(\br - \br')}
\def\orr#1{\frac{#1}{\abrr}}
\def\trr#1{\frac{#1}{\abrr^3}}
\def\drr{\delta \rr}
\def\Grn{G(\br, \br')}
\def\Ylm{Y_{\ell m}}
\def\snth{\sin\theta}
\def\csth{\cos\theta}
\def\invr#1{\frac{1}{#1}}
\def\fpc{\frac{4 \pi}{c}}
\def\lgnl{P_\ell (\csth)}
\def\lgnn#1{{P_{#1} (\csth)}}
\def\thph{\theta, \phi}
\def\eht1{{\rm \hat e_1}}
\def\ehtb{{\rm \hat e_2}}
\def\iot{i \omega t}
\def\kr{\bk \cdot \br}
\def\ele{{\rm e^-}}
\def\iwc{\frac{i \omega}{c}}
\def\iwt{i \omega t}
\def\plr{{\rm \hat \epsilon}}
\def\skp{\smallskip}
\def\mkp{\medskip}
\def\bkp{\bigskip}
\def\elc{\rm e^{-s}}
\def\bmx{\left[\matrix}
\def\Det{{\rm Det \,}}
\def\bra#1{{<\negmedspace#1|}}
\def\ket#1{{|#1\negmedspace>}}
\def\kalp{\ket{\alpha}}
\def\kbet{\ket{\beta}}
\def\balp{\bra{\alpha}}
\def\bbet{\bra{\beta}}
\def\inpr#1#2
\def\otpr#1#2{{|#1><#2|}}
\def\dgg{\dagger}
\def\dual{\Leftrightarrow}
\def\hbm{\frac{\hbar^2}{2 m}}
\def\hbu{\frac{\hbar^2}{2 \mu}}
\def\mbh{\frac{2 m}{\hbar^2}}
\def\ih{\frac{i}{\hbar}}
\def\kjm{\ket{j m}}
\def\bjm{\bra{j m}}
\def\bjmp{\bra{j' m'}}
\def\ylm{Y_{\ell m}}
\def\pof#1{(\frac{#1}{2})}
\def\tqk{T_q^k}
\def\ptm{\frac{p^2}{2 m}}
\def\psn{\psi_n}
\def\tph{\frac{2 \pi}{\hbar}}
\def\plr{{\hat \epsilon}}
\def\Veff{V_{\rm eff}}
\def\rhs{\rho^{(s)}}
\def\rhis{\rho_i^{(s)}}
\def\tilH{\tilde H}
\def\tpi{2 \pi i}
\def\prin{\intl_P}
\def\lta{\left < \,}
\def\rta{\right > \,}
\def\ltk{\left [ \,}
\def\ltp{\left ( \,}
\def\ltb{\left \{ \,}
\def\rtk{\, \right  ] }
\def\rtp{\, \right  ) }
\def\rtb{\, \right \} }
\def\dc{\tilde d}
\def\bskp{\bigskip}
\def\nV{\Omega}
\def\d3x{d^3 x}
\def\mtI{\b{\b I}}
\def\cmma{\;\;\; ,}
\def\perd{\;\;\; .}
\def\delv{\Delta v}
\def\eikr{{\rm e}^{i \bk \cdot \br}}
\def\rmikr{{\rm e}^{i \vec k \cdot \vec r}}
\def\rmnikr{{\rm e}^{-i \vec k \cdot \vec r}}
\def\eikx{{\rm e}^{i \bk \cdot \bx}}
\def\muo{\mu_0}
\def\epo{\epsilon_0}
\def\rme{{\rm e}}
\def\dnu{\Delta \nu_D}
\def\rAA{\AA \enskip}
\def\halp{{\rm H_\alpha}}
\def\hbet{{\rm H_\beta}}
\def\N#1{{N({\rm #1})}}
\def\ekz{\rme^{i ( k_z z - \omega t )}}
\def\cm#1{\, {\rm cm^{#1}}}
\def\bcm#1{\; {\rm \mathbf{cm^{#1}}}}
\def\rmK{\; \rm K}
\def\rmm#1{\; {\rm #1}}
\def\Nperp{N_\perp}
\def\spkr#1{{\big \bf {#1}}}
\def\inA{\quad $\bullet$ \quad}
\def\inB{\quad\quad $\diamond$ \quad}
\def\inC{\quad\quad\quad $\circ$ \quad}
\def\sigfc{\sigma_{f_c}}
\def\sech{\rm sech}
\def\blck{\color[named]{Black}}
\def\blue{\color[named]{Blue}}
\def\skyblue{\color[named]{SkyBlue}}
\def\red{\color[named]{Red}}
\def\black{\color[named]{Black}}
\def\green{\color[named]{Green}}
\def\white{\color[named]{White}}
\def\yellow{\color[named]{Yellow}}
\def\purple{\color[named]{Purple}}
\def\maroon{\color[named]{Maroon}}
\def\redv{\color[named]{RedViolet}}
\def\mahog{\color[named]{Mahogany}}
\def\prb{$\frac{\delta \rho}{\rho}$}
\def\trd{\frac{2}{3}}
\def\mdn{\bar \rho}
\def\ss{v_s^2}
\def\msol{M_\odot}
\def\msun{M_\odot}
\def\W#1{{W({\rm #1})}}
\def\Nav{N_a (v)}
\def\f#1{{f_{\rm #1}}}
\def\opt{$d \tau / d z \; $}
\def\btau{\bar\tau (v_{pk}) / \sigma (\bar \tau)}
\def\Ipk{I(v_{pk})/I_{c}}
\def\vrot{v_{rot}}
\def\sphr{\sqrt{R^2 + Z^2}}
\def\mbf{\mathbf}
\def\mgsqa{\, {\rm mag} / \Box ''}
\def\sqas{$\Box ''$}
\def\msqas{\Box ''}
\def\sqasec{$\Box ''$}
\newcommand{\ialii}{Al{\sc II}}
\newcommand{\iari}{Ar{\sc I}}
\newcommand{\ini}{N{\sc I}}
\newcommand{\ioi}{O{\sc I}}
\newcommand{\iovi}{O{\sc VI}}
\newcommand{\iovii}{O{\sc VII}}
\newcommand{\ioviii}{O{\sc VIII}}
\newcommand{\ifeii}{Fe{\sc II}}
\newcommand{\ifeiii}{Fe{\sc III}}
\newcommand{\ispii}{S{\sc II}}
\newcommand{\iznii}{Zn{\sc II}}
\newcommand{\icaii}{Ca{\sc II}}
\newcommand{\inai}{Na{\sc I}}
\newcommand{\icii}{C{\sc II}}
\newcommand{\iciv}{C{\sc IV}}
\newcommand{\ihii}{H{\sc II}}
\newcommand{\ihi}{H{\sc I}}
\newcommand{\isiii}{Si{\sc II}}
\newcommand{\isiiv}{Si{\sc IV}}
\newcommand{\ici}{C{\sc I}}
\newcommand{\dcost}{d \negmedspace \cos\theta}
\newcommand{\mnvhi}{n_{\rm H {\sc I}}}
\newcommand{\nvhi}{$n_{\rm H {\sc I}}$}
\newcommand{\rem}{{\rm e}}
\def\otpr#1#2{{|#1><#2|}}
\def\avgq#1{<\negmedspace#1\negmedspace>}
\newcommand{\negm}{\negmedspace}
\newcommand{\dfhf}{^2\negm D_{\frac{5}{2}}}
\newcommand{\negt}{\negthinspace}
\newcommand{\nalph}{N/$\alpha$}
\newcommand{\alphh}{$\alpha$/H}
\newcommand{\Rearth}{R_\oplus}
\newcommand{\mhmpc}{h^{-1} \, \rm Mpc}
\newcommand{\hmpc}{$\mhmpc$}

\renewcommand{\descriptionlabel}[1]%
    {\hspace{\labelsep}\textsf{#1}}

\newenvironment{Aitemize}{%
 \renewcommand{\labelitemi}{$\bullet \;$}%
	\begin{itemize}}{\end{itemize}}

\newenvironment{Renumerate}{%
 \renewcommand{\theenumi}{\Roman{enumi}}
 \renewcommand{\labelenumi}{\theenumi)}
	\begin{enumerate}}{\end{enumerate}}

\newenvironment{Aenumerate}{%
 \renewcommand{\theenumi}{\Alph{enumi}}
 \renewcommand{\labelenumi}{\theenumi.}
	\begin{enumerate}}{\end{enumerate}}

\newenvironment{dmnd}{%
 \renewcommand{\labelitemi}{$\diamond \quad$}%
	\begin{itemize}}{\end{itemize}}

\newenvironment{Penumerate}{%
 \renewcommand{\labelenumi}{(\theenumi)}
	\begin{enumerate}}{\end{enumerate}}


\pagestyle{fancyplain}

\lhead[\fancyplain{}{Intro}]{\fancyplain{}{\thepage}}
\rhead[\fancyplain{}{Intro-\thepage}]{\fancyplain{}{SaasFee16-\fnhi}}
\cfoot{}

\newcommand{\mndi}{N_{\rm DI}}
\newcommand{\ndi}{$N_{\rm DI}$}
\newcommand{\helium}{${^4\rm He}$}
\newcommand{\rmch}{\frac{\rm C}{\rm H}}
\renewcommand{\labelitemii}{$\diamond$}
\renewcommand{\labelitemiii}{$\blacktriangle$}
\renewcommand{\labelitemiv}{$\circ$}

%\newcommand{\dfhf}{^2\negm D_{\frac{5}{2}}}
%\newcommand{\dtvr}{\dot{\vec r}}
%\newcommand{\mtrx}{\bra{\phi_f} \rmikr \hat\eng \cdot \vec\nabla \ket{\phi_i}}
%\newcommand{\prfi}{\ltp \hat\eng \cdot \vec r \rtp_{fi}}


\special{papersize=8.5in,11in}

\begin{document}

\noindent {\bf{\large III. Characterization of the \lya\ Forest [\fnhi, v2.1]}}

\begin{Aenumerate}

{\bf \item Goals}
 \begin{itemize}
 \item Provide an empirical formalism and description of the \lya\ Forest
 \item Discuss modern observational constraints
 \item Discuss implications and applications (primarily at $z \sim 2-3$)
 \item Fundamentally, any precise measurement from the IGM offers a 
 valuable constraint on our cosmology
 	\begin{itemize}
 	\item We have a well-developed theory of structure formation (e.g.\ N-body
 	simulations)
 	\item On quasi-linear scales, the baryons track the dark matter
 	\item Only `free' parameter is the radiation field which may also be constrained
 	\end{itemize}
 \item This Lecture primarily considers optically thin gas ($\mnhi<10^{17} \cm{-2}$)
 	\begin{itemize}
 	\item The proper IGM, quasi-linear gas between galaxies, likely
 	corresponds to even lower \nhi\ ($<10^{14} \cm{-2}$)
 	\item We will adopt a stricter definition in future Lectures
 	\end{itemize}
 \end{itemize}

{\bf \item Visual Impressions (slides)}

 \begin{itemize}
  \item Also explore Q1422 ({\bf Notebook})
 \end{itemize}

{\bf \item \nhi\ frequency distribution \fnhi: Concept}

 \begin{itemize}
  \item Model the \lya\ Forest as a series of absorption lines ($z,N,b$)
  	\begin{itemize}
  	\item Voigt-profile fitting of the \lya\ forest
  		\begin{itemize}
  		\item Include higher order Lyman series lines to improve measurements
  		\item $\tau_0 \propto \lambda f N$
	  	\end{itemize}
  	\item e.g. Kirkman \& Tytler 1997

	\includegraphics[width=6.0in]{Paper_figs/kirkman97_fig1.pdf}

  	\item e.g. Kim et al.\ 2001 (Figure 1)

	\includegraphics[width=6.0in]{Paper_figs/kim01_fig1.pdf}

  	\item Very expensive (if human)

  	\end{itemize}
  \item Generate a distribution of $N,b$ values as a function of $z$
  \item Pros
  	\begin{itemize}
  	\item Captures the stochastic nature of the \lya\ Forest
  	\item Captures the absorption-line appearance of the \lya\ Forest
  	\item Enables the calculation of other valuable measures of the \lya\ Forest
  	\item Primarily an empirical description (also a con)
  	\end{itemize}
  \item Cons
  	\begin{itemize}
  	\item Not a physical model
  	\item Difficult to derive from IGM models/simulations
  	\item Very expensive to measure observationally (laborious)
  	\item Human interaction implies non-reproducible results
  	\item Non-HI gas (i.e. metals) abounds in the \lya\ Forest
  	\end{itemize}
 \end{itemize}

{\bf \item \fnhi\ Definition}
 \begin{itemize}
 \item $f(\mnhi, b, z) \, d\mnhi \, db \, dz = $ number of lines in the intervals
 $\mnhi, \mnhi+d\mnhi$; $b, b+db$; $z, z+dz$
 	\begin{itemize}
 	\item Akin to a luminosity function $\phi(L) dL$
 		\begin{itemize}
 		\item Number of galaxies per volume in a luminosity interval $L, L+dL$
 		\end{itemize}
 	\item But defined on a sightline instead of within a volume
	\item Cartoon
	\end{itemize}
\item Standard assumptions:
	\begin{itemize}
	\item The Doppler parameter ($b$-value) distribution has 
	minimal \nhi\ (or $z$) dependence
 		\begin{equation}
 		f(\mnhi,b) = \mfnhi \, g(b)
 		\end{equation}
	\item One frequently assumes the shape of \fnhi\ is also 
	nearly constant with $z$
	\end{itemize}

 \end{itemize}

{\bf \item Survey Path $\Delta z$}
 \begin{itemize}
 \item Absorption-line surveys are usually carried out across multiple sightlines
 	\begin{itemize}
	 \item Required to build statistics
	 \item Each sightline, however, will have its own S/N (and possibly other
	 spectral characteristics)
	 \item And its own systematic uncertainties (e.g.\ continuum); usually
	 ignored
	 \end{itemize}
 \item {\bf Slides}
 \item Single sightline
	 \begin{itemize}
	 \item Count lines across a wavelength interval
	 \item This defines a (usually small) redshift interval, $\delta z$
	 \item Sensitivity to lines may vary within $\delta z$ [ignored in 
	 these Lectures]
	 \end{itemize}
 \item Survey multiple sightlines
	 \begin{itemize}
	 \item Survey path
	 \begin{equation}
	 \Delta z = \smm_i (\delta z)_i
	 \end{equation}
	 \item Again, $\Delta z$ may have an explicit dependence on the line 
	 parameters ($N, b$)
	 \end{itemize}
 \item Aside:  I will define the $\Delta X$ survey path in a later Lecture.
 	\begin{itemize}
	\item It is conceptually the same
	\item But has physical meaning ($dz$ is observational)
	in a F-W cosmology
 	\end{itemize}
 \end{itemize}

{\bf \item Binned Evaluations of \fnhi}
 \begin{itemize}
 \item Estimate \fnhi\ within a finite $\Delta \mnhi$ interval
 	\begin{itemize}
 	\item Intervals must be large enough to have statistical significance
 	\end{itemize}
 \item Repeat along multiple sightlines to survey a redshift path $\Delta z$
 	\begin{itemize}
 	\item  Estimating $\Delta z$ (example):
 		\begin{itemize}
 		\item Count lines in $\mathcal{N}$ quasar spectra from $z=2.25$ to $2.75$
 		\item Each quasar contributes $\delta z = 0.5$
 		\item $\Delta z = \mathcal{N} \, \delta z$
 		\end{itemize}
 	\end{itemize}
 \item Simple estimator
 \begin{equation}
 f(\mnhi,z) = \frac{\rm \# \; lines \; in \; [\Delta \mnhi,\Delta z]}{
 \Delta\mnhi \, \Delta z}
 \end{equation}
 	\begin{itemize}
 	\item Assume Poission statistics to estimate uncertainty
 	\item Subtlety: 
 		\begin{itemize}
 		\item Evolution in $f(\mnhi,z)$ with \nhi\ or $z$ 
 		skews the detection within a bin (lower \nhi, higher $z$)
 		\item Simple $\chi^2$ analysis using the center of the
 		bins will be (marginally) wrong
 		\end{itemize}
	\includegraphics[width=2.7in]{Paper_figs/tytler87_fig2.pdf}
	\includegraphics[width=2.7in]{Paper_figs/petitjean93_fig1.pdf}

 	\item Early results
 		\begin{itemize}
 		\item Tytler 1987 (left); Petitjean et al.\ 1993 (right)
 		\item Well-described with a power-law
 		\item But not perfectly..
 			\begin{itemize}
 			\item Tytler proposed a single-population of gas `clouds'
 			\item Petitjean et al.\ argued for distinct populations
 			\item The modern view is even more complex
 			\end{itemize}
 		\end{itemize}
 	\item Keck arrives
 		\begin{itemize}
 		\item Hu et al.\ 1996
 		\item Kirkman \& Tytler 1997
 		\end{itemize}

	\includegraphics[width=5.0in]{Paper_figs/kirkman97_fig3.pdf}

 		\item Turn-over at low \nhi
 		\begin{itemize}
 		\item Lack of sensitivity (i.e.\ completeness?)
 			\begin{itemize}
 			\item $\tau_0 \approx 0.025$ for $\mnhi = 10^{12} \cm{-2}$!
 			\end{itemize}
 		\item Poor estimation of the continuum?
 			\begin{itemize}
 			\item 5\%\ error in the continuum overwhelms such a signal
 			\end{itemize}
 		\item Breakdown of \fnhi\ methodology (e.g.\ line-blending)? 
 		\item Physical?
 			\begin{itemize}
 			\item Non-linear ionization of low density gas
 			\end{itemize}
 		\end{itemize}

 	\item Modern Estimates
 		\begin{itemize}
 		\item Data is now exquisite (statistically)
 			\begin{itemize}
 			\item Rudie et al.\ 2013
	 		\item Kim et al.\ 2013
 			\end{itemize}
 		\item (Comparison from Prochaska+14, Fig 5)
 			\begin{itemize}
 			\item Differences in results stress systematic error
 			\item Keep in mind these are on a log-log plot
 			\item Can we resolve these?
 			\end{itemize}
 		\end{itemize}

	\includegraphics[width=5.0in]{Paper_figs/prochaska14_fig5.pdf}

 	\end{itemize}
 \end{itemize}

{\bf \item Models for \fnhi}
	\begin{itemize}
	\item Motivations
		\begin{itemize}
		\item Reduce the binned evaluations to a few, well-constrained parameters
		\item Better enable the examination of redshift evolution
		\item Required to evaluate \fnhi\ for other measures (stay tuned)
		\end{itemize}
	\item Challenges / limitations
		\begin{itemize}
		\item No physical model for \fnhi
		\item Results are sensitive to \nhi\ limits
		\end{itemize}
	\item Functional forms
		\begin{itemize}
		\item Single power-law (Tytler 87)
		\begin{equation}
		\mfnhi = B N^\beta
		\end{equation}
			\begin{itemize}
			\item $\beta = -1.51 \pm 0.02$
			\item Remarkably (over optimistic) small error
			\end{itemize}
		\item Broken power-laws (Prochaska et al.\ 2010)
		\item Broken, disjoint power law (Rudie et al.\ 2013)
		\item Sum of several functional forms (Inoue et al.\ 2014)
		\item Spline evaluation (Prochaska et al.\ 2014)
		\item Increasing complexity 
			\begin{itemize}
			\item Reflects increasing quality (statistical power) of the observations
			\item And the increasing range of \nhi\ considered
			\end{itemize}
		\end{itemize}
	\item $\chi^2$ Analysis
		\begin{itemize}
		\item Assume Gaussian statistics in the binned evaluations
			\begin{itemize}
			\item i.e. $\sigma(\mfnhi)^2 = $ number of lines in the bin
			\end{itemize}
		\item Minimize $\chi^2$
		\item Complications
			\begin{itemize}
			\item Results are sensitive to the choice of bins
			\item What centroid should be adopted for the bin center?
			\end{itemize}
		\end{itemize}
	\item Likelihood Analysis
		\begin{itemize}
		\item See Cooksey et al. (2010) (and previously Storrie-Lombardi et al.\ 1996)
		\item Consider a survey with path $\Delta z$ [Slides]
			\begin{itemize}
			\item Focus solely on $f(\mnhi,z)$ at a given $z'$ 
			(i.e.\ focused on the shape and normalization at $z=z'$)
			\item Divide the distribution space into $M$ cells with 
			volume: $\delta v = \Delta z \, \Delta \mnhi$
			\end{itemize}
		\item Expected number $\mu_i$ in $i$th cell:
			\begin{itemize}
			\item It follows from our definition of \fnhi
			\begin{equation}
			\mu_i = f(\mnhi)_i \, \Delta z \, \Delta\mnhii
			\end{equation}
			\item Probability of detecting $m$ absorbers within 
		cell $i$ is given by Poisson statistics
		\begin{equation}
		P(m; \mu_i) = {\rm e}^{-\mu_i} \frac{\mu_i^m}{m!}
		\end{equation}
			\end{itemize}
		\item Likelihood function
			\begin{itemize}
			\item Simple product over all cells
			\end{itemize}
		\begin{equation}
		\mathcal{L} = \prodl_i^M P(m; \mu_i)
		\end{equation}

		\item Swindle?!
			\begin{itemize}
			\item Reduce the volume of the cell so that each contains
			at most one absorber
			\item Let $\Delta\mnhii \to 0$
			\item $P(m;\mu_i)$ reduces to ${\rm e}^{-\mu_i}$ for an
			empty cell and ${\rm e}^{-\mu_i} \mu_i$ for a cell with 1 system
			\end{itemize}
		\item Likelihood
			\begin{itemize}
			\item Sum over all $M$ cells  
			\item Let there be $p$ lines detected giving $g=M-p$ empty cells
			\end{itemize}
		\begin{align}
		\mathcal{L} &= \prodl_{i=1}^g {\rm e}^{-\mu_i} \;
		\prodl_{j=1}^p {\rm e}^{-\mu_j} \, \mu_j \\
		  &= \prodl_{i=1}^M {\rm e}^{-\mu_i} \; \prodl_{j=1}^p \mu_j
		\end{align}
		%\begin{equation}
		%\ln \mathcal{L} = -\intl_{N_{\rm min}}^{N_{\rm max}} \mfnhi \, \Delta z \, d\mnhi - \smm_i^m \ln[f(\mnhii) \, \Delta z]
		%\end{equation}
		\item log-Likelihood

		\begin{align}
		\ln \mathcal{L} &= \smm_i^M -\mu_i \, + \, \smm_i^p \ln \mu_i \\
		   &= \smm_i^M -f(\mnhi)_i \Delta z \, \Delta\mnhi +
		   \smm_j^p \ln f(\mnhi)_j \Delta z + p \ln [\Delta\mnhi]
		\end{align}

			\begin{itemize}
			\item In our limit with $\Delta\mnhii \to 0$
			\item We ignore the last term (constant) and take the integral form
			of the first term
			\begin{equation}
		\ln \mathcal{L} = - \intl_{N_{min}}^{N_{max}} f(\mnhi) \Delta z \, d\mnhi 
		   \, + \, \smm_{j=1}^p \ln f(\mnhi)_j \Delta z
			\end{equation}
			\item If $\Delta z$ is independent of \nhi\ it may be ignored 
			\item See Cooksey et al.\ 2010 for a treatment where
			$\Delta z$ is dependent on \nhi
			\end{itemize}
		\end{itemize}
		\item Maximize $\mathcal{L}$ for the parameterization of \fnhi
			\begin{itemize}
			\item Results provide the ``best'' values for the parameters
			\item But not necessarily a good model!  (need another statistical
			test; see Lyman Limit Lecture)
			\end{itemize}
		\item Current results 
			\begin{itemize}
			\item Rudie et al.\ 2013: $\mnhi = 10^{13.5} - 10^{17} \cm{-2}$, 
			$z \approx 2.5$
			\begin{equation}
			\mfnhi \propto \mnhi^{-1.65 \pm 0.02}
			\end{equation}

	\includegraphics[width=5.0in]{Paper_figs/rudie13_fig3.pdf}

			\item Kim et al.\ 2013: $\mnhi = 10^{12.75} - 10^{18} \cm{-2}$,
			$z \approx 2.8$
			\begin{equation}
			\mfnhi \propto \mnhi^{-1.52 \pm 0.02}
			\end{equation}
			\item Competing results which follows from their different 
			measures of the binned evaluations
			\item Progress requires resolving systematic errors, including
			human biases (ugh)
			\end{itemize}

	%\includegraphics[width=4.0in]{Paper_figs/kim13_fig10.pdf}

	\end{itemize}

{\bf \item $b$-value Distribution}
	\begin{itemize}
	\item Line-profile fitting also yields a distribution of $b$-values
		\begin{itemize}
		\item These may be described by a separate distribution function
		\end{itemize}
	\item Observations (Kirkman \& Tytler 1997, Figure 4)
		\begin{itemize}
		\item No strong dependence on \nhi
		\item Possible increase in the minimum $b$-value with increasing \nhi
		\end{itemize}

	\includegraphics[width=4.0in]{Paper_figs/kirkman97_fig5.pdf}

	\includegraphics[width=5.0in]{Paper_figs/kirkman97_fig4.pdf}

	\item Distribution function
		\begin{itemize}
		\item Motivated from theory (Hui \& Rutledge 1999)
			\begin{itemize}
			\item i.e.\ optical depth fluctuations
			\item Without derivation:
			\end{itemize}
		\begin{equation}
		g(b) = \frac{4 b_\sigma^4}{b^5} \, \exp \ltp - \frac{b_\sigma^4}{b^4} \rtp
		\end{equation}
		\item Single parameter model ($b_\sigma$)
		\item Good description of the observations with $b_\sigma = 26.3$ km/s
		\item Hui \& Rutledge 1999, Figure 1

	\includegraphics[width=4.0in]{Paper_figs/hui99_fig1.pdf}

		\end{itemize}

	\item Average value
	\begin{equation}
	<b> = \frac{\int b g(b) \, db}{\int g(b) \, db} = b_\sigma \Gamma(3/4)
	\approx 32 \, {\rm km/s}
	\end{equation}

	\item Lower limit: $b_{\rm min} \approx 18$ km/s
		\begin{itemize}
		\item What sets the lower bound to the $b$-value?
		\item Recall:  $b = \sqrt{\frac{2kT}{m_A} + \xi^2}$
		\item One is tempted to associate $b_{\rm min}$ to the IGM temperature
		\item For purely thermal broadening of Hydrogen, 
		\begin{equation}
		T = 10^4 \, {\rm K} \; \ltp \frac{b}{13 \, \rm km/s} \rtp^2
		\end{equation}
		\item We return to the IGM temperature in a future Lecture
		\end{itemize}

	\item Upper bound: $b_{\rm max} \approx 100$ km/s
		\begin{itemize}
		\item Collisional ionization?  (see Broad \lya\ lines [BLAs] at low-$z$)
		\item Sensitivity? ($\tau_0 \propto N\lambda f / b$)
		\item Line-blending prohibits the measurement of larger values?
		\end{itemize}

	\end{itemize}

{\bf \item Line Density (Incidence of Absorption Lines)}
	\begin{itemize}
	\item (confusing) Notation: $dN/dz, n(z), N(z), \ell(z)$
	\item Definition:  $\ell(z) dz = $ number of lines detected
	on average in the interval $z, z+dz$ over an interval of 
	column density $\mnhi = [N_{\rm min}, N_{\rm max}]$
	\item Calculation
	\begin{equation}
	\ell(z) dz = \intl_{N_{\rm min}}^{N_{\rm max}} f(\mnhi,z) \, d\mnhi \, dz
	\end{equation}
		\begin{itemize}
		\item This is the zeroth moment of our frequency distribution
		\item Akin to the number density of galaxies derived from a 
		luminosity function
		\end{itemize}
	\item Estimate at $z \approx 2.8$ (example)
		\begin{itemize}
		\item Adopt Kim et al.\ 2013, power-law model 
		(converted to redshift space; $dX/dz = 3.5$)
		\begin{equation}
		f(\mnhi, z \approx 2.8) = 10^{9.1} \mnhi^{-1.52}
		\end{equation}
		\item Integrate 
		\begin{equation}
		\ell(z \approx 2.8) = \frac{10^{9.1}}{0.52} \mnhi^{-0.52} 
		|_{N_{\rm min}}^{N_{\rm max}}
		\end{equation}
			\begin{itemize}
			\item This is dominated by the lowest \nhi\ systems ({\bf Notebook})
			\item Consider a 5\AA\ patch of spectrum at 4600\AA\
			and $N_{\rm min} = 10^{12.75} \cm{-2}$
				\begin{itemize}
				\item $\delta z = (1+z) (\delta\lambda/\lambda) = 0.004$
				\item $\ell(z) \, \delta z = 2.4$  
				\item (This seems a bit low)
				\end{itemize}
			\end{itemize}
		\end{itemize}
	\item Observations at $z \sim 2-3$
		\begin{itemize}
		\item Kim et al.\ 2013, Figure 9

	\includegraphics[width=4.0in]{Paper_figs/kim13_fig9.pdf}

		\item Strong evolution, as evident in individual spectra
		\item Probable \nhi\ dependence to $\ell(z)$
		\end{itemize}

	\item Modeling the Redshift evolution
		\begin{itemize}
		\item What physical model might we expect?
		\item Expansion [Following Meiksin 2009]
			\begin{itemize}
			\item Cartoon
			\item Imagine a population of absorbers with physical (proper)
			number density $n_p(z)$ and proper cross-section $A_p(z)$
			\item Expect $dN/dr_p = n_p(z) A_p(z)$ absorbers per proper
			path length $r_p$
			\item Cosmology
			\begin{equation}
			\frac{dr_p}{dz} = \frac{c}{H(z) (1+z)}
			\label{eqn:drdz}
			\end{equation}
			\begin{equation}
			\frac{dN}{dz} = n_p(z) A_p(z) \frac{c}{H(z) (1+z)}
			\end{equation}
			\item Hubble Parameter
			\begin{equation}
			H(z) = H_0 \sqrt{\Omega_m (1+z)^3 + \Omega_\Lambda}
			\end{equation}
				\begin{itemize}
				\item At $z>2$, the universe is matter dominated and 
				$\Omega_\Lambda$ may be ignored
				\item $H(z) \approx H_0 \Omega_m^{1/2} (1+z)^{3/2}$
				\end{itemize}
			\item Ansatz a constant comoving
			population, $n_p(z) = n_c(z) (1+z)^{3}$, with a constant
			physical size
			\item Altogether,
			\begin{equation}
			\ell(z) = \frac{dN}{dz} \propto n_c(z) A_p(z) (1+z)^{\ohf}
			\end{equation}
			\item On its own, this implies a weak dependence with redshift
			(weaker than observed)
			\end{itemize}
		\item Mean density
		\begin{equation}
		\bar\rho \propto (1+z)^3
		\end{equation}
			\begin{itemize}
			\item Increasing neutral fraction with $z$ 
				\begin{itemize}
				\item Provided the emissivity from ionizing sources does
				not rise more steeply than $(1+z)^3$
				\item Radiation field should also scale as $(1+z)^\gamma$
				\end{itemize}
			\item Sensitive to lower densities, i.e. a greater portion 
			of the volume
			\item Together, these imply a steeper evolution in $\ell(z)$
			\end{itemize}
		\item Together, we have a physical motivation for a $(1+z)^\gamma$
		evolution in $\ell(z)$

		\item Current results at $z \sim 2-3$  (Kim et al.\ 2013, Figure 9; above)
			\begin{itemize}
			\item Not based on Maximum Likelihood
			\item For $\mnhi < 10^{14} \cm{-2}$
			\begin{equation}
			\ell(z) = 100 (1+z)^{1.12 \pm 0.24}
			\end{equation}
			\item For $\mnhi = [10^{14}, 10^{17}] \cm{-2}$
			\begin{equation}
			\ell(z) = 0.4 \, (1+z)^{4.14 \pm 0.6}
			\end{equation}
			\end{itemize}

	\includegraphics[width=4.0in]{Paper_figs/kim13_fig5.pdf}

		\item Current results including measurements at $z<1$ (Kim et al.\ 2003, Figure 5)
			\begin{itemize}
			\item Power-law extrapolation of $\ell(z)$ to $z=0$ fails
			\item Observed incidence of absorption is much *higher*
				\begin{itemize}
				\item Caution: sample size is modest
				\item Caution: Comparison at fixed \nhi\ implies varying gas density (see Schaye01 formalism to come)
				\end{itemize}
			\item Nevertheless, we infer a greatly reduced EUVB
				\begin{itemize}
				\item Consistent with declining quasar population
				\item Although not exactly (see Kollmeier et al.\ 2014)
				\end{itemize}
			\end{itemize}


		\end{itemize}
	\end{itemize}

{\bf \item Mock Spectra of the \lya\ Forest}
	\begin{itemize}
	\item Random realization of the IGM
		\begin{itemize}
		\item \fnhi\ provides a natural description of the IGM stochasticity
		\item Useful for generating mock realizations of the \lya\ Forest
		\item Cheaper (at least historically), then draws from an IGM simulation
			\begin{itemize}
			\item Simulations struggle to reproduce the incidence of
			gas with high \nhi
			\item Difficult to map simulations across cosmic time
			\end{itemize}
		\end{itemize}
	\item Recipe
		\begin{itemize}
		\item Define a redshift interval $\delta z$ for the mock Forest
		\item Define \nhi\ bounds for \fnhi\ (usually the full dynamic range)
		\item Calculate the average number of lines 
		($N_{lines} = \ell(z) \delta z$)
		\item Random draw (Poisson) from $N_{lines}$
		\item Draw \nhi\ values from \fnhi
		\item Draw $b$ from $g(b)$
		\item Draw $z$ from $\ell(z)$
		\item Calculate $\tau_\lambda$, including all 
		Lyman series lines (as applicable)
			\begin{itemize}
			\item On a wavelength grid fine enough to 
			capture the line profile
			\item i.e.\ a perfect spectrometer
			\end{itemize}
		\item Calculate $F_\lambda = \exp[-\tau_\lambda]$
		\item Add in spectral characteristics
			\begin{itemize}
			\item Convolve with instrument LSF
			\item Rebin to final wavelength array
			\item Add in Noise
			\end{itemize}
		\item Add in source SED
		\end{itemize}
	\item See {\bf Notebook}
	\item Entertaining exercise
		\begin{itemize}
		\item Generate 100 mock spectra from an input \fnhi
		\item Fit the data
		\item Do you recover \fnhi!?
		\end{itemize}
	\end{itemize}

%%%%%%%%%%%%%
{\bf \item Effective \lya\ Opacity: $\tau_{\rm eff}^{\rm Ly\alpha}$}
	\begin{itemize}
	\item Observed IGM is highly stochastic
		\begin{itemize}
		\item Modeled as random absorption lines
		\item Undulating, random density field
		\item How do we compare this to cosmological models?
		\end{itemize}
	\item Definitions
		\begin{itemize}
		\item Let \fnorm\ be the average, normalized flux in
		the \lya\ forest %(redward of \lyb)
		\item Effective \lya\ opacity
		\begin{equation}
		\tau_{\rm eff, \alpha} = - \ln \mfnorm
		\end{equation}

		\item Also define $D_A$ (\lya\ decrement; an equivalent measure)
		\begin{equation}
		D_A = 1 \, - \mfnorm
		\end{equation}
		\end{itemize}
	\item Assessing $\tau_{\rm eff, \alpha}$ 
		\begin{itemize}
		\item Empirical:  Average over many (100+) spectra
			\begin{itemize}
			\item See slides
			\end{itemize}
		\item Theoretical:
			\begin{itemize}
			\item Average opacity is a balance between 
			$\Omega_b, \bar\rho$, and the radiation field
			\item Measuring/knowing three of these permits an estimate of the 
			remaining one
			\end{itemize}
		\end{itemize}
	\item Calculate from \fnhi\ \; [Moller \& Jakobsen 1990; Press et al.\ 1993]
		\begin{itemize}
		\item Calculate the average opacity from an \nhi\ distribution
		\item Requires the $b$-value distribution too
			\begin{itemize}
			\item Especially for saturated lines
			\item $W_\lambda \propto b$
			\end{itemize}
		\item Consider the number of lines $\mathcal{N}$ per unit rest wavelength 
		with
		\begin{equation}
		d\lambda_{\rm rest} = \lambda_{\rm rest} dz / (1+z)
		\end{equation}
			\begin{itemize}
			\item We know how to calculate the number of lines per $dz$
			\item Integrate our HI frequency distribution
		  	\begin{equation}
		  	\frac{dN}{dz} = \int f(\mnhi,b,z) dN db  
	  		\end{equation}
			\item Now transfer to per rest wavelength interval
	  		\begin{equation}
		  	\mathcal{N} = \frac{1+z}{\lambda_{\rm rest}} \int f(\mnhi,b,z) dN db
	  		\end{equation}
			\end{itemize}
		\item Mean equivalent width of our lines
		  	\begin{equation}
		  	\bar{W_\lambda} = \frac{1+z}{\mathcal{N} \lambda_{\rm rest}} 
		  	\int f(\mnhi,b,z) W_\lambda(N,b) dN db 
	  		\end{equation}
		\item If the lines are randomly distributed (i.e.\ no clustering),
		then the mean transmission ($1-D_A$) is
		\begin{equation}
		1-D_A = \exp (-\mathcal{N} \bar W_\lambda)
		\end{equation}
			\begin{itemize}
			\item This equation appears intuitive
			\item Yet a proper derivation is remarkably complex! (see above references)
			\end{itemize}
		\item An expression for the effective opacity simply follows as
		\begin{equation}
		\tau_{\rm eff, \alpha} = \int f(\mnhi,b,z) 
		W^{\rm Ly\alpha}_\lambda(N,b) \, dN db 
		\label{eqn:teff}
		\end{equation}
		\end{itemize}
	\item Evaluating $\tau_{\rm eff,\alpha}$
		\begin{itemize}
		\item Equation~\ref{eqn:teff} is relatively straightforward, but
		not analytic
		\item A typical cheat is to assume the average $b$-value instead
		of our distribution
		\item This gives a one-to-one correspondence between \nhi\ and $W_\lambda$
		\item See Notebook
		\end{itemize}
	\item Using \fnhi\ from Prochaska+14
		\begin{itemize}
		\item $\tau_{\rm eff, \alpha} = 0.24$ at $z=2.5$
		\item Differential contribution with $\log \mnhi$

	\includegraphics[width=4.5in]{Figures/dteff.pdf}

			\begin{itemize}
			\item Dominated by lines with $\tau_0 \approx 1$
			\item Small contribution from damped \lya\ lines too
			\item Choice of $N_{\rm min}$ also impacts the result
			\end{itemize}

		\end{itemize}

	\item Observational results
		\begin{itemize}
		\item Kirkman et al. 2005
			\begin{itemize}
			\item 24 quasar spectra from Lick Observatory
			\item Army of undergrad students to fit continua 
			(dominant uncertainty)
			\item Masked metal absorption and lines with $\mnhi > 10^{17} \cm{-2}$

	\includegraphics[width=4.5in]{Paper_figs/kirkman05_fig4.pdf}

			\item Modeled $D_A$ as $(1+z)^\gamma$ for $z=2-3$
			\begin{equation}
			D_A = 0.0062 \, (1+z)^{2.75}
			\end{equation}
			\item $\tau_{\rm eff, \alpha}(z=2.5) = 0.22$ 
			\end{itemize}
		\item Faucher-Gigu\`ere et al.\ 2008
			\begin{itemize}
			\item Analyzed nearly 100 echelle spectra at $z>2$
			\item Corrected statistically for metal absorption
			\item Assessed continuum bias on mock spectra
			\item Identified a disturbing `wiggle' in the measurement

	\includegraphics[width=4.5in]{Paper_figs/fg08_fig3.pdf}

			\item Best-fit power law

			\begin{equation}
			\tau_{\rm eff, \alpha} = 0.0018 \, (1+z)^{3.92}
			\end{equation}

			\end{itemize}

		\item Becker et al.\ 2013
			\begin{itemize}
			\item Recognized that the quasar SED is remarkably similar with redshift 

	\includegraphics[width=4.5in]{Paper_figs/becker13_fig2.pdf}

				\begin{itemize}
				\item Implies that a relative measurement may be very precise
				\item Leveraged the statistical power of SDSS
				\item BOSS is not well fluxed...
				\end{itemize}
			\item Measure the ratio of the fluxes (divide the stacked spectra)
			\item See Slides

	\includegraphics[width=4.5in]{Paper_figs/becker13_fig4.pdf}

				\begin{itemize}
				\item Tied absolute value to previous work (F-G et al.\ 2008)
				\item Statistical corrections for $\mnhi > 10^{17} \cm{-2}$ absorbers and metals
				\item Main result (not strictly a power-law)
				\begin{equation}
				\tau_{\rm eff, \alpha}(z) = 0.751 \ltp \frac{1+z}{1+3.5} 
				\rtp^{2.9}  - 0.132
				\end{equation}
				\end{itemize}

			\end{itemize}
		\item These measurements provide a blunt yet powerful test
		for any cosmological model of the IGM
			\begin{itemize}
			\item Density field
			\item Radiation field
			\item See next sub-section
			\end{itemize}
		\end{itemize}

	\end{itemize}

{\bf \item Fluctuating Gunn-Peterson Approximation (FGPA)}
	\begin{itemize}
	\item An alternate description of the \lya\ Forest
	\item Gunn-Peterson 
		\begin{itemize}
		\item Optical depth from \lya\ at line center (Lecture 1)
		\begin{equation}
		\tau_0 = \frac{\pi e^2}{m_e c} \frac{N_{\rm HI} 
		f_{\rm Ly\alpha}}{\Delta \nu_D}
		\end{equation}
		\item \nhi\ column density
		\begin{equation}
		\mnhi = n_{\rm HI} \Delta r
		\end{equation}
		\item Cosmology (see Equation~\ref{eqn:drdz} for $dr/dz$)
		\begin{equation}
		\Delta r = \frac{c \Delta z}{H(z) (1+z)}
		\end{equation}
		\item Recognize absorption occurs over a small redshift interval
		\begin{equation}
		\Delta z/(1+z) = \Delta\nu_{\rm Ly\alpha} / \nu_{\rm Ly\alpha}
		\end{equation}
		\item Ionization balance for $n_{\rm HI}$ with ionization rate $\Gamma$
		and recombination rate $\alpha(T)$
		\begin{equation}
		n_{\rm HI} \Gamma = \alpha(T) n_{\rm HII} n_e
		\label{eqn:ion_balance}
		\end{equation}
		\item Altogether now..
		\begin{equation}
		\tau = \frac{\pi e^2 f_{\rm Ly\alpha}}{m_e \nu_{\rm Ly\alpha}}
		\frac{1}{H(z)} \frac{\alpha(T) n_{\rm HII} n_e}{\Gamma}
		\end{equation}
		\end{itemize}
	\item Introduce the over-density $\delta$
	\begin{equation}
	\rho = \bar\rho (1+\delta)
	\label{eqn:overdensity}
	\end{equation}
		\begin{itemize}
		\item Express our number densities in terms of the over-density
		\item And ionization and mass fractions ($X$ is the Hydrogen mass fraction and $x$ is the Hydrogen ionization fraction)
		\begin{equation}
		n_{\rm HII} = \frac{\rho_{\rm crit} \Omega_b}{m_p} \, X x \; (1+\delta) (1+z)^3
		\end{equation}
		\item A similar (uglier) expression for $n_e$ includes Helium ($Y$, $y_{II}, y_{III}$)
		\begin{equation}
		n_e = \frac{\rho_{\rm crit} \Omega_b}{m_p} [Xx + 0.25Y(y_{II} + 2y_{III})] \; (1+\delta) (1+z)^3
		\end{equation}
		\end{itemize}
	\item Approximate $\alpha(T)$ as a power-law (e.g. Osterbrock)
	\begin{equation}
	\alpha(T) \approx \alpha_0 T^{-0.7}
	\end{equation}
	\item Lastly, assume the IGM gas follows a power-law temperature-density
	relation (e.g. Hui \& Gnedin 1997)
	\begin{equation}
	T = T_0 (1+\delta)^\beta
	\label{eqn:rhoT}
	\end{equation}
	\item Our Gunn-Peterson expression becomes
	\begin{equation}
	\tau = A(z) \, (1+\delta)^{2 - 0.7 \beta}
	\end{equation}
		\begin{itemize}
		\item This opacity ``fluctuates'' with the local over-density
		\item aka the fluctuating Gunn-Peterson approximation
		\item Here
		\begin{equation}
		A(z) \propto (1+z)^6 / [H(z) \Gamma]
		\end{equation}
		\item Or, in its glory
		\begin{equation}
		A(z) \equiv \frac{\pi e^2 f_{\rm Ly\alpha}}{m_e \nu_{\rm Ly\alpha}}
		\ltp \frac{\rho_{\rm crit} \Omega_b}{m_p} \rtp^2 
		\frac{1}{H(z)} Xx [Xx + 0.25Y(y_{II} + 2y_{III})] \frac{\alpha_0 T_0^{-0.7}}{\Gamma}
		(1+z)^6
		\end{equation}
		\end{itemize}
	\item Given a probability density function for the gas density,
	$\Delta \equiv 1 + \delta$, we may calculate the mean flux (and opacity)
	\begin{equation}
	<F>(z) = \intl_0^\infty P(\Delta; z) \exp(-\tau) d\Delta
	\end{equation}
		\begin{itemize}
		\item See Miralda-Escud\'e et al.\ 2000 for models of $P(\Delta)$
		based on numerical simulations
		\end{itemize}
	\item Adopting a cosmology, one may solve for $\Gamma$
		\begin{itemize}
		\item e.g. Faucher-Gigu\`ere et al.\ 2008, ApJ, 682, L9
		\item The resultant estimate is remarkably flat with redshift

	\includegraphics[width=4.5in]{Paper_figs/fg08c_fig1.pdf}

		\item Extending to $z>4$, where quasar emissivity drops sharply,
		one may conclude that galaxies have to dominate
		\item But beware of the mini-quasar! (e.g. Giallongo et al.\ 2015; Haardt \& Madau 2015)
		\item And uncertain assumptions inherent to the fluctuating GP approximation
		\end{itemize}

	\end{itemize}

{\bf \item Effective Lyman Series Opacity}
	\begin{itemize}
	\item Hydrogen Lyman Series
		\begin{itemize}
		\item Imposes additional opacity at shorter wavelengths
		\item Sequentially smaller, $\tau_0 \propto \lambda f N$
		\end{itemize}
	\item At short wavelengths, a complex blending of Lyman lines
	from various redshifts occurs
		\begin{itemize}
		\item Example:  $\lambda_{\rm obs} = 4600$\AA\ for a source
		at $z_{\rm em} = 4$
		\item Ly$\alpha$ opacity from $z_{\rm Ly\alpha} = 2.78$
		\item Ly$\beta$ opacity from $z_{\rm Ly\beta} = 3.48$
		\item Ly$\gamma$ opacity from $z_{\rm Ly\gamma} = 3.73$
		\item Ly$\delta$ opacity from $z_{\rm Ly\delta} = 3.84$
		\item etc.
		\end{itemize}
	\item Approach at each $\lambda$
		\begin{itemize}
		\item Calculate $\tau_{\rm eff}$ from every contributing
		Lyman series line
		\item Sum
		\item Recognize that $\tau{\rm eff}$ for a given line {\it decreases}
		with decreasing redshift (shorter wavelength)
		\end{itemize}
	\item Result
		\begin{itemize}
		\item See {\bf Notebook}
		\item Theoretical `sawtooth' curve:

	\includegraphics[width=5.0in]{Figures/sawtooth.pdf}

		\item Observational ($z \approx 4$ stacked quasar)

	\includegraphics[width=5.0in]{Figures/obs_sawtooth.pdf}
			\begin{itemize}
			\item The agreement between data and model is stunning
			\item Recall: the Telfer+02 quasar SED is from $z=1$
			\item And \fnhi\ was derived from line counting (and LLS
			surveys, next lecture)
			\end{itemize}

		\end{itemize}
	\item Implications
		\begin{itemize}
		\item IGM attenuation of distant sources
			\begin{itemize}
			\item e.g.\ galaxies (now called Lyman break galaxies)
			\item For a $z=4$ galaxy, the $g$-band flux is reduced
			by $\Delta m_g \approx 1$\,mag
			\item This insight (Madau, Steidel, Koo) led to the discovery
			of the first high-$z$ star-forming galaxies 
			\item As we will see in the next Lecture, the IGM has an even
			greater effect beyond the Lyman limit
			\end{itemize}
		\item Further IGM attenuation of distant sources
			\begin{itemize}
			\item This attenuation will also affect our \lya\ emitters (LAEs)
			\item At $z=5$, $\tau_{\rm eff, \alpha} \approx 1.5$ 
			\item Leads to asymmetry in \lya\ emission by the IGM alone
			\end{itemize}
		\end{itemize}

	\end{itemize}

{\bf \item IGM as Collapsed Jeans-Length Clouds (Schaye 2001)}
	\begin{itemize}
	\item Consider a cloud in the IGM that undergoes collapse
	\item Characteristic time-scales:
		\begin{itemize}
		\item Dynamical time-scale
		\begin{equation}
		t_{\rm dyn} \equiv \frac{1}{\sqrt{G \rho}}
		\end{equation}
		\item Sound-crossing time
		\begin{equation}
		t_{\rm sc} \equiv \frac{L}{c_s}
		\end{equation}
		\end{itemize}
	\item Jeans-scale
		\begin{itemize}
		\item Equating the time-scales,
		\begin{equation}
		L_J \equiv \frac{c_s}{\sqrt{G\rho}} \sim 
		0.52 \, {\rm kpc} \, n_{H}^{-1/2} T_4^{1/2} \ltp \frac{f_g}{0.16} \rtp^{1/2}
		\end{equation}
		\item with $f_g \approx \Omega_g/\Omega_b$
		\item Define a Jeans Hydrogen column density
		\begin{equation}
		N_{H,J} \equiv n_H L_J \sim 1.6 \sci{21} \cm{-2} \,
		n_{H}^{-1/2} T_4^{1/2} \ltp \frac{f_g}{0.16} \rtp^{1/2}
		\end{equation}
			\begin{itemize}
			\item Weak dependence on $n_H$ and $T$
			\item Suggests each `cloud' has similar $N_H$ 
			\end{itemize}
		\end{itemize}
	\item Derive a relation for \nhi, our observable
		\begin{itemize}
		\item Assume ionization balance (Equation~\ref{eqn:ion_balance}) to 
		estimate a Jeans HI column density
		\item Also invoke our definition of overdensity (Equation~\ref{eqn:overdensity}) and $\rho-T$ relation (Equation~\ref{eqn:rhoT})
		\begin{equation}
		N_{HI} \sim 2.7\sci{13} \cm{-2} (1+\delta)^{-1.5 - 0.7\beta}
		\ltp \frac{1+z}{4} \rtp^{9/2} \times (\rm other \, simple \, factors)
		\end{equation}
		\end{itemize}


	\item Implications/insights
		\begin{itemize}
		\item Low-density \lya\ forest arises in gas with low over-density 
		\item The steep redshift dependence implies the characteristic
		over-density of a system with fixed \nhi\ increases with decreasing $z$
			\begin{itemize}
			\item Given a fixed sensitivity to \nhi\ (typical)
			\item Higher-$z$ universe probes lower density gas and vice-versa
			\end{itemize}
		\item For fiducial values of the radiation field, one estimates
		that the density in the IGM is at least 
		$\Omega_{\rm IGM} \gtrsim 0.7 \Omega_b$
		\item Differential contribution to $\Omega_g$

	\includegraphics[width=5.0in]{Paper_figs/schaye01_fig2.pdf}

		\end{itemize}

	\end{itemize}

{\bf \item Complete Models of \fnhi [after DLA Lecture]}
	\begin{itemize}
	\item Observational constraints
		\begin{itemize}
		\item \fnhi\ at low \nhi\ from line fitting of the Forest
		\item Incidence constraints for optically thick gas (next Lecture)
		\item Estimate of the mean free path (next Lecture)
		\item \fnhi\ at high \nhi\ from DLA Surveys (later Lecture)
		\end{itemize}
	\item Constraints at $z \approx 2.5$
	\item Prochaska et al.\ 2014
		\begin{itemize}
		\item Adopted a cubic Hermite spline model (monotonically decreasing)
			\begin{itemize}
			\item Not physical, purely mathematical (but captures the data)
			\item And smoother than broken power-laws
			\end{itemize}
		\item MCMC techniques to constrain the parameters
		\item Results (focusing on $z \approx 2.5$)

	\includegraphics[width=5.0in]{Paper_figs/prochaska14_fig7.pdf}

		\end{itemize}
	\item Inoue et al.\ 2014
		\begin{itemize}
		\item Formalism
		\end{itemize}
	\end{itemize}

\end{Aenumerate}

\end{document}
