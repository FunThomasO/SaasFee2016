\documentclass[12pt,letterpaper]{article}

\font\bfit = cmbxti10 scaled \magstephalf

\usepackage{amsmath}
\usepackage{latexsym}
\usepackage{graphicx}
\usepackage{amssymb}
\usepackage{ulem}
%\usepackage{psfig}
\usepackage{fancyheadings}
 
\pretolerance=10000
\textwidth=6.7in
\textheight=8.9in
\topmargin=-0.6in
\headheight=.15in
\hoffset = -0.2in
\headsep=.35in
\oddsidemargin=0in
\evensidemargin=0in
\parindent=2em
\parskip=0.2ex

\newcommand{\nhi}{$N_{\rm HI}$}
\newcommand{\mnhii}{N_{{\rm HI},i}}
\newcommand{\mnhi}{N_{\rm HI}}
\def\mfnhi{f(\mnhi)}
\def\fnhi{$\mfnhi$}
\def\mfnorm{<F>_{\rm norm}}
\def\fnorm{$\mfnorm$}
\def\mlmfp{\lambda_{\rm mfp}^{912}}
\def\lmfp{$\mlmfp$}
\def\mtll{\tau_{\rm eff}^{\rm LL}}
\def\tll{$\mtll$} 
%%%%%
\def\rexp{{\rm e}}
\def\katzt{\ket{\alpha, t_0=0; t}}
\def\katt{\ket{\alpha, t_0; t}}
\def\usc{\mathcal{U}}
\def\tsc{\mathcal{T}}
\special{papersize=8.5in,11in}
\newcommand{\nid}{\noindent}
\newcommand{\ew}{W_\lambda}
\def\aap{A \& A}
\def\aj{AJ}
\def\apj{ApJ}
\def\apss{Ap\&SS}
\def\apjl{ApJL}
\def\apjs{ApJS}
\def\apjsupp{ApJS}
\def\araa{ARAA}
\def\mnras{MNRAS}
\def\nat{Nature}
\def\pasp{PASP}
\def\prd{PhRvD}
\def\hangpara{\par\hangindent 25pt\noindent}
\def\Lya{Ly$\alpha$}
\def\lya{Ly$\alpha$}
\def\Lyb{Ly$\beta$ }
\def\lyb{Ly$\beta$ }
\def\Lyg{Ly$\gamma$ }
\def\Lyd{Ly$\delta$ }
\def\etal{et al. }
\def\kms{km~s$^{-1}$ }
\def\mkms{{\rm \; km\,s\,^{-1}}}
\def\bmkms{{\rm\bf \; km\,s\,^{-1}}}
\def\L{$\lambda$}
\def\s-1{s$^{-1}$}
\def\Hz-1{Hz$^{-1}$}
\def\pohf{\rm p_\ohf}
\def\prhf{\rm p_\rhf}
\def\ohf{\frac{1}{2}}
\def\nohf{\frac{-1}{2}}
\def\rhf{\frac{3}{2}}
\def\sohf{\rm s_\ohf}
\def\esp{\Delta E_{2s, 2p}}
\def\chal{\frac{\Delta \alpha}{\alpha}}
\def\ylab{y_{\rm lab}}
\def\yaY{y \; {\rm and} \; Y}
\def\dadz{\frac{1}{\alpha} |\frac{{\rm d} \alpha}{{\rm d} z}|}
\def\abdadz{\frac{1}{\alpha} \left|\frac{{\rm d} \alpha}{{\rm d} z}\right|}
\def\abdadt{\frac{1}{\alpha} \left|\frac{{\rm d} \alpha}{{\rm d} t}\right|}
\def\sci#1{{\; \times \; 10^{#1}}}
\def\yold{y_{\rm old}}
\def\und#1{{\rm \underline{#1}}}
\def\dlmb{\Delta \lambda}
\def\abchal{| \chal |}
\def\lbar{\bar\lambda}
\def\elc{\rm e^-}
\def\qnn#1{\eqno {\rm {#1}}}
\def\l1l2{\lambda_2 \; {\rm and} \; \lambda_1}
\def\id{\indent}
\def\wid{\id\id}
\def\rid{\id\id\id}
\def\uid{\id\id\id\id}
\def\imp{\, \Rightarrow \>}
\def\di{\partial}
\def\ba{{\bf a}}
\def\bx{{\bf x}}
\def\bv{{\bf v}}
\def\bu{{\bf u}}
\def\bq{{\bf q}}
\def\br{{\bf r}}
\def\bs{{\bf s}}
\def\bp{{\bf p}}
\def\bj{{\bf j}}
\def\bk{{\bf k}}
\def\bm{{\bf m}}
\def\bn{{\bf n}}
\def\bd{{\bf d}}
\def\bg{{\bf g}}
\def\by{{\bf y}}
\def\bB{{\bf B}}
\def\bE{{\bf E}}
\def\bD{{\bf D}}
\def\bH{{\bf H}}
\def\bA{{\bf A}}
\def\bI{{\bf I}}
\def\bJ{{\bf J}}
\def\bK{{\bf K}}
\def\bL{{\bf L}}
\def\bS{{\bf S}}
\def\bF{{\bf F}}
\def\bN{{\bf N}}
\def\bP{{\bf P}}
\def\bR{{\bf R}}
\def\bU{{\bf U}}
\def\bM{{\bf M}}
\def\bY{{\bf Y}}
\def\bOm{{\bf \Omega}}
\def\bom{{\bf \omega}}
\def\bell{{\bf \ell}}
\def\di{\partial}
\def\ddi#1#2{\frac{\di #1}{\di #2}}
\def\gmm{\sqrt{1 - \frac{v^2}{c^2}}}
\def\eng{\varepsilon}
\def\dd#1#2{\frac{d #1}{d #2}}
\def\dt#1{\frac{d #1}{dt}}
\def\ddt{\frac{d}{d t}}
\def\ddit{\frac{\di}{\di t}}
\def\arr{\to \quad}
\def\smm{\sum\limits}
\def\intl{\int\limits}
\def\ointl{\oint\limits}
\def\liml{\lim\limits}
\def\prodl{\prod\limits}
\def\grd{{\bf \nabla}}
\def\crl{{\bf \grd} \times}
\def\dv{{\bf \grd} \cdot}
\def\ec{\frac{e}{c} \,}
\def\oc{\frac{1}{c}}
\def\xht{{\; \rm \hat x}}
\def\yht{{\; \rm \hat y}}
\def\zht{{\; \rm \hat z}}
\def\rht{{\; \rm \hat r}}
\def\pht{{\; \rm \hat \phi}}
\def\iht{{\; \rm \hat i}}
\def\jht{{\; \rm \hat j}}
\def\kht{{\; \rm \hat k}}
\def\theht{{\; \rm \hat \theta}}
\def\nht{{\; \rm \hat n}}
\def\oht{{\; \rm \hat 1}}
\def\tht{{\; \rm \hat 2}}
\def\vht{{\; \rm \hat v}}
\def\thht{{\; \rm \hat 3}}
\def\Rht{{\; \rm \hat R}}
\def\cyc{\Omega_c}
\def\dl{{d {\bf \ell}}}
\def\da{{\rm d {\bf a}}}
\def\ds{{\rm d {\bf s}}}
\def\smt{{\smll \rm T}}
\def\opi#1{\frac{1}{#1 \pi}}
\def\strs{\underline {\underline {\rm T}}}
\def\dtr{\, d^3 r}
\def\dftr{d^3 r \,}
\def\dtp{\, d^3 p}
\def\dftp{d^3 p \,}
\def\abrr{|\br - \br'|}
\def\rr{(\br - \br')}
\def\orr#1{\frac{#1}{\abrr}}
\def\trr#1{\frac{#1}{\abrr^3}}
\def\drr{\delta \rr}
\def\Grn{G(\br, \br')}
\def\Ylm{Y_{\ell m}}
\def\snth{\sin\theta}
\def\csth{\cos\theta}
\def\invr#1{\frac{1}{#1}}
\def\fpc{\frac{4 \pi}{c}}
\def\lgnl{P_\ell (\csth)}
\def\lgnn#1{{P_{#1} (\csth)}}
\def\thph{\theta, \phi}
\def\eht1{{\rm \hat e_1}}
\def\ehtb{{\rm \hat e_2}}
\def\iot{i \omega t}
\def\kr{\bk \cdot \br}
\def\ele{{\rm e^-}}
\def\iwc{\frac{i \omega}{c}}
\def\iwt{i \omega t}
\def\plr{{\rm \hat \epsilon}}
\def\skp{\smallskip}
\def\mkp{\medskip}
\def\bkp{\bigskip}
\def\elc{\rm e^{-s}}
\def\bmx{\left[\matrix}
\def\Det{{\rm Det \,}}
\def\bra#1{{<\negmedspace#1|}}
\def\ket#1{{|#1\negmedspace>}}
\def\kalp{\ket{\alpha}}
\def\kbet{\ket{\beta}}
\def\balp{\bra{\alpha}}
\def\bbet{\bra{\beta}}
\def\inpr#1#2
\def\otpr#1#2{{|#1><#2|}}
\def\dgg{\dagger}
\def\dual{\Leftrightarrow}
\def\hbm{\frac{\hbar^2}{2 m}}
\def\hbu{\frac{\hbar^2}{2 \mu}}
\def\mbh{\frac{2 m}{\hbar^2}}
\def\ih{\frac{i}{\hbar}}
\def\kjm{\ket{j m}}
\def\bjm{\bra{j m}}
\def\bjmp{\bra{j' m'}}
\def\ylm{Y_{\ell m}}
\def\pof#1{(\frac{#1}{2})}
\def\tqk{T_q^k}
\def\ptm{\frac{p^2}{2 m}}
\def\psn{\psi_n}
\def\tph{\frac{2 \pi}{\hbar}}
\def\plr{{\hat \epsilon}}
\def\Veff{V_{\rm eff}}
\def\rhs{\rho^{(s)}}
\def\rhis{\rho_i^{(s)}}
\def\tilH{\tilde H}
\def\tpi{2 \pi i}
\def\prin{\intl_P}
\def\lta{\left < \,}
\def\rta{\right > \,}
\def\ltk{\left [ \,}
\def\ltp{\left ( \,}
\def\ltb{\left \{ \,}
\def\rtk{\, \right  ] }
\def\rtp{\, \right  ) }
\def\rtb{\, \right \} }
\def\dc{\tilde d}
\def\bskp{\bigskip}
\def\nV{\Omega}
\def\d3x{d^3 x}
\def\mtI{\b{\b I}}
\def\cmma{\;\;\; ,}
\def\perd{\;\;\; .}
\def\delv{\Delta v}
\def\eikr{{\rm e}^{i \bk \cdot \br}}
\def\rmikr{{\rm e}^{i \vec k \cdot \vec r}}
\def\rmnikr{{\rm e}^{-i \vec k \cdot \vec r}}
\def\eikx{{\rm e}^{i \bk \cdot \bx}}
\def\muo{\mu_0}
\def\epo{\epsilon_0}
\def\rme{{\rm e}}
\def\dnu{\Delta \nu_D}
\def\rAA{\AA \enskip}
\def\halp{{\rm H_\alpha}}
\def\hbet{{\rm H_\beta}}
\def\N#1{{N({\rm #1})}}
\def\ekz{\rme^{i ( k_z z - \omega t )}}
\def\cm#1{\, {\rm cm^{#1}}}
\def\bcm#1{\; {\rm \mathbf{cm^{#1}}}}
\def\rmK{\; \rm K}
\def\rmm#1{\; {\rm #1}}
\def\Nperp{N_\perp}
\def\spkr#1{{\big \bf {#1}}}
\def\inA{\quad $\bullet$ \quad}
\def\inB{\quad\quad $\diamond$ \quad}
\def\inC{\quad\quad\quad $\circ$ \quad}
\def\sigfc{\sigma_{f_c}}
\def\sech{\rm sech}
\def\blck{\color[named]{Black}}
\def\blue{\color[named]{Blue}}
\def\skyblue{\color[named]{SkyBlue}}
\def\red{\color[named]{Red}}
\def\black{\color[named]{Black}}
\def\green{\color[named]{Green}}
\def\white{\color[named]{White}}
\def\yellow{\color[named]{Yellow}}
\def\purple{\color[named]{Purple}}
\def\maroon{\color[named]{Maroon}}
\def\redv{\color[named]{RedViolet}}
\def\mahog{\color[named]{Mahogany}}
\def\prb{$\frac{\delta \rho}{\rho}$}
\def\trd{\frac{2}{3}}
\def\mdn{\bar \rho}
\def\ss{v_s^2}
\def\msol{M_\odot}
\def\msun{M_\odot}
\def\W#1{{W({\rm #1})}}
\def\Nav{N_a (v)}
\def\f#1{{f_{\rm #1}}}
\def\opt{$d \tau / d z \; $}
\def\btau{\bar\tau (v_{pk}) / \sigma (\bar \tau)}
\def\Ipk{I(v_{pk})/I_{c}}
\def\vrot{v_{rot}}
\def\sphr{\sqrt{R^2 + Z^2}}
\def\mbf{\mathbf}
\def\mgsqa{\, {\rm mag} / \Box ''}
\def\sqas{$\Box ''$}
\def\msqas{\Box ''}
\def\sqasec{$\Box ''$}
\newcommand{\ialii}{Al{\sc II}}
\newcommand{\iari}{Ar{\sc I}}
\newcommand{\ini}{N{\sc I}}
\newcommand{\ioi}{O{\sc I}}
\newcommand{\iovi}{O{\sc VI}}
\newcommand{\iovii}{O{\sc VII}}
\newcommand{\ioviii}{O{\sc VIII}}
\newcommand{\ifeii}{Fe{\sc II}}
\newcommand{\ifeiii}{Fe{\sc III}}
\newcommand{\ispii}{S{\sc II}}
\newcommand{\iznii}{Zn{\sc II}}
\newcommand{\icaii}{Ca{\sc II}}
\newcommand{\inai}{Na{\sc I}}
\newcommand{\icii}{C{\sc II}}
\newcommand{\iciv}{C{\sc IV}}
\newcommand{\ihii}{H{\sc II}}
\newcommand{\ihi}{H{\sc I}}
\newcommand{\isiii}{Si{\sc II}}
\newcommand{\isiiv}{Si{\sc IV}}
\newcommand{\ici}{C{\sc I}}
\newcommand{\dcost}{d \negmedspace \cos\theta}
\newcommand{\mnvhi}{n_{\rm H {\sc I}}}
\newcommand{\nvhi}{$n_{\rm H {\sc I}}$}
\newcommand{\rem}{{\rm e}}
\def\otpr#1#2{{|#1><#2|}}
\def\avgq#1{<\negmedspace#1\negmedspace>}
\newcommand{\negm}{\negmedspace}
\newcommand{\dfhf}{^2\negm D_{\frac{5}{2}}}
\newcommand{\negt}{\negthinspace}
\newcommand{\nalph}{N/$\alpha$}
\newcommand{\alphh}{$\alpha$/H}
\newcommand{\Rearth}{R_\oplus}
\newcommand{\mhmpc}{h^{-1} \, \rm Mpc}
\newcommand{\hmpc}{$\mhmpc$}

\renewcommand{\descriptionlabel}[1]%
    {\hspace{\labelsep}\textsf{#1}}

\newenvironment{Aitemize}{%
 \renewcommand{\labelitemi}{$\bullet \;$}%
	\begin{itemize}}{\end{itemize}}

\newenvironment{Renumerate}{%
 \renewcommand{\theenumi}{\Roman{enumi}}
 \renewcommand{\labelenumi}{\theenumi)}
	\begin{enumerate}}{\end{enumerate}}

\newenvironment{Aenumerate}{%
 \renewcommand{\theenumi}{\Alph{enumi}}
 \renewcommand{\labelenumi}{\theenumi.}
	\begin{enumerate}}{\end{enumerate}}

\newenvironment{dmnd}{%
 \renewcommand{\labelitemi}{$\diamond \quad$}%
	\begin{itemize}}{\end{itemize}}

\newenvironment{Penumerate}{%
 \renewcommand{\labelenumi}{(\theenumi)}
	\begin{enumerate}}{\end{enumerate}}


\pagestyle{fancyplain}

\lhead[\fancyplain{}{Intro}]{\fancyplain{}{\thepage}}
\rhead[\fancyplain{}{Intro-\thepage}]{\fancyplain{}{SaasFee16-Spectral Analysis}}
\cfoot{}

\newcommand{\mndi}{N_{\rm DI}}
\newcommand{\ndi}{$N_{\rm DI}$}
\newcommand{\helium}{${^4\rm He}$}
\newcommand{\rmch}{\frac{\rm C}{\rm H}}
\renewcommand{\labelitemii}{$\diamond$}
\renewcommand{\labelitemiii}{$\blacktriangle$}
\renewcommand{\labelitemiv}{$\circ$}

%\newcommand{\dfhf}{^2\negm D_{\frac{5}{2}}}
%\newcommand{\dtvr}{\dot{\vec r}}
%\newcommand{\mtrx}{\bra{\phi_f} \rmikr \hat\eng \cdot \vec\nabla \ket{\phi_i}}
%\newcommand{\prfi}{\ltp \hat\eng \cdot \vec r \rtp_{fi}}


\special{papersize=8.5in,11in}

\begin{document}

\noindent {\bf{\large II. Spectral Analysis [A Few Basics] (v2.0)}}

\begin{Aenumerate}

{\bf \item Goals}
 \begin{itemize}
 \item Provide a few useful definitions on spectroscopy
 \item Introduce a few critical aspects of the spectral analysis
 for absorption
 \end{itemize}

{\bf \item Spectral Characteristics}
 \begin{itemize}
  \item Line Spread Function (LSF)
  	\begin{itemize}
  	\item Definition: Functional form a monochromatic source exhibits 
  	on the detector of a spectrometer
  	\item Diffraction of light (e.g.\ slit + dispersing element)
  	transforms the input PSF (typically a Moffat) into a specific 
  	profile in wavelength
  	\item LSF depends on type of spectrometer (e.g. Robertson 2013):

\begin{table}[ht]
\begin{center}
\caption{{\sc LSFs}}
\label{tab:F}
\vskip 0.05in
\begin{tabular}{ccc}
\hline
Type of Spectrograph & LSF & R \\
\hline
Diffraction-limited slit & sinc$^2$     & $mN$ \\
Single-mode fiber        & Gaussian     & FWHM \\
Multi-mode fiber         & Half ellipse & ??\\ 
Fabry-Perot etalon       & Lorentzian   & $mF$ \\
\hline
\end{tabular}
\end{center}
\end{table}

  	\item Comparing two standard LSFs:

	\includegraphics[scale=0.60]{Figures/lsf.pdf}

  	\item System imperfections + Central Limit Theorem pushes the LSF 
  	towards a Gaussian
  		\begin{itemize}
  		\item For this school, we will assume a Gaussian LSF
  		\item But caution that careful analysis must measure the LSF
  		(e.g.\ atmospheric absorption lines)
  		\end{itemize}

  	\end{itemize}
  \item Resolution: $R$ 
  	\begin{itemize}
  	\item Definition: Characterization of the width of the LSF
  	\item Dependent on the LSF (see Table; Robertson 2013)
  		\begin{itemize}
  		\item For our discussion, use FWHM of the LSF
  		\item In km/s (or \AA\ if you insist)
  		\end{itemize}
	\item Dependent on slit (or fiber) width
	\item Dependent on dispersing element (grating)
  	\item $R$ is {\it not} dependent on the characteristics of the detector

	\begin{equation}
	R = \frac{\lambda}{\Delta\lambda_{\rm FWHM}}
	\end{equation}
  	\item $R$ values 
  		\begin{itemize}
  		\item SDSS: $R \approx 2000$
  		\item Echelle (HIRES, UVES): $R > 30000$
  		\item Echellette (ESI, X-Shooter): $R \approx 8000$
  		\end{itemize}
	\item Can express in terms of velocity:  $c/R$
		\begin{itemize}
		\item e.g. an echelle spectrograph has a ``resolution'' of $\approx 10$km/s
		\end{itemize}
	\item Old-timers often refer to the FWHM in \AA, e.g. $2$\AA
  	\end{itemize}
  \item Dispersion: $\delta\lambda$
  	\begin{itemize}
  	\item Definition: Spectral width of one pixel (native or binned)
  	in the recorded spectral image
  	\item Highly dependent on characteristics of the detector
  		\begin{itemize}
  		\item Size of native pixels (e.g.\ $15 \mu$m)
  		\item Binning on the detector
  		\end{itemize}
  	\item Many astronomers refer to dispersion as the Resolution!
  	\end{itemize}

  \item Sampling
  	\begin{itemize}
  	\item Number of pixels per resolution element
  	\item i.e.\  $\Delta\lambda_{\rm FWHM} / \delta\lambda$
  	\item Need 2 or greater for the Nyquist limit
  	\end{itemize}

  \item Signal-to-Noise (S/N)
  	\begin{itemize}
  	\item Ratio of the observed flux to its uncertainty
  	\item May be specified per pixel or resolution element
  	\end{itemize}
  \item Examples of varying spectral characteristics (3 spectra of FJ0812+32)

	\includegraphics[scale=0.80]{Figures/fj0812.pdf}

 \end{itemize}

{\bf \item Continuum Normalization}
 \begin{itemize}
 \item Studies of HI absorption require a background, light source
 \item With emission at FUV wavelengths (and bluer!)

	\includegraphics[scale=0.70]{Figures/qso_fuv.pdf}

	\includegraphics[scale=0.70]{Figures/qso_absorbed.pdf}

 	\begin{itemize}
 	\item Quasar
 	\item GRB
 	\item SF galaxy
 	\end{itemize}
 \item Quasar SEDs
 	\begin{itemize}
 	\item Significant source to source diversity
 		\begin{itemize}
 		\item e.g. BALs
	 	\item Varying spectral slope (intrinsic, dust)
 		\end{itemize}
 	\item And no precise physical model
 		\begin{itemize}
 		\item Synchrotron emission
 		\item Optically thick emission lines (in a wind)
 		\item Hot accretion disk with UV emission
 		\end{itemize}

	\includegraphics[scale=0.70]{Figures/qso_sed.pdf}

 	\item Emission is poorly constrained at $\lambda_{\rm rest} < 900$\AA
 	(e.g.\ Lusso et al.\ 2015)
 		\begin{itemize}
 		\item Limited physical intuition
 		\item IGM masks the SED
 		\end{itemize}
 	\end{itemize}

 \item Quasar composite template
 	\begin{itemize}
 	\item Composite SED 
 		\begin{itemize}
 		\item HST: Telfer et al.\ 2001
 		\item SDSS: Van den Berk et al.\ 2001; Richards et al.\ 2006
 		\end{itemize}

	\includegraphics[scale=0.70]{Figures/qso_template.pdf}

 	\item Remarkable uniformity with redshift (Becker+13; Figure 2)
 		\begin{itemize}
 		\item Small differences in emission lines
 		\item Otherwise nearly identical in the FUV
 		\end{itemize}

	\includegraphics[width=6.0in]{Paper_figs/becker13_fig2.pdf}

 	\end{itemize}

 \item Continuum fitting (Historical)
 	\begin{itemize}
 	\item Fit to unabsorbed regions (high-order Legendre polynomial)
 	\item By-eye estimate in the \lya\ Forest (spline)
 		\begin{itemize}
 		\item At $z>3$, IGM absorbs at nearly every pixel!
 		\item Guess-work at its finest
 		\end{itemize}
 	\item 5\% error redward of \lya; at least 10\% within the Forest
 		\begin{itemize}
 		\item Error on absorption-line analysis is less for $\tau_0 \sim 1$ lines
 		\item But dominant for $\tau_0 \ll 1$
 		\end{itemize}
 	\item See Notebook for a fitting example
 	\end{itemize}

 \item Principal Component Analysis
 	\begin{itemize}
 	\item QSO emission varies in several, systematic manners
 	  \begin{itemize}
 	  \item Spectral tilt (blue/red)
 	  \item Emission line strength and width
 	  \end{itemize}
 	\item Describe with several eigenvectors (Principal Components)
	\item Suzuki 2006
		\begin{itemize}
		\item Formalism
 	    \begin{equation}
 	    \ket{q_i} = \ket{\mu} + \smm_{j=1}^{m} c_{ij} \ket{\xi_j}
 	    \end{equation}
 	    	\begin{itemize}
	 	    \item $\ket{q_i}$ is any quasar spectrum
	 	    \item $\ket{\mu}$ is the mean spectrum
	 	    \item $\ket{\xi_j}$ are the Principal Components
	 	    \end{itemize}

	\includegraphics[width=6.0in]{Paper_figs/suzuki06_fig2.pdf}

	 	\item First 3 components span $>80\%$ of the variance
	 	\item Derive $\ket{\mu}$ and $\ket{\xi_j}$ from 
	 	$z < 1$ quasars where IGM absorption is minimal
	 		\begin{itemize}
		 	\item Apply to quasars at higher redshift
		 	\item i.e.\ estimate the continuum in the \lya\ forest 
	 		\end{itemize}
		\end{itemize}

	\item Modern analysis: Paris et al.\ 2011; Figure 3

	\includegraphics[width=6.0in]{Paper_figs/paris11_fig3.pdf}

	  \begin{itemize}
	  \item Looks reasonable
	  \item But the flux is likely never unabsorbed in the
	  $z \sim 2.5$ \lya\ forest
	  \end{itemize}

	\item Mean Flux Regulation (Lee, Suzuki, \& Spergel 2011) 
	  \begin{itemize}
	  \item Cosmological analysis of the \lya\ Forest may often ignore the
	  mean absorption
	  \item Approach:  Perform PCA analysis and renormalize in the \lya\ Forest
	  to yield the average flux

	\includegraphics[width=6.0in]{Paper_figs/lee11_fig6.pdf}

	  \end{itemize}

 	\end{itemize}
 \end{itemize}

{\bf \item Redshift}
	\begin{itemize}
	\item While we all understand the concept of redshift, its effects on 
	absorption-line analysis are somewhat subtle
	\item Redshift of the background source is easy enough

	\begin{equation}
	\lambda_{\rm obs}^{\rm source} = (1+z_{\rm source}) \, \lambda_{\rm rest}^{\rm source}
	\end{equation}
		\begin{itemize}
		\item Generating the Ly$\alpha$ forest is a bit less obvious
		\item See Figures/evolving\_forest.pdf 
		\end{itemize}
 	\end{itemize}

{\bf \item Equivalent Width Analysis}
	\begin{itemize}
	\item Boxcar integration
		\begin{itemize}
		\item Simple summation over the pixels covering the absorption line
		\item Requires continuum normalized spectrum 
		(express the normalized flux as $\bar f$)
		\begin{equation}
		W_\lambda = \smm_i (1- \bar f_i) \delta\lambda_i
		\end{equation}
		\item Uncertainty -- simple propagation of error (neglecting
		continuum uncertainty)
		\begin{equation}
		\sigma^2(W_\lambda) = \smm_i \sigma^2(\bar f_i) (\delta\lambda_i)^2
		\label{eqn:sigEW}
		\end{equation}
		\end{itemize}
	\item Line-fitting
		\begin{itemize}
		\item Aside from damped \lya\ lines, the profile is well
		approximated by a Gaussian  (upside-down)
		\item Fit to $(1-\bar f)$:
		\begin{equation}
		g(\lambda) = A \exp[- (\lambda-\lambda_0)^2 / (2 \sigma_\lambda^2)]
		\end{equation}
		\item Equivalent-width is simply area under the curve
		\begin{equation}
		W_\lambda = A \sigma_\lambda \sqrt{2 \pi}
		\end{equation}
		\item Uncertainty should include co-variance between $A$ and $\sigma$
		\begin{equation}
		\sigma(W_\lambda) = W_\lambda \sqrt{\sigma^2(A)/A^2 +
		\sigma^2(\sigma_\lambda)/\sigma_\lambda^2 + 2 \sigma(A)\sigma(\sigma_\lambda)/[A \sigma_\lambda]}
		\end{equation}
		\end{itemize}
	\item See Notebook
	\item Rest-frame
		\begin{itemize}
		\item Analysis is generally performed in the observed (measured) frame
		\item Observed $W_\lambda$ depends on $\lambda$ and therefore redshift
		\item Relating $W_\lambda$ to physical quantities ($N,b$) 
		requires a shift to the rest-frame
		\begin{equation}
		W_\lambda^{\rm rest} = \frac{W_\lambda^{\rm obs}}{1+z}
		\end{equation}
		\end{itemize}
	\item Limiting $W_\lambda$
		\begin{itemize}
		\item Ask: What is the limiting equivalent width one can measure from
		data with given spectral characteristics?
		\item Invert the situation
		\item Return to our boxcar estimate of the uncertainty 
		(Equation~\ref{eqn:sigEW})
			\begin{itemize}
			\item Assume constant dispersion $\delta\lambda$
			\item Recognize that $\sigma(\bar f)$ is the inverse of
			the S/N (per pixel) for normalized data
			\item Let $M$ be the number of pixels covered by the line
			\end{itemize}
		\begin{equation}
		\sigma(W_\lambda) = \frac{\sqrt{M} \delta\lambda}{S/N}
		\end{equation}
		\item Key: How many pixels do we integrate over?
			\begin{itemize}
			\item If the line is unresolved (or barely resolved), 
			$M$ is simply the sampling
			\item Otherwise, $M$ depends on the line-width, i.e. $b$-value
			\end{itemize}
		\item Of course, we generally are interested in a 3$\sigma$
		or 5$\sigma$ limit for a proper detection
		\begin{equation}
		W_{\rm lim}  = 3 \, \sigma(W_\lambda)
		\end{equation}
		\item Example (Ly$\alpha$ in SDSS) -- See Notebook
			\begin{itemize}
			\item For $R=2,000$, S/N=10 per pixel, sampling=2 pixels
			\item $W_{\rm lim}^{3\sigma} \approx 1$\AA
			\item This falls on the saturated portion of the COG for \lya
			\item Poor sensitivity to \nhi\ in such spectra (other than DLAs)
			\end{itemize}
		\end{itemize}
	\end{itemize}

{\bf \item Line Profile Analysis}
	\begin{itemize}
	\item Motivation
		\begin{itemize}
		\item Derive physical parameters from spectral observation
		\item i.e.\ $N$, $b$, $z$ 
		\end{itemize}
	\item Simple least-squares Voigt model  (see Notebook)
	\item Sophisticated profile-fitting packages
		\begin{itemize}
		\item Full Lyman series modeling
		\item $\chi^2$ minimization
		\item VPFIT (R. Carswell):  {\tt http://www.ast.cam.ac.uk/$\sim$rfc/vpfit.html}
			\begin{itemize}
			\item Tried and true
			\item FORTRAN based
			\end{itemize}
		\item ALIS (R. Cooke): {\tt https://github.com/rcooke-ast/ALIS}
			\begin{itemize}
			\item Python based
			\item Will continue to develop
			\end{itemize}
		\end{itemize}
	\end{itemize}

\end{Aenumerate}

\end{document}
