\documentclass[12pt,letterpaper]{article}

\font\bfit = cmbxti10 scaled \magstephalf

\usepackage{amsmath}
\usepackage{latexsym}
\usepackage{graphicx}
\usepackage{amssymb}
\usepackage{ulem}
%\usepackage{psfig}
\usepackage{fancyheadings}
 
\pretolerance=10000
\textwidth=6.7in
\textheight=8.9in
\topmargin=-0.6in
\headheight=.15in
\hoffset = -0.2in
\headsep=.35in
\oddsidemargin=0in
\evensidemargin=0in
\parindent=2em
\parskip=0.2ex

\newcommand{\nhi}{$N_{\rm HI}$}
\newcommand{\mnhii}{N_{{\rm HI},i}}
\newcommand{\mnhi}{N_{\rm HI}}
\def\mfnhi{f(\mnhi)}
\def\fnhi{$\mfnhi$}
\def\mfnorm{<F>_{\rm norm}}
\def\fnorm{$\mfnorm$}
\def\mlmfp{\lambda_{\rm mfp}^{912}}
\def\lmfp{$\mlmfp$}
\def\mtll{\tau_{\rm eff}^{\rm LL}}
\def\tll{$\mtll$} 
%%%%%
\def\rexp{{\rm e}}
\def\katzt{\ket{\alpha, t_0=0; t}}
\def\katt{\ket{\alpha, t_0; t}}
\def\usc{\mathcal{U}}
\def\tsc{\mathcal{T}}
\special{papersize=8.5in,11in}
\newcommand{\nid}{\noindent}
\newcommand{\ew}{W_\lambda}
\def\aap{A \& A}
\def\aj{AJ}
\def\apj{ApJ}
\def\apss{Ap\&SS}
\def\apjl{ApJL}
\def\apjs{ApJS}
\def\apjsupp{ApJS}
\def\araa{ARAA}
\def\mnras{MNRAS}
\def\nat{Nature}
\def\pasp{PASP}
\def\prd{PhRvD}
\def\hangpara{\par\hangindent 25pt\noindent}
\def\Lya{Ly$\alpha$}
\def\lya{Ly$\alpha$}
\def\Lyb{Ly$\beta$ }
\def\lyb{Ly$\beta$ }
\def\Lyg{Ly$\gamma$ }
\def\Lyd{Ly$\delta$ }
\def\etal{et al. }
\def\kms{km~s$^{-1}$ }
\def\mkms{{\rm \; km\,s\,^{-1}}}
\def\bmkms{{\rm\bf \; km\,s\,^{-1}}}
\def\L{$\lambda$}
\def\s-1{s$^{-1}$}
\def\Hz-1{Hz$^{-1}$}
\def\pohf{\rm p_\ohf}
\def\prhf{\rm p_\rhf}
\def\ohf{\frac{1}{2}}
\def\nohf{\frac{-1}{2}}
\def\rhf{\frac{3}{2}}
\def\sohf{\rm s_\ohf}
\def\esp{\Delta E_{2s, 2p}}
\def\chal{\frac{\Delta \alpha}{\alpha}}
\def\ylab{y_{\rm lab}}
\def\yaY{y \; {\rm and} \; Y}
\def\dadz{\frac{1}{\alpha} |\frac{{\rm d} \alpha}{{\rm d} z}|}
\def\abdadz{\frac{1}{\alpha} \left|\frac{{\rm d} \alpha}{{\rm d} z}\right|}
\def\abdadt{\frac{1}{\alpha} \left|\frac{{\rm d} \alpha}{{\rm d} t}\right|}
\def\sci#1{{\; \times \; 10^{#1}}}
\def\yold{y_{\rm old}}
\def\und#1{{\rm \underline{#1}}}
\def\dlmb{\Delta \lambda}
\def\abchal{| \chal |}
\def\lbar{\bar\lambda}
\def\elc{\rm e^-}
\def\qnn#1{\eqno {\rm {#1}}}
\def\l1l2{\lambda_2 \; {\rm and} \; \lambda_1}
\def\id{\indent}
\def\wid{\id\id}
\def\rid{\id\id\id}
\def\uid{\id\id\id\id}
\def\imp{\, \Rightarrow \>}
\def\di{\partial}
\def\ba{{\bf a}}
\def\bx{{\bf x}}
\def\bv{{\bf v}}
\def\bu{{\bf u}}
\def\bq{{\bf q}}
\def\br{{\bf r}}
\def\bs{{\bf s}}
\def\bp{{\bf p}}
\def\bj{{\bf j}}
\def\bk{{\bf k}}
\def\bm{{\bf m}}
\def\bn{{\bf n}}
\def\bd{{\bf d}}
\def\bg{{\bf g}}
\def\by{{\bf y}}
\def\bB{{\bf B}}
\def\bE{{\bf E}}
\def\bD{{\bf D}}
\def\bH{{\bf H}}
\def\bA{{\bf A}}
\def\bI{{\bf I}}
\def\bJ{{\bf J}}
\def\bK{{\bf K}}
\def\bL{{\bf L}}
\def\bS{{\bf S}}
\def\bF{{\bf F}}
\def\bN{{\bf N}}
\def\bP{{\bf P}}
\def\bR{{\bf R}}
\def\bU{{\bf U}}
\def\bM{{\bf M}}
\def\bY{{\bf Y}}
\def\bOm{{\bf \Omega}}
\def\bom{{\bf \omega}}
\def\bell{{\bf \ell}}
\def\di{\partial}
\def\ddi#1#2{\frac{\di #1}{\di #2}}
\def\gmm{\sqrt{1 - \frac{v^2}{c^2}}}
\def\eng{\varepsilon}
\def\dd#1#2{\frac{d #1}{d #2}}
\def\dt#1{\frac{d #1}{dt}}
\def\ddt{\frac{d}{d t}}
\def\ddit{\frac{\di}{\di t}}
\def\arr{\to \quad}
\def\smm{\sum\limits}
\def\intl{\int\limits}
\def\ointl{\oint\limits}
\def\liml{\lim\limits}
\def\prodl{\prod\limits}
\def\grd{{\bf \nabla}}
\def\crl{{\bf \grd} \times}
\def\dv{{\bf \grd} \cdot}
\def\ec{\frac{e}{c} \,}
\def\oc{\frac{1}{c}}
\def\xht{{\; \rm \hat x}}
\def\yht{{\; \rm \hat y}}
\def\zht{{\; \rm \hat z}}
\def\rht{{\; \rm \hat r}}
\def\pht{{\; \rm \hat \phi}}
\def\iht{{\; \rm \hat i}}
\def\jht{{\; \rm \hat j}}
\def\kht{{\; \rm \hat k}}
\def\theht{{\; \rm \hat \theta}}
\def\nht{{\; \rm \hat n}}
\def\oht{{\; \rm \hat 1}}
\def\tht{{\; \rm \hat 2}}
\def\vht{{\; \rm \hat v}}
\def\thht{{\; \rm \hat 3}}
\def\Rht{{\; \rm \hat R}}
\def\cyc{\Omega_c}
\def\dl{{d {\bf \ell}}}
\def\da{{\rm d {\bf a}}}
\def\ds{{\rm d {\bf s}}}
\def\smt{{\smll \rm T}}
\def\opi#1{\frac{1}{#1 \pi}}
\def\strs{\underline {\underline {\rm T}}}
\def\dtr{\, d^3 r}
\def\dftr{d^3 r \,}
\def\dtp{\, d^3 p}
\def\dftp{d^3 p \,}
\def\abrr{|\br - \br'|}
\def\rr{(\br - \br')}
\def\orr#1{\frac{#1}{\abrr}}
\def\trr#1{\frac{#1}{\abrr^3}}
\def\drr{\delta \rr}
\def\Grn{G(\br, \br')}
\def\Ylm{Y_{\ell m}}
\def\snth{\sin\theta}
\def\csth{\cos\theta}
\def\invr#1{\frac{1}{#1}}
\def\fpc{\frac{4 \pi}{c}}
\def\lgnl{P_\ell (\csth)}
\def\lgnn#1{{P_{#1} (\csth)}}
\def\thph{\theta, \phi}
\def\eht1{{\rm \hat e_1}}
\def\ehtb{{\rm \hat e_2}}
\def\iot{i \omega t}
\def\kr{\bk \cdot \br}
\def\ele{{\rm e^-}}
\def\iwc{\frac{i \omega}{c}}
\def\iwt{i \omega t}
\def\plr{{\rm \hat \epsilon}}
\def\skp{\smallskip}
\def\mkp{\medskip}
\def\bkp{\bigskip}
\def\elc{\rm e^{-s}}
\def\bmx{\left[\matrix}
\def\Det{{\rm Det \,}}
\def\bra#1{{<\negmedspace#1|}}
\def\ket#1{{|#1\negmedspace>}}
\def\kalp{\ket{\alpha}}
\def\kbet{\ket{\beta}}
\def\balp{\bra{\alpha}}
\def\bbet{\bra{\beta}}
\def\inpr#1#2
\def\otpr#1#2{{|#1><#2|}}
\def\dgg{\dagger}
\def\dual{\Leftrightarrow}
\def\hbm{\frac{\hbar^2}{2 m}}
\def\hbu{\frac{\hbar^2}{2 \mu}}
\def\mbh{\frac{2 m}{\hbar^2}}
\def\ih{\frac{i}{\hbar}}
\def\kjm{\ket{j m}}
\def\bjm{\bra{j m}}
\def\bjmp{\bra{j' m'}}
\def\ylm{Y_{\ell m}}
\def\pof#1{(\frac{#1}{2})}
\def\tqk{T_q^k}
\def\ptm{\frac{p^2}{2 m}}
\def\psn{\psi_n}
\def\tph{\frac{2 \pi}{\hbar}}
\def\plr{{\hat \epsilon}}
\def\Veff{V_{\rm eff}}
\def\rhs{\rho^{(s)}}
\def\rhis{\rho_i^{(s)}}
\def\tilH{\tilde H}
\def\tpi{2 \pi i}
\def\prin{\intl_P}
\def\lta{\left < \,}
\def\rta{\right > \,}
\def\ltk{\left [ \,}
\def\ltp{\left ( \,}
\def\ltb{\left \{ \,}
\def\rtk{\, \right  ] }
\def\rtp{\, \right  ) }
\def\rtb{\, \right \} }
\def\dc{\tilde d}
\def\bskp{\bigskip}
\def\nV{\Omega}
\def\d3x{d^3 x}
\def\mtI{\b{\b I}}
\def\cmma{\;\;\; ,}
\def\perd{\;\;\; .}
\def\delv{\Delta v}
\def\eikr{{\rm e}^{i \bk \cdot \br}}
\def\rmikr{{\rm e}^{i \vec k \cdot \vec r}}
\def\rmnikr{{\rm e}^{-i \vec k \cdot \vec r}}
\def\eikx{{\rm e}^{i \bk \cdot \bx}}
\def\muo{\mu_0}
\def\epo{\epsilon_0}
\def\rme{{\rm e}}
\def\dnu{\Delta \nu_D}
\def\rAA{\AA \enskip}
\def\halp{{\rm H_\alpha}}
\def\hbet{{\rm H_\beta}}
\def\N#1{{N({\rm #1})}}
\def\ekz{\rme^{i ( k_z z - \omega t )}}
\def\cm#1{\, {\rm cm^{#1}}}
\def\bcm#1{\; {\rm \mathbf{cm^{#1}}}}
\def\rmK{\; \rm K}
\def\rmm#1{\; {\rm #1}}
\def\Nperp{N_\perp}
\def\spkr#1{{\big \bf {#1}}}
\def\inA{\quad $\bullet$ \quad}
\def\inB{\quad\quad $\diamond$ \quad}
\def\inC{\quad\quad\quad $\circ$ \quad}
\def\sigfc{\sigma_{f_c}}
\def\sech{\rm sech}
\def\blck{\color[named]{Black}}
\def\blue{\color[named]{Blue}}
\def\skyblue{\color[named]{SkyBlue}}
\def\red{\color[named]{Red}}
\def\black{\color[named]{Black}}
\def\green{\color[named]{Green}}
\def\white{\color[named]{White}}
\def\yellow{\color[named]{Yellow}}
\def\purple{\color[named]{Purple}}
\def\maroon{\color[named]{Maroon}}
\def\redv{\color[named]{RedViolet}}
\def\mahog{\color[named]{Mahogany}}
\def\prb{$\frac{\delta \rho}{\rho}$}
\def\trd{\frac{2}{3}}
\def\mdn{\bar \rho}
\def\ss{v_s^2}
\def\msol{M_\odot}
\def\msun{M_\odot}
\def\W#1{{W({\rm #1})}}
\def\Nav{N_a (v)}
\def\f#1{{f_{\rm #1}}}
\def\opt{$d \tau / d z \; $}
\def\btau{\bar\tau (v_{pk}) / \sigma (\bar \tau)}
\def\Ipk{I(v_{pk})/I_{c}}
\def\vrot{v_{rot}}
\def\sphr{\sqrt{R^2 + Z^2}}
\def\mbf{\mathbf}
\def\mgsqa{\, {\rm mag} / \Box ''}
\def\sqas{$\Box ''$}
\def\msqas{\Box ''}
\def\sqasec{$\Box ''$}
\newcommand{\ialii}{Al{\sc II}}
\newcommand{\iari}{Ar{\sc I}}
\newcommand{\ini}{N{\sc I}}
\newcommand{\ioi}{O{\sc I}}
\newcommand{\iovi}{O{\sc VI}}
\newcommand{\iovii}{O{\sc VII}}
\newcommand{\ioviii}{O{\sc VIII}}
\newcommand{\ifeii}{Fe{\sc II}}
\newcommand{\ifeiii}{Fe{\sc III}}
\newcommand{\ispii}{S{\sc II}}
\newcommand{\iznii}{Zn{\sc II}}
\newcommand{\icaii}{Ca{\sc II}}
\newcommand{\inai}{Na{\sc I}}
\newcommand{\icii}{C{\sc II}}
\newcommand{\iciv}{C{\sc IV}}
\newcommand{\ihii}{H{\sc II}}
\newcommand{\ihi}{H{\sc I}}
\newcommand{\isiii}{Si{\sc II}}
\newcommand{\isiiv}{Si{\sc IV}}
\newcommand{\ici}{C{\sc I}}
\newcommand{\dcost}{d \negmedspace \cos\theta}
\newcommand{\mnvhi}{n_{\rm H {\sc I}}}
\newcommand{\nvhi}{$n_{\rm H {\sc I}}$}
\newcommand{\rem}{{\rm e}}
\def\otpr#1#2{{|#1><#2|}}
\def\avgq#1{<\negmedspace#1\negmedspace>}
\newcommand{\negm}{\negmedspace}
\newcommand{\dfhf}{^2\negm D_{\frac{5}{2}}}
\newcommand{\negt}{\negthinspace}
\newcommand{\nalph}{N/$\alpha$}
\newcommand{\alphh}{$\alpha$/H}
\newcommand{\Rearth}{R_\oplus}
\newcommand{\mhmpc}{h^{-1} \, \rm Mpc}
\newcommand{\hmpc}{$\mhmpc$}

\renewcommand{\descriptionlabel}[1]%
    {\hspace{\labelsep}\textsf{#1}}

\newenvironment{Aitemize}{%
 \renewcommand{\labelitemi}{$\bullet \;$}%
	\begin{itemize}}{\end{itemize}}

\newenvironment{Renumerate}{%
 \renewcommand{\theenumi}{\Roman{enumi}}
 \renewcommand{\labelenumi}{\theenumi)}
	\begin{enumerate}}{\end{enumerate}}

\newenvironment{Aenumerate}{%
 \renewcommand{\theenumi}{\Alph{enumi}}
 \renewcommand{\labelenumi}{\theenumi.}
	\begin{enumerate}}{\end{enumerate}}

\newenvironment{dmnd}{%
 \renewcommand{\labelitemi}{$\diamond \quad$}%
	\begin{itemize}}{\end{itemize}}

\newenvironment{Penumerate}{%
 \renewcommand{\labelenumi}{(\theenumi)}
	\begin{enumerate}}{\end{enumerate}}


\pagestyle{fancyplain}

\lhead[\fancyplain{}{Intro}]{\fancyplain{}{\thepage}}
\rhead[\fancyplain{}{Intro-\thepage}]{\fancyplain{}{SaasFee16-DLAs}}
\cfoot{}

\newcommand{\mndi}{N_{\rm DI}}
\newcommand{\ndi}{$N_{\rm DI}$}
\newcommand{\helium}{${^4\rm He}$}
\newcommand{\rmch}{\frac{\rm C}{\rm H}}
\renewcommand{\labelitemii}{$\diamond$}
\renewcommand{\labelitemiii}{$\blacktriangle$}
\renewcommand{\labelitemiv}{$\circ$}
\newcommand{\prfi}{\ltp \hat\eng \cdot \vec r \rtp_{fi}}

%\newcommand{\dfhf}{^2\negm D_{\frac{5}{2}}}
%\newcommand{\dtvr}{\dot{\vec r}}
%\newcommand{\mtrx}{\bra{\phi_f} \rmikr \hat\eng \cdot \vec\nabla \ket{\phi_i}}
%\newcommand{\prfi}{\ltp \hat\eng \cdot \vec r \rtp_{fi}}


\special{papersize=8.5in,11in}

\begin{document}

\noindent {\bf{\large V. Damped \lya\ Systems [v2.0]}}

\begin{Aenumerate}

{\bf \item Overview}
 \begin{itemize}
 \item Extrema of HI absorption
 \item Pioneered by Wolfe who was searching for HI in high-$z$ galaxies
  \begin{itemize}
  \item Began by searching for 21cm absorption
  \item Gave up and pursued \lya\ instead
  \end{itemize}
 \item Obvious link between DLAs and galaxies (and vice versa)
  \begin{itemize}
  \item High HI surface density
  \item Dominate the Universe's neutral gas
  \end{itemize}
 \item But empirical links to galaxies have been challenging..
 \end{itemize}

  \includegraphics[width=5.0in]{Paper_figs/boomsma08_fig1.pdf}

{\bf \item Damped \lya\ (DLA) System Defined}
	\begin{itemize}

  	\item Simply: $\mnhi \ge 2 \sci{20} \cm{-2}$
      \begin{itemize}
      \item Motivated by HI surface density of local galaxies 
      \item But otherwise an arbitrary choice
      \item Varying \nhi\ (idealized DLA)

  \includegraphics[width=5.0in]{Figures/dla_vary_NHI.pdf}

%      \item Varying \nhi\ (real DLA)
%
%  \includegraphics[width=5.0in]{Figures/dla_vary_NHI.pdf}

      \end{itemize}

    \item Damping wings
      \begin{itemize}
      \item Doppler broadening is lost within the core (see HI\_Lyman\_Series Notebook)
      \item Profile is a 2-parameter fit: $z, \mnhi$
        \begin{itemize}
        \item In principle, constrained by hundreds (and more) pixels
        \item Dominant uncertainty is systematic: continuum
        \end{itemize}
      \item See {\bf Notebook} for an example DLA fit
      \end{itemize}

    \item Deviations from the Lorentzian
      \begin{itemize}
        \item The `wings' of the \lya\ cross-section are only approximated
        by the Voigt profile (i.e. Lorentzian)
        \item Lee et al.\ (2003) discusses the lowest order perturbations
        from second-order, time-dependent perturbation theory
        \item Only (potentially) measurable at very high \nhi
      \end{itemize}

  \includegraphics[width=5.0in]{Figures/dla_deviation.pdf}

  \end{itemize}

{\bf \item DLA Surveys}
  \begin{itemize}
  \item Launched in 1986 by Wolfe et al.\
    \begin{itemize}
    \item Low-dispersion ($R \approx 2000$), low S/N spectra
    \item Sufficient for resolving DLA absorption
    \end{itemize}
  \item Standard approach
    \begin{itemize}
    \item Randomly survey IGM sightlines for damped \lya\ profiles
    \item Only modest spectral resolution required (observed $W_\lambda > 10$\AA)
    \end{itemize}
  \item Sensitivity function
    \begin{itemize}
    \item Similar to surveys for Optically thick gas
    \item Except one DLA does not preclude the detection of another
    \item Modern surveys have leveraged the power of wide-field spectroscopic
    surveys (SDSS, BOSS)
      \begin{itemize}
      \item These are limited in redshift coverage, however
      \item Primarily $z=2 - 3.5$  (Prochaska et al.\ 2005, 2009; Noterdaeme et al.\ 2012)

  \includegraphics[width=5.0in]{Paper_figs/phw05_fig1.pdf}

  \includegraphics[width=5.0in]{Paper_figs/noterdaeme12_fig1.pdf}

      \item Dedicated surveys to control for systematics (S\'anchnez-Ram\'irez et al.\ 2016)

  \includegraphics[width=5.0in]{Paper_figs/ruben16_fig5.pdf}

      \item Dedicated surveys to extend to the limits of UV/Optical spectra
      \item Work at low-$z$ is doubly painful (Neeleman et al.\ 2016)
        \begin{itemize}
        \item Require UV space telescope
        \item Incidence of DLAs is considerably smaller
        \end{itemize}

  \includegraphics[width=5.0in]{Paper_figs/neeleman16_fig1.pdf}

      \end{itemize}

  \item Analysis
      \begin{itemize}
      \item Fit DLA profiles (by-hand or by computer)
      \item Measure \nhi, $z$
      \item Analyze \fnhi\ (next section)
      \end{itemize}

  \item Biases
    \begin{itemize}
    \item Dust in HI gas may obscure background sources (miss DLAs; e.g.\ Valdilo \& Peroux 2005)
      \begin{itemize}
      \item Solar metallicity gas with Milky Way dust and $\mnhi = 10^{21} \cm{-2}$ gives $A_V \approx 1$
      \end{itemize}

  \includegraphics[width=3.0in]{Paper_figs/vladilo05_fig1.pdf}

    \item Mass of DLA host galaxies may lens background quasars (bias toward DLAs;
    M\'enard \& Peroux 2003)
    \item DLA absorption reduces the spectral S/N, biasing search path
    (Prochaska et al.\ 2005; Noterdaeme et al.\ 2008)
    \end{itemize}

    \end{itemize}


  \end{itemize}

{\bf \item \fnhi\ of the DLAs}
  \begin{itemize}
  \item Binned evaluation
    \begin{itemize}
    \item Familiar estimator
    \begin{equation}
    f(\mnhi,X) = \frac{\rm Number \; of \; systems \; in \; \Delta X, \Delta N
    \, interval}{\Delta N \, \Delta X}
    \end{equation}
    \item Early days (1990s)
      \begin{itemize}
      \item 4-m class telescopes
      \item Limited number of known bright quasars
      \item One source at at time..
      \item Sample of several tens of DLAs
      \end{itemize}

  \includegraphics[width=3.0in]{Paper_figs/sl96_fig5.pdf}

      \begin{itemize}
      \item Limited statistical significance
      \item Data well represented by a single power-law that diverged!
      \end{itemize}
    \item SDSS
      \begin{itemize}
      \item With over 100 DLAs in the sample, a steep decline
      in \fnhi\ at high \nhi\ became obvious
      \item Break in \fnhi\ at $\approx 10^{21.7} \cm{-2}$

  \includegraphics[width=3.0in,angle=90]{Paper_figs/phw05_fig6.pdf}

      \end{itemize}
    \end{itemize}

  \item Models
    \begin{itemize}
    \item A population of randomly inclined exponential disks
    yields a broken power-law (Fall \& Pei 1993; Wolfe et al.\ 1995)
      \begin{itemize}
      \item Assume a face-on column density distribution with radius $r$
      \begin{equation}
      N_\perp (r) = N_{\perp,0} \, \exp(-r/r_d)
      \end{equation}
        \begin{itemize}
        \item $N_{\perp,0}$ is the central column density
        \item $r_d$ is a scale-length
        \end{itemize}
      \item For random, face-on disks (no inclination) the probability
      of intersection at $r$ is proportional to $r$ giving
      \begin{equation}
      f(N_\perp,X) = \frac{2 \pi c r_d^2 n_c}{H_0} \ltk \frac{\ln (N_{\perp,0}/N_\perp)}{N_\perp} \rtk
      \end{equation}
        \begin{itemize}
        \item $n_c$ is the assumed comoving number density of DLAs
        \end{itemize}
      \item Lastly, allow for random inclination
      \begin{equation}
      f(N,X) = \intl_0^{min(N,N_{\perp,0})} dN_\perp \, \ltk \frac{N_\perp^2 f(N_\perp)}{N^3} \rtk
      \end{equation}
      \item Evaluating 
      \begin{equation}
      f(N,X) = 
        \begin{cases}
        \frac{1}{N} \ltk 1 - 2 \, \ln \ltp \frac{N}{N_{\perp,0}} \rtp \rtk & N \le N_{\perp, 0} \\
        \frac{N_{\perp,0}^2}{N^3} & N \ge N_{\perp,0}
        \end{cases}
      \end{equation}
        \begin{itemize}
        \item The first term scales as $N^{-2}$ for $N \ll N_{\perp,0}$
        \item i.e., we expect a broken power-law!
        \end{itemize}

      \end{itemize}
    \item Empirically, one also considers a single power-law
    and $\Gamma$ functions
    \end{itemize}

  \item Maximum Likelihood Analysis
  \item Results
    \begin{itemize}
    \item Prochaska \& Wolfe 2009, Fig 1

  \includegraphics[width=4.0in]{Paper_figs/pw09_fig1.pdf}

      \begin{itemize}
      \item Well described with a broken power-law 
      \item With a steep decline (too steep)
      \end{itemize}
    \item Noterdaeme et al.\ 2012 (BOSS)

  \includegraphics[width=4.0in]{Paper_figs/noterdaeme12_fig2.pdf}

      \begin{itemize}
      \item Approximately $\mnhi^{-3}$ decline at the highest column densities
      \item Discovery of DLAs towards quasars with $\mnhi > 10^{22} \cm{-2}$
      \end{itemize}


    \end{itemize}

  \item Redshift evolution
    \begin{itemize}
    \item Weak evolution in the shape, if any (see PW09 figure above)
    \item This includes $z=0$, i.e.\ 21\,cm observations 
    (e.g.\ Zwaan et al. 2005; but see Braun 2012)
    \item Suggests a nearly non-evolving population of HI gas from $z=2$ to 0
    \end{itemize}

  \end{itemize}

{\bf \item $\ell(X)$}
  \begin{itemize}
  \item Incidence of DLA absorption
  \item Recall
  \begin{equation}
  \ell(X) = \frac{c}{H_0} n_c A_p
  \end{equation}
    \begin{itemize}
    \item $n_c$ is the co-moving number density of the systems
    \item $A_p$ is the physical cross-section (size)
    \end{itemize}
  \item Measurements of $\ell(X)$ constrain the evolution of 
  HI gas within (and around) galaxies across cosmic time
  \item Results (Prochaska \& Wolfe 2009, Fig 2)

  \includegraphics[width=4.0in]{Paper_figs/pw09_fig2.pdf}

    \begin{itemize}
    \item Sharp evolution from $z=2-4$
    \item Requires a $2\times$ decline in $n_c A_p$ in this
    $\approx 2$\,Gyr window
    \item Recall (discussed for LLS), $n_c$ is expected to
    {\it increase} with decreasing $z$
    \item Explanations?
      \begin{itemize}
      \item Photoionization of the gas? (decrease $n_c, A_p$)
      \item Galaxy feedback? (decrease $n_c, A_p$)
      \item Mergers (to decrease $n_c$)
      \end{itemize}
    \item Challenge for numerical simulations of galaxy formation
    (e.g.\ Bird et al.\ 2014, Figure 7)

\includegraphics[width=4.5in]{Paper_figs/bird14_fig7.pdf}

    \item $z=2$ evaluation of $\ell(X)$ matches the incidence at $z=0$
     \item Both the shape and normalization of \fnhi\ are nearly constant
      \begin{itemize}
      \item Is the HI gas in galaxies roughly invariant across
      10\,Gyr?!  
      \item Swimming Pool model of Galaxy formation (now Bathtub)

\includegraphics[width=4.5in]{Figures/swimming_pool.pdf}

      \end{itemize}
    \end{itemize}

  \end{itemize}

{\bf \item DLAs as Neutral Gas}
  \begin{itemize}
  \item Theoretical
    \begin{itemize}
    \item Consider photoionization of DLA gas by EUVB or
    local sources
    \item $\tau_{\rm LL}$ for DLA is over 1,000!
    \item Would require a tremendous radiation field to 
    ionize the gas appreciably (or a surprisingly diffuse gas 
    $n_{\rm H} \ll 1 \cm{-3}$)
    \item Simple calculation
      \begin{itemize}
      \item Bathe a slab of constant density gas in a radiation
      field 
      \item Insist $\mnhi \ge 2\sci{20} \cm{-2}$
      \item Calculate (Prochaska et al.\ 1996; Figure~15)

\includegraphics[width=4.5in]{Paper_figs/prochaska96_fig15.pdf}

      \end{itemize}
    \end{itemize}

  \item Empirical
    \begin{itemize}
    \item Examine multiple ionization states of metals associated
    to the HI gas
    \item Neutral gas should be dominated by low-ion states
      \begin{itemize}
      \item Those whose ionization potential lies just above 1\,Ryd
      \item e.g., Si$^+$, C$^+$, O$^0$
      \end{itemize}
    \item Observations (Prochaska et al.\ 2015, Fig 9)

\includegraphics[width=4.5in]{Figures/fig_ionization.pdf}
      \begin{itemize}
      \item By $\mnhi = 10^{20} \cm{-2}$, the gas is dominated by
      low-ions
      \item Empirical demonstration that the Hydrogen gas is
      predominantly neutral
      \end{itemize}

  \item DLAs, therefore, dominate the neutral gas in our Universe

    \end{itemize}

  \end{itemize}

{\bf \item Cosmological HI Mass Density}
  \begin{itemize}
  \item Define: $\rho_{\rm HI}$ to be the mass density of
  HI atoms per comoving Mpc$^3$
    \begin{itemize}
    \item Historically, often written as $\Omega_{\rm HI} = \rho_{\rm HI}/\rho_c$
    \item A bit more intuitive to consider $\rho_{\rm HI}$, e.g. to
    compare against the mass in stars
    \end{itemize}
  \item Evaluate from \fnhi

    \begin{equation}
    \rho_{\rm HI} = \frac{m_p H_0}{c} \intl_{N_{\rm min}}^\infty
    \mnhi \, f(\mnhi, X) \, dX \, d\mnhi
    \end{equation}

    \begin{itemize}
    \item $N_{\rm min}$ is generally taken to be the DLA threshold 
    $(2 \sci{20} \cm{-2})$ 
    \item HI mass density from neutral gas
    \item HI mass density from (putative) galaxies
    \item Can also make a (30\%) correction for He
    \end{itemize}

  \item Differential contribution of $\rho_{\rm HI}$ per $\log \mnhi$

\includegraphics[width=4.5in]{Figures/drho_dNHI.pdf}

    \begin{itemize}
    \item $\rho_{\rm HI}$ is dominated by DLA gas
    \end{itemize}

  \item Important: This calculation made {\it no} reference to the
  origin, nature, size, etc.\ of the DLA systems
    \begin{itemize}
    \item $\rho_{\rm HI}$ is entirely independent of such considerations
    \item Although these questions are of great interest to the community..
    \end{itemize}

  \item Evaluations (across redshift)
    \begin{itemize}
    \item PW09 again (Figure above)
    \item Crighton et al.\ 2015 (extending to $z \approx 5$)

\includegraphics[width=4.5in]{Paper_figs/crighton15_fig12.pdf}

      \begin{itemize}
      \item Extends measurements to $z \approx 5$
      \item Modest increase (if any) in $\rho_{\rm HI}$
      \end{itemize}

    \item Neeleman et al.\ 2016, Figure~6

\includegraphics[width=4.5in]{Paper_figs/neeleman16_fig6.pdf}
      \begin{itemize}
      \item Survey $z<1$ universe
      \item Limited statistics, even with the entire {\it HST} archive
      \item Ironically, it is easier to measure $\rho_{\rm HI}$ in the
      $z>2$ universe than today
      \item 21\,cm measurements will be required to achieve reasonable
      precision
      \end{itemize}

    \end{itemize}

  \end{itemize}

  \clearpage
{\bf \item Associating DLA Gas to Galaxies}
  \begin{itemize}
  \item Connecting this HI gas to galaxies
  \item Implied/expected physical association
    \begin{itemize}

    \item Heavy Element Enrichment (Rafelski et al.\ 2015)

\includegraphics[width=4.5in]{Figures/dla_metallicity.pdf}

      \begin{itemize}
      \item All DLAs exhibit heavy element enrichment
      \item Implies previous/current star-formation activity
      \item Wide dispersion in metallicity suggests wide
      dispersion in mass/luminosity
      \end{itemize}

    \item Fundamental Plane of DLAs (Neeleman et al.\ 2013)

\includegraphics[width=4.5in]{Figures/fund.pdf}

      \begin{itemize}
      \item Third parameter is velocity width (kinematics)
      \item Follows from mass-metallicity relation of galaxies
      \end{itemize}

    \item Modern Universe
      \begin{itemize}
      \item Essentially all HI gas detected with 21\,cm with 
      $\mnhi > 10^{20} \cm{-2}$ has been associated to a galaxy
      \item Few exceptions (e.g.\ dark galaxies in Virgo)
      \end{itemize}

    \item Theoretical prediction (simulations)
      \begin{itemize}
      \item Only regions of the universe with sufficient density
      are within dark matter halos
        \begin{itemize}
        \item These are, however, difficult calculations
        \item And few have treated the radiative transfer properly
        \end{itemize}
      \item Fumagalli et al.\ 2011; Bird et al.\ 2013
      \end{itemize}

\includegraphics[width=4.5in]{Paper_figs/fumagalli11_fig3.pdf}

\includegraphics[width=4.5in]{Paper_figs/bird13_fig1.pdf}

    \end{itemize}
  \item Observational efforts to identify counterparts (see Slides)
  \end{itemize}

{\bf \item GRB DLA}
  \begin{itemize}
  \item Long-duration GRBs
    \begin{itemize}
    \item Massive, exploding star
    \item Trace central SF regions of distant galaxies
    \item GRB afterglows are highly luminous, synchrotron sources
    that emit at rest-frame UV wavelengths
    \end{itemize}
  \item Example spectrum (Prochaska et al.\ 2009; Fig )

\includegraphics[width=4.5in]{Paper_figs/prochaska09_fig1.pdf}

    \begin{itemize}
    \item $\mnhi = 10^{22.7} \cm{-2}$
    \item Large \nhi\ are characteristic of GRB afterglow spectra
    \end{itemize}

  \item \nhi\ distribution from GRB afterglows (Jakobsson et al.\ 2006, Figure 3)

\includegraphics[width=4.5in]{Paper_figs/jakobsson06_fig3.pdf}

  \item GRB Observations
    \begin{itemize}
    \item Highly complementary to surveys with QSOs
    \item GRBs probe the central ISM of SF galaxies
      \begin{itemize}
      \item This gas has small cross-section
      \item Rarely probed by random quasar sightlines
      \end{itemize}
    \item GRBs bright enough to probe regions with $A_V \gtrsim 1$
    \end{itemize}

  \end{itemize}

\end{Aenumerate}

\end{document}
