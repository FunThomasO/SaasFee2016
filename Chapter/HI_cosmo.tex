%%%%%%%%%%%%%%%%%%%% author.tex %%%%%%%%%%%%%%%%%%%%%%%%%%%%%%%%%%%
%
% sample root file for your "contribution" to a contributed volume
%
% Use this file as a template for your own input.
%
%%%%%%%%%%%%%%%% Springer %%%%%%%%%%%%%%%%%%%%%%%%%%%%%%%%%%


% RECOMMENDED %%%%%%%%%%%%%%%%%%%%%%%%%%%%%%%%%%%%%%%%%%%%%%%%%%%
\documentclass[graybox]{svmult}

% choose options for [] as required from the list
% in the Reference Guide

\usepackage{mathptmx}       % selects Times Roman as basic font
\usepackage{helvet}         % selects Helvetica as sans-serif font
\usepackage{courier}        % selects Courier as typewriter font
\usepackage{type1cm}        % activate if the above 3 fonts are
                            % not available on your system
%
\usepackage{makeidx}         % allows index generation
\usepackage{graphicx}        % standard LaTeX graphics tool
                             % when including figure files
\usepackage{multicol}        % used for the two-column index
\usepackage[bottom]{footmisc}% places footnotes at page bottom

% see the list of further useful packages
% in the Reference Guide
\usepackage{amsmath}
\usepackage{amssymb}

\makeindex             % used for the subject index
                       % please use the style svind.ist with
                       % your makeindex program

%%%%%%%%%%%%%%%%%%%%%%%%%%%%%%%%%%%%%%%%%%%%%%%%%%%%%%%%%%%%%%%%%%%%%%%%%%%%%%%%%%%%%%%%%

\newcommand{\HI}{H{\sc I}}
\def\lya{Ly$\alpha$}
\def\mlya{{\rm Ly}\alpha}
\def\ohf{\frac{1}{2}}
\def\ket#1{{|#1\negmedspace>}}
\def\bra#1{{<\negmedspace#1|}}
\def\ltk{\left [ \,}
\def\ltp{\left ( \,}
\def\ltb{\left \{ \,}
\def\rtk{\, \right  ] }
\def\rtp{\, \right  ) }
\def\rtb{\, \right \} }
\def\imp{\, \Rightarrow \>}
\def\sci#1{{\; \times \; 10^{#1}}}
\def\rhf{\frac{3}{2}}
\def\cmma{\;\;\; ,}
\def\perd{\;\;\; .}
\def\smm{\sum\limits}
\def\intl{\int\limits}
\def\rme{{\rm e}}
\newcommand{\mnhi}{N_{\rm HI}}
\newcommand{\nhi}{$\mnhi$}
\def\cm#1{\, {\rm cm^{#1}}}
\def\mfnhi{f(\mnhi)}
\def\fnhi{$\mfnhi$}
\def\prodl{\prod\limits}
\def\mfnorm{<F>_{\rm norm}}
\def\fnorm{$\mfnorm$}

\begin{document}

\title*{\HI\ Absorption in the Intergalactic Medium}
% Use \titlerunning{Short Title} for an abbreviated version of
% your contribution title if the original one is too long
\author{J. Xavier Prochaska}
% Use \authorrunning{Short Title} for an abbreviated version of
% your contribution title if the original one is too long
\institute{J. Xavier Prochaska \at University of California, 
1156 High St., Santa Cruz, CA 95064 USA \email{xavier@ucolick.org}}
%
% Use the package "url.sty" to avoid
% problems with special characters
% used in your e-mail or web address
%
\maketitle

\abstract*{Each chapter should be preceded by an abstract (10--15 lines long) that summarizes the content. The abstract will appear \textit{online} at \url{www.SpringerLink.com} and be available with unrestricted access. This allows unregistered users to read the abstract as a teaser for the complete chapter. As a general rule the abstracts will not appear in the printed version of your book unless it is the style of your particular book or that of the series to which your book belongs.
Please use the 'starred' version of the new Springer \texttt{abstract} command for typesetting the text of the online abstracts (cf. source file of this chapter template \texttt{abstract}) and include them with the source files of your manuscript. Use the plain \texttt{abstract} command if the abstract is also to appear in the printed version of the book.}

\abstract{Each chapter should be preceded by an abstract (10--15 lines long) that summarizes the content. The abstract will appear \textit{online} at \url{www.SpringerLink.com} and be available with unrestricted access. This allows unregistered users to read the abstract as a teaser for the complete chapter. As a general rule the abstracts will not appear in the printed version of your book unless it is the style of your particular book or that of the series to which your book belongs.\newline\indent
Please use the 'starred' version of the new Springer \texttt{abstract} command for typesetting the text of the online abstracts (cf. source file of this chapter template \texttt{abstract}) and include them with the source files of your manuscript. Use the plain \texttt{abstract} command if the abstract is also to appear in the printed version of the book.}

\section{Historical Introduction}
\label{sec:history}

The discovery of the intergalactic medium (IGM)
was, in essence, precipitated by the discovery of 
quasars in 1963\footnote{There were (failed) attempts
to search for extragalactic gas in 21\,cm absorption
\cite{field59}} \cite{schmidt63}.
It was through spectroscopy of these enigmatic, distant
sources that one could resolve the absorption lines
from gas -- especially \HI\ \lya\ -- 
in the foreground universe.  
Figure~\ref{fig:burb} shows an early example from
\cite{bb+65} taken with the prime-focus spectrograph
on the Shane 120-inch telescope at Lick Observatory.
Even in these early data, one identifies apparently
discrete absorption lines of Hydrogen and heavy
elements establishing the presence of diffuse yet
enriched gas along the sightline.



% For figures use
%
\begin{figure}[b]
\sidecaption
% Use the relevant command for your figure-insertion program
% to insert the figure file.
% For example, with the graphicx style use
\includegraphics[scale=.4]{Figures/burbidge65}
%
% If no graphics program available, insert a blank space i.e. use
%\picplace{5cm}{2cm} % Give the correct figure height and width in cm
%
\caption{Lick spectrum of 3C 191 obtained in February 1666
with the prime-focus spectrograph on the Shane 120-inch
telescope at Lick Observatory.  The comparison lamp spectrum
shown is that of He+Ar.
}
\label{fig:burb}       % Give a unique label
\end{figure}


Spectra like these inspired the first models of
the IGM as discrete absorption lines \cite{bahcall65}
and by inference the first physical insight.
The positive detection of flux at rest wavelengths
shortward of the quasar \lya\ emission line
($\lambda_{\rm rest} < 1215$\AA)
demands a highly ionized IGM.
\cite{gunn65} recognized that a universe with predominantly
neutral hydrogen gas should be opaque to these far-UV
photons and [inferred] -- correctly -- 
that the gas must have a neutral fraction $x_{\rm HI}$ of less 
than 1 part in $10^5$.
As an introduction to the material presented in this Chapter,
we may offer our own rough estimate.
The optical depth of \HI\ \lya\ through
a $\ell = 100$\,kpc portion of the $z=3$ universe
at the mean hydrogen density $\bar n_{\rm H}$ 
is simply

\begin{equation}
\tau(\nu) = \ell \, \bar n_{\rm H} \, x_{\rm HI} \, \sigma_{\mlya}(\nu)
\end{equation}
with $\sigma_{\mlya}$ the \lya\ cross-section.
We estimate the latter assuming Doppler broadening dominates
with a characteristic velocity given by Hubble expansion,
$\Delta v \approx H(z) * \ell \approx 30$\,km/s 
(see Equation~\ref{eqn:Dopp_linecenter}).
Taking a baryonic mass density $\rho_b = 0.0486 \rho_c$
at $z=0$ with $\rho_c$ the critical density and
taking 75\%\ of the baryonic mass as Hydrogen,
we find $\tau(\nu_{jk}) \approx 10^6 x_{\rm HI}$.
Therefore, the positive detection of flux in the 
\lya\ forest demands a highly ionized IGM.

The remainder of the 1960's introduced a series of 
fundamental papers on the astrophysics of absorption-line
analysis, especially by Bachall and his collaborators.
These included the discussion of fundamental diagnostics
of the gas \cite{bahcall66}, 
the application of absorption from the fine-structure 
levels of heavy elements \cite{bahcall67}, and the
assertion that the majority of heavy element absorption
may be associated to the halos of galaxies
\cite{bahcall69a,bahcall69b}.
In a number of respects, the theory had outpaced the
observations.
This held throughout the 1970's, especially for IGM
studies with \HI\ \lya\ although \cite{brown73}
reported the first intergalactic detection of 
\HI\ in 21\,cm absorption.

% For figures use
%
\begin{figure}[b]
\sidecaption
% Use the relevant command for your figure-insertion program
% to insert the figure file.
% For example, with the graphicx style use
\includegraphics[scale=.4]{Figures/young78}
%
% If no graphics program available, insert a blank space i.e. use
%\picplace{5cm}{2cm} % Give the correct figure height and width in cm
%
\caption{\HI\ \lya\ forest spectrum of the quasar PKS 2126--158
obtained by \cite{young78}.  Data like these provided the first
detailed view of the IGM.
}
\label{fig:young}       % Give a unique label
\end{figure}


In the early 1980's, advances in spectroscopic technology
(especially the CCD detector) led to the first high-quality
views of the \HI\ \lya\ forest 
\cite[Figure~\ref{fig:young}]{young78,boks,sargent}.
It was evident from spectra like these that the IGM was
characterized by a stochastic forest of \HI\ absorption
well-described by discrete lines.  [add another line]
This decade also witnessed the first surveys on gas
optically thick at the \HI\ Lyman limit (aka Lyman Limit Systems 
or LLSs; \cite{tytler82})
and on the \HI\ \lya\ absorbers with sufficient column
density to generate damped \lya\ profiles
(aka damped \lya\ systems or DLAs; \cite{wolfe86}).
The field was suddenly awash with data and theory had
now fallen behind.
[Mention Bergeron]
The observers took to developing models of 
`spherical' \HI\ clouds and bull's-eye cartoons to 
describe the gas around galaxies.  
J. Ostriker was the most active theorist on the IGM,
publishing a series of papers on applications of the
IGM including its first phase diagram 
\cite{ostriker83a,ostriker83b,bajtlik87,duncan89}.
But one of his leading models of the day envisioned
"Galaxy formation in an IGM dominated by explostion"
\cite{ostriker80}.


In several respects, the 1990's witnessed the
true maturation of studies the IGM.  Observationally,
the HIRES spectrometer \cite{vogt94} on the 10-m W.M.
Keck telescope fully resolved the IGM at terrific
S/N.  In several respects, these spectra represent
the pinnacle (analogous to Planck on the CMB).
The advance over even the 1980's was profound
as Figure~\ref{fig:HIRES} illustrates.

% For figures use
%
\begin{figure}[b]
\sidecaption
% Use the relevant command for your figure-insertion program
% to insert the figure file.
% For example, with the graphicx style use
\includegraphics[scale=.4]{Figures/Q0636}
%
% If no graphics program available, insert a blank space i.e. use
%\picplace{5cm}{2cm} % Give the correct figure height and width in cm
%
\caption{Comparison of the high quality Palomar spectrum
of Q0636 against the Keck/HIRES data.
}
\label{fig:HIRES}       % Give a unique label
\end{figure}


a new paradigm for the IGM emerged
from hydrodynamic cosmological simulations \cite{miralda96}
and related analytic treatments \cite{huixx}.
The \lya\ clouds were replaced by the Cosmic Web (Figure~\ref{fig:web}),
the filamentary network of dark matter and baryons that 
describes the large-scale structure of a CDM universe.
The \HI\ \lya\ forest traces the undulations in this web
and this so-called 
With this paradigm established, the IGM o
This so-called fluctuating Gunn-Peterson approximation
offers a terrific description of the IGM with sound
analytic underpinnings.


% For figures use
%
\begin{figure}[b]
\sidecaption
% Use the relevant command for your figure-insertion program
% to insert the figure file.
% For example, with the graphicx style use
\includegraphics[scale=.4]{Figures/miralda96}
%
% If no graphics program available, insert a blank space i.e. use
%\picplace{5cm}{2cm} % Give the correct figure height and width in cm
%
\caption{Cosmic web (from \cite{miralda96})
}
\label{fig:web}       % Give a unique label
\end{figure}


Figure~\ref{fig:web_vs_data} compares an early generation
model of the IGM from a hydrodynamic simulation 
against a portion of a Keck/HIRES spectrum.  The
agreement is remarkable and even the expert reader
is challenged to identify which panel is real and
which is simulated.
The cosmic web paradigm is a true triumph of CDM
cosmology and its development ushered in the 
opportunity to leverage IGM observations for fundamental
cosmological [constraints].

% For figures use
%
\begin{figure}[b]
\sidecaption
% Use the relevant command for your figure-insertion program
% to insert the figure file.
% For example, with the graphicx style use
\includegraphics[scale=.4]{Figures/web_vs_HIRES}
%
% If no graphics program available, insert a blank space i.e. use
%\picplace{5cm}{2cm} % Give the correct figure height and width in cm
%
\caption{Cosmic web vs.\ HIRES spectrum
}
\label{fig:web_vs_HIRES}       % Give a unique label
\end{figure}


For the last decade, the observational advances
have stemmed largely form the massive spectroscopic
surveys of SDSS and BOSS.
These have yielded terrific statistical descriptions
of the IGM \cite{PDF} 
across large areas of the sky for BAO \cite{IGM_BAO}.
Large surveys of the IGM have also been comprised
\cite{phw05,pow10,noterdaeme} and
analysis probing the underlying dark matter density
field probed by the IGM have emerged \cite{font}.
In addition, the ongoing discovery of $z>6$ quasars 
and GRBs coupled with high-performance echellette
spectrometers have probed the IGM to the epoch
of \HI\ reionization.  And, a series of increasingly
sensitive UV spectrometers on the
{\it Hubble Space Telescope}
have anchored the results in the modern universe
\cite{penton,tripp,xx}.

This chapter is organized into the following sections:
 (i) the physics of \HI\ \lya\ absorption;
 (ii) key concepts of spectral-line analysis;
 (iii) characterizing the \HI\ \lya\ forest as absorption lines;
 (iv) optically thick \HI\ absorption;
 (v) an overview of modern analysis and results.
The focus throughout is observational and the approaches
are largely traditional;  excellent reviews with a greater
emphasis on theory are given by \cite{meiksin0X} and
\cite{mcquinn1X}.  

This Chapter is also supplemented by the lecture notes
and slides presented in SaasFee, and a set of iPython
Notebooks illustrating concepts and providing 
example code for related calculations and modeling.
These supplementary materials are publicly available
at https://github.com/profxj/SaasFee2016.
Python code relevant to \HI\ \lya\ absorption and 
IGM analysis are packaged as  {\tt linetools}
and {\tt pyigm} on github. [give links]

%%%%%%%%%%%%%%%%%%%%%%%%%%%%%%%%%%%%%%%%%%%%
%%%%%%%%%%%%%%%%%%%%%%%%%%%%%%%%%%%%%%%%%%%%
\section{Physics of Lyman Series Absorption}
\label{sec:physics}

The \HI\ \lya\ photon may be emitted following one
of several processes:
 (i) the resonant absorption of a \lya\ photon by atomic hydrogen;
 (ii) as the final emission in the recombination cascade of \HI;
 (iii) following the collisional excitation of atomic hydrogen.
In this Chapter, we concern ourselves with the first process --
\HI\ \lya\ resonant-line scattering -- although the other
processes may play a critical role in \lya\ radiative transfer.

In this section we will
  describe the physics of \HI\ line absorption,
  introduce the concepts of a line-profile, and
  and illustrate the basics of \lya\ absorption lines.
This section is supplemented by the iPython Notebook
"HI\_Lyman\_Series.ipynb" and the "sassfee16\_lymanseries"
Lecture.

\subsection{\HI\ Energy Levels}

We begin with a derivation of the energy levels 
for atomic hydrogen.  Energies in the classic
Rutherford-Bohr model of a Hydrogenic ion with charge
$Ze$ are solved from a standard Hamiltonian ($H^{(0)}$)
with an electrostatic potential

\begin{equation}
H^{(0)} = \frac{-\hbar \nabla^2}{2m} - \frac{Z e^2}{r}
\end{equation}
For energy level $n$, one recovers

\begin{equation}
E_n = -\ohf \mu c^2 \frac{(Z \alpha)^2}{n^2}
\label{eqn:En}
\end{equation}
and the quantum states described by $n$, $\ell$, $m$, $m_s$
or $\ket{n \ell m m_s}$ 
are degenerate in $\ell, m, m_s$
because our Hamiltonian is rotationally invariant.
In Equation~\ref{eqn:En}, we have
the fine-structure constant 
$\alpha \equiv e^2/\hbar c \approx 1/137$ and
the reduced mass $\mu = \frac{m_e (Z m_p)}{m_e + Z m_p}$.
For Hydrogen, $\mu \approx 0.999 m_e$. 

From $E_n$, we may evaluate the wavelengths for the Lyman
series, transitions from/to the ground state $E_1$, as

\begin{equation}
\lambda_{rest,n} = \frac{hc}{E_n - E_1}
\label{eqn:lrest}
\end{equation}
Table~\ref{tbl:wrest} lists the calculated values 
from Equations~\ref{eqn:En} and \ref{eqn:lrest}
for $\approx 20$ Lyman series lines, compared against
empirical measurements.  One identifies a systematic
offset of $\delta\lambda \approx 0.015$\AA\ between
the Rutherford-Bohr energies and experiment.
These result from perturbations to the 
standard Hamiltonian that we now consider.

\begin{table}[ht]
\begin{center}
\caption{Lyman Series Lines \label{tab:energies}}
\begin{tabular}{cccccc}
\hline
Transition & $n$ & $E_n - E_1$ & $\lambda_{rest}$ & $\lambda_{exp}$ \\
& & (eV) & (\AA) & (\AA) \\
\hline
Ly$\alpha$&           2&       10.200000&       1215.6845& 1215.6701 \\
Ly$\beta$&            3&       12.088889&       1025.7338& 1025.7223\\
Ly$\gamma$&           4&       12.750000&       972.54759&  972.5368\\
Ly$\delta$&           5&       13.056000&       949.75351&  949.7431\\
Ly$\epsilon$&         6&       13.222223&       937.81375&  937.8035\\
Ly6&                  7&       13.322449&       930.75844&  930.7483\\
Ly7&                  8&       13.387500&       926.23580&  926.2257\\
Ly8&                  9&       13.432099&       923.16041&  923.1504\\
Ly9&                  10&       13.464000&       920.97310& 920.9631\\
Ly10&                 11&       13.487604&       919.36139& 919.3514\\
Ly11&                 12&       13.505556&       918.13933& 918.1294\\
Ly12&                 13&       13.519527&       917.19053& 917.1806\\
Ly13&                 14&       13.530613&       916.43908& 916.429\\
Ly14&                 15&       13.539556&       915.83374& 915.824\\
Ly15&                 16&       13.546875&       915.33891& 915.329\\
Ly16&                 17&       13.552942&       914.92921& 914.919\\
Ly17&                 18&       13.558025&       914.58616& 914.576\\
Ly18&                 19&       13.562327&       914.29604& 914.286\\
Ly19&                 20&       13.566000&       914.04849& 914.039\\
Ly$\infty$&         $\infty$&   13.6&            912.6    & \\
\hline
\end{tabular}
\end{center}
\end{table}


There are two perturbations with energies that scale
as $\alpha^4$, the next term $E_n$:
(1) spin-orbit coupling and
(2) the first expansion of the relativistic kinetic energy.
Classically, the spin-orbit coupling is described as 
a magnetic dipole interaction between the spin of the electron
and the orbit of the nucleus, i.e.\ 
the e$^{-}$ observes a magnetic field due
to the current driven by the nucleus:

\begin{equation}
\vec B = - \frac{1}{c} \vec v \times \vec 
E = \frac{1}{m_e c r} \; \vec \ell \;
\frac{d\phi}{dr}
\end{equation}
with energy

\begin{equation}
E = - \vec \mu_s \cdot \vec B  \quad\quad \text{with} \;\;\; 
\vec \mu_s = -\frac{e g \vec s}{2 m_e c}
\end{equation}
where $g \approx 2$ for an electron.
The proper Hamiltonian for the perturbation is:

\begin{equation}
H_{SO} = \frac{1}{2 m_e^2 c^2} \; \vec S \cdot \vec L \; \frac{1}{r} \;
\frac{d\phi}{dr}
\end{equation}
and the standard treatment is to 
identify an operator which commutes with 
$H^{(0)}$ and $H_{SO}$, and also
uniquely identifies the degenerate states.
We choose  $\vec J \equiv \vec L + \vec S$
recognizing

\begin{equation*}
\vec L \cdot \vec S = \ohf \ltp |\vec J|^2 - |\vec L|^2 - |\vec S|^2 \rtp
\end{equation*}
which yields energies

\begin{equation}
E_{SO} = \, <H_{SO}> \, = \ohf C \ltk j(j+1) - \ell(\ell+1) - s(s+1) \rtk
\end{equation}
with $C$ a constant.
For fixed $\vec L$ and $\vec S$ (i.e.\ splitting within a level), 
e.g.\ 2P, we have $\Delta E_{SO} = E_{J+1} - E_J = C (j+1)$.
For our Hydrogenic ion,
$\phi = Z e^2/r \imp d \phi/dr = -Z e^2/r^2$ and

\begin{equation}
H_{SO} = \frac{Z e^2}{2 m_e^2 c^2} \frac{1}{r^3} \vec L \cdot \vec S  \;\; .
\end{equation}
Following standard perturbation 
theory\footnote{See the Lecture notes for an expanded derivation.},
we find

\begin{equation}
<H_{SO}> = -E_n \frac{Z^2 \alpha^2}{2n} \frac{[j(j+1) - 
\ell(\ell+1) - s(s+1)]}{\ell(\ell+\ohf)(\ell+1)}
\end{equation}
with $E_n$ given by Equation~\ref{eqn:En}.
As advertised, spin-orbit coupling is 4th order in $\alpha$
and we now have explicit energy dependence on $j,\ell$ and $s$.
For the Hydrogen $n=2$ levels ($Z=1; n=2; \ell = 0,1; j= 1/2, 3/2$),
we find

\begin{equation}
<H_{SO}> = 
  \begin{cases}
	0 &  2 ^2{\rm S}_\ohf \quad\quad (\ell = 0; j=0) \\
     \frac{mc^2 \alpha^4}{96} &  2 ^2{\rm P}_{3/2} \quad\quad (\ell=1; j=3/2) \\
     \frac{-mc^2 \alpha^4}{48} &  2 ^2{\rm P}_{1/2} \quad\quad (\ell=1; j=1/2) \\
  \end{cases}
\end{equation}
giving a 2P$_{3/2}$ - 2P$_{1/2}$ splitting of $4.5 \sci{-5}$~eV
or $\Delta v \approx \Delta E / cE \approx 1$~km/s.

To derive the Relativistic correction to order $\alpha^4$,
we expand the K.E. to the next term in $v^2/c^2$ from the Lagrangian

\begin{equation}
\text{K.E.} = \frac{p^2}{2m} \ltp 1 - \frac{1}{4} \frac{v^2}{c^2} \rtp
\end{equation}
This gives a relativistic perturbation 

\begin{equation}
H_{rel} = -\frac{1}{2mc^2} \ltp \frac{p^2}{2m} \rtp^2
\end{equation}
that has no spin dependence (spherically symmetric)
such that $[H_{rel}, L^2] = [H_{rel}, L] = 0$
and the standard $\ket{n \ell m m_s}$ diagonalize $H_{rel}$.
If we recognize that

\begin{equation*}
H_{rel} = -\frac{1}{2mc^2} \ltp H^{(0)} - V^{(0)} \rtp^2
\end{equation*}
with $V^{(0)} = -Z e^2/r$, it is straightforward to
compute the energies

\begin{equation}
<H_{rel}> = -E_n \frac{Z^2 \alpha^2}{n} \ltp \frac{3}{4n} - \frac{1}{\ell + \ohf} \rtp
\end{equation}

Combining this result with spin-orbit coupling,
we recover (truly remarkably), 

\begin{equation}
<H_{SO}> + <H_{rel}> \, = E_n \frac{Z^2 \alpha^2}{n} \, 
\ltk \frac{1}{j+\ohf} - \frac{3}{4n} \rtk
\end{equation}
Altogether, the $\alpha^4$ term for $E_n$ has
no explicit $\ell$ dependence, nor any $s$ dependence.
Furthermore, higher $j$ implies higher energy 
following the 3rd Hund's rule.
The energy shifts from $E_n$ for Hydrogen are then

\begin{equation}
\Delta E_{nj} = -7.25 \sci{-4} {\rm eV} \; \frac{1}{n^3} 
   \ltk \frac{1}{j+\ohf} - \frac{3}{4n} \rtk
\end{equation}
which for the $n=1,2$ states of Hydrogen evaluate to
the results in Table~\ref{tab:Enshift}.

\begin{table}
\label{tab:Enshift}
\caption{Perturbations to the $n=1,2$ levels of Hydrogen}
\begin{center}
\begin{tabular}{lccc}
\hline
State & $n$ & $j$ & $<H_{SO}> + <H_{rel}>$ \\
\hline
1\,$^2$S$_{\ohf}$ & 1 & $\ohf$ & $-1.8 \sci{-4}$~eV \\
2\,$^2$S$_{\ohf}$, 2\,$^2$P$_\ohf$ & 2 & $\ohf$ & $-5.7 \sci{-5}$~eV \\
2\,$^2$P$_\rhf$ & 2 & $\rhf$ & $-1.1 \sci{-5}$~eV \\
\hline
\end{tabular}
\end{center}
\end{table}

The splitting of the $n=2$ level implies that
Ly$\alpha$ (1S-2P) is a doublet with $\approx 1$km/s separation
which is generally too small to resolve observationally
but could be important for radiative transfer treatments.

Returning to Table~\ref{tab:En}, the implied
shift from the Rutherford-Bohr energies
$\delta\lambda/\lambda \sim \delta E/E$
is calculated using $\delta E$ from Table~\ref{tab:Enshift}
and accounting for the relative degeneracy of the 
2\,$^2$P$_\ohf$, 2\,$^2$P$_\rhf$ states
yields

\begin{align}
\delta E &= \frac{1}{3} \ltk 2\delta E_{1S \to 2P_\rhf} + \delta E_{1S \to 2P_\ohf}
   \rtk \\
         &= 1.55 \sci{-4}~{\rm eV}
\end{align}
or
\begin{equation}
\Delta \lambda = -\frac{\delta E}{E} \times 1215.68 {\rm \AA} = -0.0138 {\rm \AA} \perd
\end{equation}
Voila!

%%%%%%%%%%%%%%%%%%%%%%%%%%%%%%%%%%%%%%%%%%%%%%%%%%%%%%%%
\subsection{The Line Profile}

We now derive the line profile of \HI\ Lyman series absorption
which expresses the energy dependence of the
photon cross-section.   This results from two physical
effects: 
(1) the quantum mechanical coupling of the energy
levels described in the previous sub-section; 
(2) Doppler broadening from the kinetic motions of the gas.
We reserve a discussion of the observed
line-profile related to the instrument to the next section.

We express the opacity $\kappa_{\rm jk}(\nu)$
of a gas with number
density $n_j$ in state $j$ as:

\begin{equation} 
\kappa_{\rm jk}(\nu) = n_j \sigma_{jk}(\nu)
\end{equation} 
with $\sigma_\nu$ the photon cross-section 
at frequency $\nu$ for a transition to state $k$.
Separate the frequency dependence by introducing
the line-profile $\phi(\nu)$

\begin{equation}
\sigma(\nu) = \sigma_{jk} \phi(\nu)
\end{equation}
with $\sigma_{jk}$ the integrated cross-section over all frequencies
and $\phi_\nu \, d\nu$ reflects the probability an atom will absorb
a photon with energy in $\nu, \nu + d\nu$.

A naive guess for $\phi(\nu)$ is the Delta function,
i.e.\ the transitions occur only at the exact energies
splitting the energy levels from the previous subsection:

\begin{equation}
\phi_\delta(\nu) = \delta \ltp \nu - \nu_{jk} \rtp
\end{equation}
Quantum mechanically, however, the excited $n=2$
state has a half-life $\tau_\ohf$ 
given by the Spontaneous emission coefficient,
$\tau_\ohf = 1/A_{\rm jk}$.
This finite lifetime implies a finite width
$\Delta E$ to the energy level which may be estimated
from the Heisenberg Uncertainty principle

\begin{equation}
\Delta E \sim \frac{\hbar}{\Delta t} \sim \hbar A
\end{equation}
Define $W_{jk}(E)$ as the quantum mechanical
probability of a transition occurring 
between states $j$ and $k$ with energy $E$
and $W_j(E)$ 
as the probability of state $j$ being characterized
by the energy interval $(E_j, E_j+dE_j)$.
A standard Quantum mechanical treatment shows that
$W_j(E)$ has a Lorentzian shape (aka the Breit-Wigner
profile): 

\begin{equation}
W_j(E_j) dE_j = \frac{\gamma_j dE_j/h}{(2\pi/h)^2 \ltk E_j - 
<E_j>\rtk^2 + (\gamma_j/2)^2}
\end{equation}
with  

\begin{equation}
\gamma_j \equiv \smm_{i<j} A_{ij}
\end{equation}


For a coupling between only two states $j,k$, one
derives $W_{jk}$ by convolving $W_j$ and $W_k$: 

\begin{equation}
W_{jk}(E) dE = \frac{\ltk \gamma_j + \gamma_k \rtk dE/h}{
(2\pi/h)^2 \ltk E - E_{jk}\rtk^2 + ([\gamma_j + \gamma_k]/2)^2}
\label{eqn:Wjk}
\end{equation}
We reemphasize that this calculation was restricted 
to the $j$ and $k$ states whereas a proper 
calculation needs to consider the coupling of all the 
energy levels (see below).  For our Lyman series lines we
note that $\gamma_j=0$ as, by definition, there are 
no energy levels below the ground state.

From Equation~\ref{eqn:Wjk}, we introduce
the Natural line-profile normalized to have 
unit integral value in frequency,

\begin{equation}
\phi_N(\nu) = \frac{1}{\pi} 
\ltk \frac{(\gamma_j + \gamma_k)/4\pi}{(\nu - \nu_{jk})^2
+ (\gamma_j + \gamma_k)^2 / (4\pi)^2} \rtk \perd
\end{equation}
giving (at last)

\begin{equation}
\sigma(\nu) = \sigma_{jk} \phi_N(\nu)
\end{equation}
As illustrated in the Notebook, 
the "wings" of $\phi_N(\nu)$ are 
$\approx 10$ orders of magnitude down 
from line-center.  Remarkably
these are very important for \lya.

It is standard practice to express the normalization
$\sigma_{jk}$ in terms of the oscillator strength $f_{jk}$
which is either measured empirically (preferred)
or estimated theoretically

\begin{equation}
\sigma_{jk} = \frac{\pi e^2}{m_e c} f_{jk}
\end{equation}

Our final expression becomes

\begin{equation}
\sigma_\nu = \frac{\pi e^2}{m_e c} f_{jk} 
\ltk \frac{(\gamma_j + \gamma_k)/4\pi^2}{(\nu - \nu_{jk})^2
+ (\gamma_j + \gamma_k)^2 / (4\pi)^2} \rtk
\end{equation}

[NEED to GIVE $A$ for \lya\ somewhere]

Expressing the FWHM of the line-profile as 
the frequency width where $\sigma(\nu)/\sigma(\nu)_{max} = 1/2$,
we have $\Delta\nu_{\rm FWHM} = \pm \frac{\gamma_j + \gamma_k}{4\pi}$
which very nearly matches our estimate from the
Uncertainty Principle!
For \HI\ \lya\ with $\gamma_1=0, \gamma_2=A_{21}$,
we find the FWHM in velocity to be 

\begin{equation}
\Delta v_{\rm FWHM} = c \Delta\nu_{\rm FWHM}/\nu
\approx 1.5\sci{-2} \, {\rm km/s}  
\label{eqn:vFWHM}
\end{equation}
For astrophysical purposes,
this nearly is a delta function.

In an astrophysical event, each 
atom in a gas has its own motion 
which spreads the line without changing the total 
amount of absorption.  
This Doppler effect, to lowest order in $v/c$, is

\begin{equation}
\Delta\nu = \nu - \nu_{jk} = \nu_{jk} \frac{v}{c}
\end{equation}
Assuming first that the gas motions are characterized
soley by $T$ (i.e.\ no turbulence),
we adopt a Maxwellian distribution for particles of mass $m_A$
giving a profile function:

\begin{align}
\phi_D(\nu) &= \frac{1}{\Delta \nu_D \sqrt{\pi}} 
\exp \ltk - \frac{(\nu - \nu_{jk})^2}{\Delta \nu_D^2} \rtk \\
{\rm with} \; \Delta \nu_D &\equiv \frac{\nu_{jk}}{c} \sqrt{\frac{2kT}{mA}}
\end{align}
At line-center ($\nu = \nu_{jk}$),
the cross-section (neglecting stimulated emission) is

\begin{equation}
(\sigma_\nu^D)_{\rm max} = \sigma_{jk} \phi_D(\nu_{jk})  
	= \frac{\pi e^2}{mc} f_{jk} \frac{1}{\Delta \nu_D \sqrt{\pi}}
\label{eqn:Dlinecenter}
\end{equation}
This cross-section is several orders of magnitude
lower than $(\sigma_\nu^N)_{\rm max}$ as the absorption has
been `spread' over a velocity interval several orders of
magnitude larger than the width estimated by Equation~\ref{eqn:vFWHM}.

We can generalize the profile to include random, turbulent
motions (characterized by $\xi$)
by modifying the Doppler width,

\begin{equation}
\Delta\nu_D = \frac{\nu_{jk}}{c} \ltp \frac{2kT}{m_A} + 
\xi^2 \rtp^\ohf
\end{equation}
Expressing the line-profile in velocity space, we introduce
the Doppler parameter

\begin{equation}
b \equiv \sqrt{\frac{2kT}{m_A} + \xi^2}
\end{equation}
and the velocity line-profile for Doppler motions is

\begin{equation}
\phi_D(v) = \frac{1}{b\sqrt{\pi}} \, \exp \ltk -\frac{v^2}{b^2} \rtk \perd
\label{eqn:phi_Dv}
\end{equation}
See the Notebook for a series of examples illustrating this
profile in comparison to the Natural profile.


Generally, the cross-section has contributions from both
Natural and Doppler broadening,
with Doppler broadening dominating the line-center 
and the Lorentzian of Natural broadening dominates the wings.
The overall profile is a convolution of the two terms,

\begin{equation}
\phi_V(\nu) = \frac{\gamma}{4\pi} \intl_{-\infty}^\infty
\frac{ \ltp \frac{m}{2\pi kT} \rtp^\ohf \, \exp \ltp - \frac{mv^2}{2kT} \rtp}
{\ltp \nu-\nu_{jk}-\nu_{jk}v/c \rtp^2 + \ltp \gamma/4\pi \rtp^2} \, dv
\end{equation}
and one is inspired to introduce the Voigt function

\begin{equation}
H(a,u) = \frac{a}{\pi} \intl_{-\infty}^\infty 
\frac{\rme^{-y^2} \, dy}{a^2 + (u-y)^2}
\end{equation}
where we identify

\begin{align}
a &\equiv \frac{\gamma}{4\pi \Delta\nu_D} \\
u &\equiv \frac{\nu-\nu_{jk}}{\Delta\nu_D}
\end{align}
Altogether, we have 

\begin{equation}
\phi_V(\nu) = \frac{H(a,u)}{\Delta\nu_D \sqrt{\pi}}
\end{equation}
which has no analytic solution.
For speed, one often relies on look-up tables.
In Python, the real part of scipy.special.wofz is both accurate 
and fast (see Voigt documentation in the {\tt linetools} package).

The Lorentzian profile from Natural broadening is only an approximation
because it ignores the coupling between states other than $j,k$.
Scattering is a second-order quantum mechanical process: annihilation of 
one photon and the creation of a scattered photon.
A proper treatment requires second-order time dependent perturbation
theory and requires one to sum over all bound-states and integrate over
all continuum-state contributions.
\cite{lee03} have calculated a series expansion of the corrections
to the Voigt profile and we refer the reader to his paper for a
more detailed presentation.  Here, we report the first term
in the series for \HI\ \lya:

\begin{equation}
\sigma_\nu = \sigma_{\rm T} \ltp \frac{f_{jk}}{2} \rtp^2 
\ltp \frac{\nu_{jk}}{\delta\nu} \rtp^2
\ltk 1 - 1.792 \frac{\delta\nu}{\nu_{jk}} \rtk  \perd
\end{equation}
with $\sigma_T$ the Thompson cross-section.
It is evident that the correction is not symmetric about line-center.
This leads to an asymmetry which
shifts the measured line-center if the gas opacity
is very high (e.g.\ $\mnhi > 10^{21.7} \cm{-2}$).

\subsection{Optical Depth ($\tau_\nu$) and Column Density ($N$)}
\label{subsec:tauN}

We now introduce two quantities central to absorption-line analysis.
First, the optical depth $\tau_\nu$ which is defined as the integrated
opacity along a sightline.  In differential form
$d\tau(\nu) = -\kappa(\nu) ds$ implying

\begin{equation}
\tau(\nu) = \sigma(\nu) \int n_j ds
\end{equation}
which gives an explicit frequency dependence related to the line-profiles
of the previous sub-section.

We define the second term as the column density $N_j$,

\begin{equation}
N_j \equiv \int n_j ds
\end{equation}
which has units of cm$^{-2}$ and is akin to a surface density
($n_j \to \rho$; $N_j \to \Sigma$).
As an example, consider the column density of 
of O$_2$ through 1\,m of air.  
With $\rho_{\rm O_2} = 1.492$\,g/L,
$N_{\rm O_2} = 3\sci{21} \cm{-2}$.

A quantity of observational interest is the 
optical depth at line-center $\tau_0$ of a gas with $N_j$
and Doppler parameter $b$.
At line-center ($\nu = \nu_{jk})$, our line-profile is 
dominated by Doppler motions
$\phi_V(\nu_{jk}) \approx \phi_D (\nu_{jk})$ and from 
Equation~\ref{eqn:phi_Dv}, we recover
\begin{equation}
\tau_0 =  \frac{\sqrt{\pi} e^2}{m_e c} \, \frac{N_j \lambda_{jk} f_{jk}}{b} \perd
\end{equation}
For Ly$\alpha$, with $b$ expressed in km/s:
\begin{equation}
\tau_0^{\rm Ly\alpha} =  7.6 \sci{-13} \, 
\frac{N \, [\rm cm^{-2}]}{b \, [\rm km/s]}
\label{eqn:tau0}
\end{equation}

\subsection{Idealized Absorption Lines}

Without derivation (see the Lecture notes),
we express the radiative transfer for a source with
intensity $I^*_\nu$ through a medium with
optical depth $\tau_\nu$ simply as

\begin{equation}
I_\nu = I_\nu^* \rme^{-\tau_\nu}
\end{equation}
Therefore, we may consider the formation of absorption lines as the 
simple integration of the optical depth.

% For figures use
%
\begin{figure}[b]
%\sidecaption
% Use the relevant command for your figure-insertion program
% to insert the figure file.
% For example, with the graphicx style use
\includegraphics[scale=.4]{Figures/fig_lya_lines}
%
% If no graphics program available, insert a blank space i.e. use
%\picplace{5cm}{2cm} % Give the correct figure height and width in cm
%
\caption{\lya\ lines
}
\label{fig:lyalines}       % Give a unique label
\end{figure}


In Figure~\ref{fig:lyalines}, we show a series of idealized \HI\ \lya\
lines for a range of \HI\ column densities and Doppler parameters
(\nhi, $b$).  These illustrate the shapes of the line-profiles, here
dominated by Doppler broadening, and the varying optical depth
at line-center.  These are plotted in velocity space, taking
$\delta v = \delta \lambda / \lambda_0$ with 
$\delta v=0$\,km/s corresponding to the line-center.  
In Figure~\ref{fig:dla_compare}
we contrast these lines, which have $\tau_0 \approx 1$, with 
an \HI\ absorption line with very high \nhi\ and correspondingly
high $\tau_0$.  Here the line profile is entirely determined
by Natural broadening.

% For figures use
%
\begin{figure}[b]
%\sidecaption
% Use the relevant command for your figure-insertion program
% to insert the figure file.
% For example, with the graphicx style use
\includegraphics[scale=.4]{Figures/fig_dla_compare}
%
% If no graphics program available, insert a blank space i.e. use
%\picplace{5cm}{2cm} % Give the correct figure height and width in cm
%
\caption{DLA vs.\ \lya\ lines
}
\label{fig:lyalines}       % Give a unique label
\end{figure}

\subsection{Equivalent Width}
\label{subsec:EW}

The equivalent width for absorption 
$W_\lambda$ is a gross measure of the 
flux absorbed (scattered) by the gas cloud.
Strictly, $W_\lambda$ is the convolution of the 
optical depth with the line profile.  
And although it is primarily an observational quantity,
its value is {\it independent} of the instrument profile
and it does depend on physical properties of the
absorbing gas.


% For figures use
%
\begin{figure}[b]
\sidecaption
% Use the relevant command for your figure-insertion program
% to insert the figure file.
% For example, with the graphicx style use
\includegraphics[scale=.4]{Figures/example_ew.pdf}
%
% If no graphics program available, insert a blank space i.e. use
%\picplace{5cm}{2cm} % Give the correct figure height and width in cm
%
\caption{Box car description of EW
}
\label{fig:EWbox}       % Give a unique label
\end{figure}

The value, expressed in \AA, may be
visualized as the width of a box-car profile
that matches the absorbed flux in a normalized 
spectrum, e.g. Figure~\ref{fig:EWbox}.
Analytically, we define

\begin{equation}
W_\lambda = \intl_0^\infty \ltk 1 - \frac{I_\nu}{I_\nu^*} \rtk \, d\lambda
\end{equation}
Substituting our simple radiative transfer equation, this gives

\begin{equation}
W_\lambda = \intl_0^\infty \ltk 1 - \exp(-\tau_\lambda) \rtk \, d\lambda
\label{eqn:EW}
\end{equation}
One may contrast this equation with analysis of stellar 
atmospheres which has
a very different radiative transfer equation.

\subsection{Curve of Growth for \HI\ \lya}
\label{subsec:COGlya}

It is valuable to develop an intuition 
of the relation bbetween equivalent width and the
physical properites of a gas ($N,b$).  This 
relationship is generally referred to as the
curve-of-growth (COG) and may be inverted to
constrain $N,b$ from $W_\lambda$ measurements.
The COG also nicely describes the transition from a 
Doppler-dominated line to a naturally-broadened line.

On typically considers three regimes for the COG
which depend on the central optical depth of the
absorption line $\tau_0$.
In the Weak limit ($\tau_0 \ll 1$), Natural 
broadening is negligible and Doppler broadening
dominates,

\begin{equation}
\tau_\nu = \frac{\pi e^2}{m_e c} \lambda f_{jk} N_j \phi_D(\nu) 
\end{equation}
With $\tau_0$ small, Equation~\ref{eqn:EW} reduces to
\begin{align}
W_\lambda &= \frac{\lambda^2}{c} \int \ltk 1 - \exp(-\tau_\nu) \rtk d\nu \\
  & \approx \frac{\lambda^2}{c} \intl_0^\infty \tau_\nu \, d\nu \\
  & = \frac{\lambda^2}{c} \frac{\pi e^2}{m_e c} f_{jk} N_j 
\end{align}
revealing a linear relationship between $W_\lambda$ and $N_j$
and no dependence on $b$.
The Weak limit, therefore, is also referred
to as the `linear' portion of the COG.
Evaluating for \lya, we find
\begin{equation}
W_\lambda \approx (0.1 \, {\rm \AA}) \, \frac{N_{\rm HI}}{1.83 \sci{13} \cm{-2}} 
\end{equation}
Our estimate of $\tau_0$ (Equation~\ref{eqn:tau0})
requires $\mnhi \ll 10^{14} \cm{-2}$ for $\tau_0 \ll 1$,
which also implies $W_\lambda \ll 1$\AA.
This column density is many orders of magnitude less than 
that typcial of the Galactic ISM.  It implies a gas that 
is extremely diffuse and, likely, highly ionized.  These
are the physical conditions of the IGM.



% For figures use
%
\begin{figure}[b]
\sidecaption
% Use the relevant command for your figure-insertion program
% to insert the figure file.
% For example, with the graphicx style use
\includegraphics[scale=.4]{Figures/ew_sat_fill}
%
% If no graphics program available, insert a blank space i.e. use
%\picplace{5cm}{2cm} % Give the correct figure height and width in cm
%
\caption{Saturated \lya\ line in the Strong regime of the
COG
}
\label{fig:EWstrong}       % Give a unique label
\end{figure}



The Strong line limit of the COG refers to $\tau_0 \gtrsim 1$
and we may still ignore Natural broadening.
In this regime, the light is strongly absorbed at line-center
and the equivalent width is well described by the width
of the line (i.e.\ it is nearly approximated as a `box';
Figure~\ref{fig:EWstrong}).
Expressing the frequency dependence of the optical
depth as $\tau_x = \tau_0 \rme^{-x^2}$, with
$x \equiv \Delta \nu / \Delta \nu_D$, we may estimate the 
line width by 
considering the value of $x$ that gives $\tau_x = 1$.
Trivially, we find $x_1 = \sqrt{\ln \tau_0}$ and therefore

\begin{equation}
W_\lambda \approx 2 x_1 \approx 2 \sqrt{\ln \tau_0}
\label{eqn:EWstrong}
\end{equation}
in the Strong regime, also known as the saturated limit.
To change $W_\lambda$ in this saturated
limit, we need to increase $\tau_0$ immensely
and likewise $N$, i.e.\ $W_\lambda \propto (\ln N)^\ohf$.
Whereas $W_\lambda$ is insensitive to
$\tau_0$, it is sensitive to the internal structure of the cloud
$W_\lambda \propto b$.

Lastly, there is the Damping regime with $\tau_0 \gg 1$
and where Natural broadening dominates.
Now the optical depth is given by the Lorentzian profile

\begin{equation}
\tau_x \approx \frac{\tau_0 A}{\sqrt{\pi}} \frac{1}{x^2}
\end{equation}
and the width (estimated from $\tau_x = 1$) is
$x_1 = \ltk \tau_0 A \rtk^\ohf$.
Therefore, $W_\lambda \approx 2x_1 \propto N^\ohf$
or formally

\begin{equation}
\frac{W_\lambda}{\lambda} \approx \frac{2}{c} \ltk \lambda^2 N_j 
\frac{\pi e^2}{m_e c} f_{jk} A \rtk^\ohf
\end{equation}
In the Damping regime, the equivalent width scales
as $\sqrt{N}$ with no dependence on the Doppler parameter
as the core is fully saturated.

% For figures use
%
\begin{figure}[b]
\sidecaption
% Use the relevant command for your figure-insertion program
% to insert the figure file.
% For example, with the graphicx style use
\includegraphics[scale=.4]{Figures/fig_cog}
%
% If no graphics program available, insert a blank space i.e. use
%\picplace{5cm}{2cm} % Give the correct figure height and width in cm
%
\caption{COG
}
\label{fig:COG}       % Give a unique label
\end{figure}


Figure~\ref{fig:COG} shows the COG for a single
\HI\ \lya\ line with varying \nhi\ and Doppler parameter.
The three COG regimes are well-described.  We stress
that only a small portion of the Weak limit ($W_\lambda \ll 1$\AA)
permits actual detections with modern spectrometers
and that the Damping regime is limited to gas with
galactic surface densities.
This leaves approximately 6 orders of magnitude in 
\nhi\ in the Strong (saturated) regime where observations
of the equivalent width for
\lya\ alone offer a weak constrain on the gas column
density.

\subsection{Curve of Growth for the Lyman Series}

Another application of the COG is to evaluate the
absorption from a series of transitions from as 
single ion in a single `cloud' of gas.
Such analysis may enable one to derive more precisely
the physical parameters ($N,b$) of the absorbing gas.
Recalling that 

\begin{equation}
\tau_0 = \frac{\sqrt{\pi} e^2}{m_e c} \frac{\lambda_{jk} f_{jk} N_j}{b} \cmma
\end{equation}
a single hydrogen gas cloud with fixed $N,b$ will
show decreasing $\tau_0$ in increasing terms of the
Lyman series. Therefore, the Lyman series absorption 
generates a COG that may span several regimes.

% For figures use
%
\begin{figure}[b]
\sidecaption
% Use the relevant command for your figure-insertion program
% to insert the figure file.
% For example, with the graphicx style use
\includegraphics[scale=.4]{Figures/fig_excog}
%
% If no graphics program available, insert a blank space i.e. use
%\picplace{5cm}{2cm} % Give the correct figure height and width in cm
%
\caption{COG fit
}
\label{fig:COGfit}       % Give a unique label
\end{figure}


The standard analysis is to fit COG curves to a
series of $W_\lambda$ measurements to constrain $N,b$
as illustrated in Figure~\ref{fig:COGfit}.
Another example and related code are provided in the
Notebook.


\section{Basics of Spectral Analysis (for Absorption)}
\label{sec:specanaly}

In this section, we provide a brief introduction to 
several key concepts of spectroscopy and then describes
several of the fundamental aspects relatd to absorption-line
analysis with data.

\subsection{Characteristics of a Spectrum}

It is beyond the scope of this Chapter to describe in detail the
fundamentals of spectrometers in astronomy.  For what follows,
the reader need only appreciate that spectrometers' implement
a dispersion element (e.g.\ grating) to redirect incident light 
as a function of wavelength such that a packet of photons with
a distribution of energies are spread (monotonically) across a detector.
The number of photons collected at a given detector pixel depends
on the flux and the dispersion of the spectrograph.
Of course, one cannot achieve infinite resolution and, further, the
optics of the instrument affect the resultant spectral image.

One defines the Line Spread Function (LSF)
as the functional form a monochromatic source exhibits 
on the detector of a spectrometer.
The diffraction of light (e.g.\ slit + dispersing element)
transforms the input PSF of the source (typically a Moffat) into a line 
profile in wavelength.  The precise 
LSF that results depends on type of spectrometer 
\cite[e.g.][]{Robertson2013} as summarized in Table~\ref{tab:LSF}.

\begin{table}[ht]
\begin{center}
\caption{{\sc Line Spread Functions}}
\label{tab:LSF}
\vskip 0.05in
\begin{tabular}{ccc}
\hline
Type of Spectrograph & LSF & R \\
\hline
Diffraction-limited slit & sinc$^2$     & $mN$ \\
Single-mode fiber        & Gaussian     & FWHM \\
Multi-mode fiber         & Half ellipse & ??\\ 
Fabry-Perot etalon       & Lorentzian   & $mF$ \\
\hline
\end{tabular}
\end{center}
\end{table}

In Figure~\ref{fig:LSF}, we compare the idealized LSFs for the
most common spectrometers in use: a diffraction-limited slit 
spectrometer and a fiber-fed spectrograph.
Although the functional forms are distinct and one recognizes
quantitative differences in the wings of the profiles,
the inner line-profiles are very similar and one would
require high signal data to distinguish the two.
Furthermore, system imperfections in the spectrograph, variations
in the PSF, and the Central Limit Theorem tends the LSF 
towards a Gaussian. For this school, we will adopt a Gaussian LSF.
In practice, the LSF may be assessed empirically, e.g.\ through
analysis of arc-line emission lamps and/or 
atmospheric absorption lines imprinted on an external source.


% For figures use
%
\begin{figure}[b]
\sidecaption
% Use the relevant command for your figure-insertion program
% to insert the figure file.
% For example, with the graphicx style use
\includegraphics[scale=.4]{Figures/lsf}
%
% If no graphics program available, insert a blank space i.e. use
%\picplace{5cm}{2cm} % Give the correct figure height and width in cm
%
\caption{Comparison of the LSF for a slit spectrometer (black) with
that for a fiber-fed spectrometer (blue).  Despite quantitative 
differences in the wings of the profiles, the majority of light is
similarly describe in the inner portion of the profile.
}
\label{fig:LSF}       % Give a unique label
\end{figure}


The primary impact of the LSF is to artificially broaden
any spectral features through the convolution of the instrument
profile with the intrinsic spectrum.  While the LSF offers a
complete description of this effect, we tend to describe it
(and the spectrometer) in terms of the width of the LSF, 
i.e.\ the resolution $R$.
The actual metric for $R$ depends on the shape of the 
LSF (i.e.\ Table~\ref{tab:LSF}) and for a Gaussian profile
the standard metric is the full-width at half-maximum (FWHM).
For an optical slit/fiber spectrograph, $R$ is primarily 
dependent on the slit/fiber width (inversely proportional)
and the grating.  It is generally independent of any detector
properties.  Formally, we define

\begin{equation}
R = \frac{\lambda}{\Delta\lambda_{\rm FWHM}}
\label{eqn:R}
\end{equation}
with $\lambda$ the wavelength where the LSF has a measured
FWHM of $\lambda_{\rm FWHM}$.

While $R$ describes the width of the LSF, the spectral
width of a pixel on the detector is given by the dispersion
$\delta \lambda$.  Obviously, this depends on both $R$ 
and the characteristics of the detector (i.e.\ the
physical size of the pixel).  To further confuse matters,
many detectors can be `binned' in electronics
to increase $\delta\lambda$ and reduce the detector noise
per effective pixel.  

The sampling of the LSF is the number of pixels per resolution element,
i.e.\  $\Delta\lambda_{\rm FWHM} / \delta\lambda$.  To achieve the
full resolution of the spectrograph in the theoretical limit,
one must sample the LSF by 2 or more pixels.  This is commonly
referred to as the Nyquist limit.  


% For figures use
%
\begin{figure}[b]
\sidecaption
% Use the relevant command for your figure-insertion program
% to insert the figure file.
% For example, with the graphicx style use
\includegraphics[scale=.4]{Figures/fj0812}
%
% If no graphics program available, insert a blank space i.e. use
%\picplace{5cm}{2cm} % Give the correct figure height and width in cm
%
\caption{Three spectra of the quasar FJ0812+32 observed with the
SDSS, Keck/ESI, and Keck/HIRES spectrometers with resolution ranging
from $R \approx 2,000 - 30,000$.  The lower-right panel shows
a zoom-in where one can easily appreciate the effects of varying
spectral resolution.
}
\label{fig:R}       % Give a unique label
\end{figure}

Commonly used spectrometers for \HI\ absorption have $R$ ranging
from 2,000 (Keck/LRIS; SDSS) to many thousands (echellettes; 
Keck/ESI, VLT/X-Shooter), to tens of thousands
(echelles; Keck/HIRES, VLT/UVES).  Figure~\ref{fig:R}
compares spectra of the quasar FJ0812+32 observed
with spectrometers having a range of $R$.  It is evident
that for the lower resolution spectra (SDSS), the intrinsic
width of the lines have been substantially smeared by the 
instrumental LSF.

Although Equation~\ref{eqn:R} is the formal definition of $R$,
astronomers frequently refer to the FWHM.  In velocity,
$\Delta v_{\rm FWHM} = c/R$, e.g. 
$\Delta v_{\rm FWHM} \approx 10$\,km/s for $R = 30,000$.
This provides a physical intuition for the sensitivity
one can achieve.  Old-timers (and many instrument web-pages)
frequently refer to $\Delta \lambda_{\rm FWHM}$, e.g.\ $\approx 2$\AA\
for SDSS.

\subsection{Continuum Normalization}
Studies of \HI\ in absorption require a background, light source
with emission at FUV wavelengths (and bluer!) to excite \HI\
at \lya\ and higher order transitions.
For extragalactic research, this has included the O and B stars
in star-forming galaxies \cite{lee16}, the bright afterglows
of gamma-ray bursts \cite{fynbo1X}, and quasars (the focus
of this Chapter).   To estimate the opacity, one must estimate
the source continuum $f^C_\lambda$ then normalize the flux, 

\begin{equation}
\bar f_\lambda = \frac{f_\lambda^{\rm obs}}{f_\lambda^C}
\end{equation}
In essence, we aim to remove any trace of the background
source as it is scientifically irrelevant to the analysis.
This process is referred to as continuum normalization and
it is often the dominant systematic uncertainty in absorption
analysis.

To date, quasars have been the most commonly observed background
sources because they are the most common, steadily luminous, distant
phenomena.   Unfortunately, they may also exhibit the most complexity
in their spectral energy distribution (SED).  This includes
very strong and wide emission lines, an underlying power-law
continuum with varying exponent, intrinsic absorption from gas
associated with the quasars, and a plethora of unresolved emission
features (e.g.\ Figure~\ref{fig:qso_SED}.


% For figures use
%
\begin{figure}[b]
\sidecaption
% Use the relevant command for your figure-insertion program
% to insert the figure file.
% For example, with the graphicx style use
\includegraphics[scale=.4]{Figures/qso_sed}
%
% If no graphics program available, insert a blank space i.e. use
%\picplace{5cm}{2cm} % Give the correct figure height and width in cm
%
\caption{Three quasar spectra from the SDSS survey
which demonstrate diversity in the degree of associated
absorption due to highly ionized gas in the quasar
environment.
}
\label{fig:qso_SED}       % Give a unique label
\end{figure}

The challenge is compounded by the fact that no well-developed physical
model exists for the quasar SED.  The phenomenon is well-modeled
by a combination of synchrotron emission, optically thick
emission lines arising in a wind, a hot accretion disks 
emitting preferentially at UV wavelengths, and a partially
obscuring dust torus that radiates at IR wavelengths.
But each of these is at best empirically parameterized, may
contribute largely independent of one another, and 
with great diversity.  Furthermore, the emission is
poorly constrained at $\lambda_{\rm rest} < 900$\AA\
in part because the IGM greatly absorbs the SED
making it very difficult to assess the intrinsic flux 
\cite[see][]{lusso+16}.

% For figures use
%
\begin{figure}[b]
\sidecaption
% Use the relevant command for your figure-insertion program
% to insert the figure file.
% For example, with the graphicx style use
\includegraphics[scale=.4]{Figures/qso_template}
%
% If no graphics program available, insert a blank space i.e. use
%\picplace{5cm}{2cm} % Give the correct figure height and width in cm
%
\caption{Comparison of the quasar composite spectrum derived
from {\it HST} spectra of $z \sim 1$ quasars 
\cite{telfer2001} with a composite spectrum derived
primarily from $z \sim 2$ quasars taken from the SDSS survey
\cite{vandden0X}.  Each composite is normalized to unity
at $\lambda_{\rm rest} \approx 1450$\AA.
While there are notable differences in the
strength and widths of the emission line features (dominated
by the so-called Baldwin effect), the overall SED is remarkably
similar.  
}
\label{fig:qso_template}       % Give a unique label
\end{figure}



Despite all of these vagaries and variabilities, the average
quasar spectrum exhibits remarkably little evolution across
cosmic time.  Figure~\ref{fig:qso_template}
compares composite quasar SED spectra generated by
averaging tens of $z \sim 1$ sources \cite{telfer01}
and thousands of $z \sim 2$ sources \cite{vanden01}.
There are significant differences in the broad emission
lines (e.g.\ \lya, CIV~1550), but the underlying
SED is very similar between the two samples.  This commonality
holds to the highest redshifts where quasars are
observed \cite[Figure~\ref{fig:becker13_fig2}][]{becker+13}.
This offers hope that quasar continua can at least
empirically be estimated \cite{new_paper_by_AZ_guy}.



% For figures use
%
\begin{figure}[b]
\sidecaption
% Use the relevant command for your figure-insertion program
% to insert the figure file.
% For example, with the graphicx style use
\includegraphics[scale=.4]{Figures/becker13_fig2}
%
% If no graphics program available, insert a blank space i.e. use
%\picplace{5cm}{2cm} % Give the correct figure height and width in cm
%
\caption{Composite spectra for quasars drawn
from the SDSS over redshifts $z \approx 2.2-5$
by \cite{becker+13}.  Note the closer similarity
in the SEDs across these several Gyr.  The obvious
exception is at $\lambda_{\rm rest} < 1215$\AA, where
the IGM opacity increase with increasing redshift. 
We will return to an analysis of the spectra in this
figure in the following section.
}
\label{fig:becker13_fig2}       % Give a unique label
\end{figure}

Historically, and even today \cite[e.g.]{kodiaq_dr2}, the majority of 
continuum normalization has been performed manually.  Redward
of \lya\ emission, where absorption from intervening gas is
minimal (e.g.\ Figure~\ref{fig:qso_SED}), one can identify
unabsorbed regions (by-eye or by-algorithm) to fit the
continuum with a model.  A higher order ($n \sim 7$)
polynomial suffices over each span of $\approx 100$\AA\ to
capture undulations due to underlying emission features.
One can routinely achieve a precision of several percent,
depending on the data quality.  An example in the 
{\it Spectral\_Analysis} Notebook shows the call to a 
GUI used for continuum fitting.

Blueward of \lya\ emission, where the IGM opacity is
substantial, continuum estimation is very difficult.
Human (by-eye) analysis typically involves spline evaluations
through spectral regions purportedly free of IGM absorption.
But even at echelle resolution, there may be not a single
pixel unscathed by the IGM.

Modern analysis has aimed to eliminate the human asepct
by empirically predicting the quasar continuum.  
\cite{suzuki06} were the first to perform a Principal
Component Analysis (PCA) of quasar continua.  This 
mathematical technique analyzes a cohort of individual
spectra (they used the same $z \sim 1$ quasar spectra
of the Telfer composite) to calculate the eigenvectors
that best describe variations in the spectra off the mean.
Formally, we have
\begin{equation}
\ket{q_i} = \ket{\mu} + \smm_{j=1}^{m} c_{ij} \ket{\xi_j}
\label{eqn:PCA}
\end{equation}
where $\ket{q_i}$ is any quasar spectrum,
$\ket{\mu}$ is the mean (composite) spectrum,
$\ket{\xi_j}$ are the Principal Components, and
$c_{ij}$ are the eigenvalues.
These eigenvectors are orthogonal and generally
have little physical meaning.  The eigenvectors
derived by \cite{susuki06}
are shown in Figure~\ref{fig:PCA} and they found
the first 3 PCA components accounted for over
80\%\ of the observed variance in the quasar spectra.

% For figures use
%
\begin{figure}[b]
\sidecaption
% Use the relevant command for your figure-insertion program
% to insert the figure file.
% For example, with the graphicx style use
\includegraphics[scale=.4]{Figures/suzuki06_fig2}
%
% If no graphics program available, insert a blank space i.e. use
%\picplace{5cm}{2cm} % Give the correct figure height and width in cm
%
\caption{Each of the sub panels shows one PCA eigenvector 
$\xi_j$  (left) 
and the distribution of eigenvalues $c_{ij}$ (right) for the 
$z\sim 1$ quasars analyzed by \cite{suzuki06}.
The eigenvectors from top-left to lower-right are ordered
in decreasing relevance.
Taken from \cite{suzuki06}.
}
\label{fig:PCA}       % Give a unique label
\end{figure}

The modern approach to quasar fitting, especially in 
lower resolution spectra, is to fit the quasar SED
with these PCA eigenvectors outside the \lya\ forest
($\lambda_{\rm rest} > 1215.67$\AA) and then extrapolate
the model to wavelengths impacted by the IGM.
A more recent example of this technique is described
in \cite{paris11}.  We also refer the student to analysis
that includes Mean Flux Regulation \cite{lee11} where the
PCA estimate is further refined in the \lya\ forest by
imposing that the IGM opacity match the average (which
we derive in a later section).

\subsection{Equivalent Width Analysis}

Now that we have normalized the spectrum by the
intrinsic source continuum, we may proceed to analyze
the observed absorption.  The fundamental observable
of absorption strength is the equivalent width $W_\lambda$.
Observationally, this metric describes the fraction of
incident flux absorbed by the gas.  Empirically, there
are two standard approaches to estimating $W_\lambda$.
A non-parametric evaluation may be performed with
boxcar integration, i.e.,
the simple summation over the pixels covering the absorption line.
With the flux continuum normalized  (and expressed as $\bar f$),
we have
\begin{equation}
W_\lambda = \smm_i (1- \bar f_i) \delta\lambda_i
\label{eqn:EWtwo}
\end{equation}
where $\delta\lambda_i$ is the dispersion of each pixel.
The uncertainty follows from propagation of error (neglecting
continuum uncertainty),
\begin{equation}
\sigma^2(W_\lambda) = \smm_i \sigma^2(\bar f_i) (\delta\lambda_i)^2
\label{eqn:sigEW}
\end{equation}

One may also adopt parametric techniques by fitting a model
of the line-profile to the observed absorption.  This is
referred to as Line fitting.  Most \HI\ absorption features
(aside from the very strongest) may be
well approximated by a (upside-down) Gaussian.  Or by   
considering $(1-\bar f)$ one can fit with
\begin{equation}
g(\lambda) = A \exp[- (\lambda-\lambda_0)^2 / (2 \sigma_\lambda^2)]
\end{equation}
and calculate the equivalent-width as the 
area under the curve
\begin{equation}
W_\lambda = A \sigma_\lambda \sqrt{2 \pi}
\end{equation}
Uncertainty should include the co-variance between $A$ and $\sigma$
\begin{equation}
\sigma(W_\lambda) = W_\lambda \sqrt{\sigma^2(A)/A^2 +
		\sigma^2(\sigma_\lambda)/\sigma_\lambda^2 + 2 \sigma(A)\sigma(\sigma_\lambda)/[A \sigma_\lambda]}
\end{equation}
We show examples of calculate $W_\lambda$ in the
{\it Spectral\_Analysis} Notebook.

Analysis of $W_\lambda$ is 
generally performed in the observer (i.e.\ measured) frame,
but $W_\lambda^{\rm obs}$ depends on $\lambda$ and therefore redshift.
To relate $W_\lambda$ to physical quantities (e.g.\ $N,b$ via a
curve-of-growth analysis),  one requires a shift to the rest-frame
\begin{equation}
W_\lambda^{\rm rest} = \frac{W_\lambda^{\rm obs}}{1+z}
\end{equation}

When designing an absorption-line observation, one generally
must estimate the signal-to-noise (S/N) required to achieve
the desired sensitivity.   The latter may follow from a
desired equivalent width limit $W_\lambda^{\rm lim}$.
We relate the two quantities by asking (and answering):
{\it What is the limiting equivalent width one can measure from
data with a given set of spectral characteristics?}
Returning to our boxcar estimate of the uncertainty 
(Equation~\ref{eqn:sigEW}), we 
assume a constant dispersion $\delta\lambda$ and
recognize that $\sigma(\bar f)$ is equivalent to 
the inverse of the S/N per pixel for normalized data.
If the S/N is roughly constant across the absorption,
then the terms inside the sum in Equation~\ref{eqn:sigEW}
come out and the sum evaluates to $M$, the number of pixels
across the line, 
\begin{equation}
\sigma(W_\lambda) = \frac{\sqrt{M} \delta\lambda}{S/N}
\label{eqn:sigSN}
\end{equation}
For a $3\sigma$ detection, we relate 
$W_\lambda^{\rm lim} = 3 \sigma(W_\lambda)$ and we can invert
Equation~\ref{eqn:sigEW} to calculate S/N:

\begin{equation}
S/N = 3 \frac{\sqrt{M} \delta\lambda}{W_\lambda^{\rm lim}}
\label{eqn:SN}
\end{equation}
The lingering unknown is the number of pixels in the
integration.
If the line is unresolved (or barely resolved), 
$M$ is simply the sampling ($\Delta\lambda_{\rm FWHM} / \delta\lambda$).
Otherwise, $M$ depends on the intrinsic line-width. 

We may also characterize the data quality in terms of a
limiting equivalent width.  As an example, consider 
\HI\ Ly$\alpha$ in SDSS spectra (see also the Notebook).
For $R=2,000$, S/N=10 per pixel, and a sampling of 2 pixels,
we estimate  $W_{\rm lim}^{3\sigma} \approx 1$\AA.
This falls on the saturated portion of the COG for \lya\
(Figure~\ref{fig:COG}).  One expects, therefore,
to have  poor sensitivity to \nhi\ in such spectra. 
As another example, consider the required S/N for a 
30m\AA\ detection at $3\sigma$ significance in an echelle
spectrum with $R=30,000$ and 3 pixel sampling.  Using
Equation~\ref{eqn:SN}, we estimate that a 
S/N$\approx 10$ is desired.

\subsection{Line-Profile Analysis}

While $W_\lambda$ offers an intuitive, observationally based
measure of the absorption, it is difficult to translate
such measurements to physical quantities, i.e.\ the column
density and/or kinematics of the gas.  At high spectral
resolution ($R > 20,000$), one may fully resolve
the line profile and estimate the optical depth directly.
In turn, a line-profile analysis yields the line-centroid
(i.e.\ redshift) and the $N,b$ values described in the
previous section.
A full discussion of line-profile fitting techniques is
beyond the scope of this Chapter.  We refer the reader
to a simple example in the {\it Spectral\_Analysis} Notebook.
And we further refer the reader to two of the widely adopted
packages for line-profile analysis:
VPFIT (developed by R. Carswell):  {\tt http://www.ast.cam.ac.uk/$\sim$rfc/vpfit.html}
and ALIS (R. Cooke): {\tt https://github.com/rcooke-ast/ALIS}.
Each package performs $\chi^2$ minimization on a set of
input absorption lines to derive physical quantities.
We will see examples of the results from such analysis in the
following sections.


%%%%%%%%%%%%%%%%%%%%%%%%%%%%%
\begin{figure}[ht]
\sidecaption
% Use the relevant command for your figure-insertion program
% to insert the figure file.
% For example, with the graphicx style use
\includegraphics[scale=.8]{Figures/evolving_forest_in_chapter}
%
% If no graphics program available, insert a blank space i.e. use
%\picplace{5cm}{2cm} % Give the correct figure height and width in cm
%
\caption{Snapshots of the \lya\ forest in a series of spectra
with increasing source redshift.  Note the evolution from a nearly
transparent \lya\ forest at $z \approx 0$ to essentially 
zero transmission at $z \approx 6$.
}
\label{fig:lyf_z}       % Give a unique label
\end{figure}

\clearpage

%%%%%%%%%%%%%%%%%%%%%%%%%%%%%%%%%%%%%%%%%%%%%%%%
\section{The Fluctuating and Evolving Opacity of the
\lya\ Forest $f(\mnhi,z)$}

The term \lya\ forest refers to the thicket of absorption
observed in the spectra of a distant source at 
$\lambda_{\rm rest} < 1215.67$\AA\ in the source rest-frame.
The absorption lies blueward of the source \lya\ emission
because these photos redshift while traveling to the
intervening \HI\ gas at $z < z_{\rm source}$ to then
scatter at \lya\ in the rest-frame of the gas.
Figure~\ref{fig:lyf_z} shows \lya\ forest spectra for
sources ranging from $z \approx 0-6$.  Two aspects are
immediately obvious: 
(i) the opacity fluctuations are substantial at any given
redshift;
(ii) there is a significant increase in the average opacity
with increasing redshift.  

In this section, we introduce
the standard approaches used to characterize the \lya\ 
forest opacity.  These techniques are primarily empirical
although the results offer valuable constraints on modern
cosmology.
Fundamentally, any precise measurement from the IGM offers a 
 valuable constraint on our cosmology
Because we have a well-developed theory of structure formation (e.g.\ N-body
simulations) and
on quasi-linear scales, the baryons track the dark matter,
one may predict the \lya\ opacity provided an estimate
of the universe's ambient
radiation field (which may also be constrained).
This section focuses on results for gas that is
optically thin at the \HI\ Lyman limit (i.e.\ $\mnhi<10^{17} \cm{-2}$,
as derived in the following section).  We note further
that true intergalactic gas 
likely corresponds to even lower column densities
($\mnhi <10^{14} \cm{-2}$).

% For figures use
%
\begin{figure}[b]
\sidecaption
% Use the relevant command for your figure-insertion program
% to insert the figure file.
% For example, with the graphicx style use
\includegraphics[scale=.6]{Figures/kirkman97_fig1}
%
% If no graphics program available, insert a blank space i.e. use
%\picplace{5cm}{2cm} % Give the correct figure height and width in cm
%
\caption{Examples of a series of fits to lines in
high S/N, echelle spectra of the \lya\ forest.
Taken from \cite{kt97}.
}
\label{fig:kirkman97_fig1}       % Give a unique label
\end{figure}
        
\subsection{\nhi\ frequency distribution \fnhi: Concept and Definition}

Motivated by the discrete appearance of lines in the 
\lya\ forest (e.g. Figure~\ref{fig:lyf_z}), the
first approach developed was to model the gas
as a series of absorption lines.  Each line has
three physical parameters ($N,b,z$) and 
the \lya\ forest is described by the distributions of
$N$ and $b$ across cosmic time.  This technique demands
echelle resolution spectra and high S/N to precisely 
constrain the absorption-line parameters.

In practice, one laboriously fits the individual lines in a set of
spectra (as described in the previous section).  
Figure~\ref{fig:kirkman97_fig1} shows a snippet of spectra
analyzed in this manner by \cite{kt97}.  The approach has the
positive features of 
capturing the stochastic nature of the \lya\ Forest and
its absorption-line appearance.  Furthermore, it is 
primarily model-independent aside from the underlying
ansatz that the gas arises primarily in discrete lines.
On the other hand, the $N,b,z$ distributions do not
represent a physical model and similar quantities are not
easily derived from actual models or simulations of the IGM.
Furthermore, the analysis tends to 
be very expensive (laborious), any
human interaction implies non-reproducible results,
and the data required is expensive to obtain
demanding many hours of integration with 
echelle spectrometers on 10-m class telescopes.
Nonetheless, it is the most precise description of the IGM.

Formally, one defines a frequency distribution for the
lines of the \lya\ forest
$f(\mnhi, b, z) \, d\mnhi \, db \, dz $, the number of lines on
average in the intervals $\mnhi, \mnhi+d\mnhi$; $b, b+db$; $z, z+dz$.
This distribution function is
akin to a luminosity function $\phi(L) dL$, which describes
the average number of galaxies per volume in a luminosity interval $L, L+dL$.
The absorption, however, occurs along 
sightlines instead of within a volume.  
A standard assumption adopted in most analyses is that
the Doppler parameter ($b$-value) distribution has 
minimal \nhi\ (or $z$) dependence, i.e.\ the \nhi\ and $b$-value
distributions are separable,

\begin{equation}
f(\mnhi,b) = \mfnhi \, g(b)
\end{equation}
Indeed, in the following, we discuss these separately.

\subsection{Binned Evaluations of \fnhi}
We may estimate \fnhi\ from a set of absorption-line 
measurements as follows. Consider first, a single sightline
to a source at $z=3$.  A spectrum spanning $\lambda = 3500-5000$\AA\
will cover \lya\ absorption at $z \approx 2-3$.
One may slice the sightline into multiple redshift intervals,
e.g. each with $\delta z = 0.5$.  To bolster the statistics
for evaluating \fnhi\ at a given redshift, we repeat the
experiment for many $\mathcal{N}$
sightlines and cover a survey path 
(see the {\it characterizing\_lya\_forest} Slides for
an illustration):

\begin{equation}
\Delta z = \smm_i^{\mathcal{N}} (\delta z)_i
\label{eqn:survey_path}
\end{equation}
Later in this chapter,
I will define the $\Delta X$ survey path which offers
a more physical definition than the observational redshift path.

The simplest estimator of \fnhi\ is a binned evaluation
within a finite $\Delta \mnhi$ interval
large enough to have statistical significance.
We evaluate,
\begin{equation}
f(\mnhi,z) = \frac{\rm \# \; lines \; in \; [\Delta \mnhi,\Delta z]}{
\Delta\mnhi \, \Delta z}
 \end{equation}
where again $\Delta z$ is the survey path for lines
in the $\Delta \mnhi$ interval.  Uncertainty is simply
assumed to follow Poisson statistics. 
An important subtlety with this estimator is that 
evolution in \fnhi\ with \nhi\ or $z$ 
skews the detection within a bin (to lower \nhi and higher $z$).
Therefore, simple model fitting with $\chi^2$ analysis using the 
center of the bins will be (marginally) wrong.

% For figures use
%
\begin{figure}[b]
\sidecaption
% Use the relevant command for your figure-insertion program
% to insert the figure file.
% For example, with the graphicx style use
\includegraphics[width=2.7in]{Figures/tytler87_fig2.pdf}
\includegraphics[width=2.7in]{Figures/petitjean93_fig1.pdf}
% If no graphics program available, insert a blank space i.e. use
%\picplace{5cm}{2cm} % Give the correct figure height and width in cm
%
\caption{Binned evaluations of \fnhi\ from early data sets
on the \lya\ forest (left) \cite{tytler87}, (right) \cite{petitjean93}.
}
\label{fig:binfN}       % Give a unique label
\end{figure}

Early results adopting this approach are shown in 
Figure~\ref{fig:binfN}.  There is a steep and monotonic
decrease in \fnhi\ with increasing \nhi.  As the figures
illustrate, over many orders of magnitude in \nhi, the
data are reasonably  well-described with a power-law.
This led \cite{tytler87}
to propose a single-population of gas `clouds'.
In contrast, \cite{petitjean93} observed significant
departures from a single power-law and 
argued for distinct populations. Indeed, the
modern data and interpretation is even more complex.

With the commissioning of an echelle spectrometer on
the 10-m Keck I telescope \cite[HIRES]{vogt94}, 
observers had access to exquisite data on multiple 
sightlines to assess \fnhi.  Figure~\ref{fN_kt97}
show early results from \cite{kt97} derived from
line-profile fitting analysis along the sightline
to HS~1946+7658 observed to very high S/N.  
Their results show the decline in \fnhi\ for
$\mnhi > 10^{12} \cm{-2}$ as in previous work, but 
also an apparent turn-over in \fnhi\ at lower \nhi. 

This turn-over could reflect a lack of sensitivity 
(i.e.\ incompleteness) as one notes that \HI\ \lya\
with  $\mnhi = 10^{12} \cm{-2}$ has $\tau_0 \approx 0.025$.
However, the data are exquisite (see Figure~\ref{fig:kirkman97_fig1})
and a careful assessment of incompleteness was performed.
Another possibility is that they have underestimated the
quasar continuum; indeed even a few percent error in
in the continuum would overwhelm the signal at these
optical depths.  Again, the data quality (S/N~$>100$)
is such that this is not the most likely explanation.
Instead, we are likely witnessing the transition from
describing the \lya\ forest by individual lines to 
a more continuous opacity.  We return to this concept
towards the end of the section.

Modern estimates of \fnhi\ now include tens of sightlines
with exquisitely small statistical error \cite{rudie13,kim13}.
The results, however, are not in uniform agreement suggesting
that systematic error is the limiting factor.

%\includegraphics[width=5.0in]{Paper_figs/prochaska14_fig5.pdf}


% For figures use
%
\begin{figure}[b]
\sidecaption
% Use the relevant command for your figure-insertion program
% to insert the figure file.
% For example, with the graphicx style use
\includegraphics[scale=0.4]{Figures/kirkman97_fig3.pdf}
%
% If no graphics program available, insert a blank space i.e. use
%\picplace{5cm}{2cm} % Givethe correct figure height and width in cm
%
\caption{Estimation of \fnhi\ from Keck/HIRES spectroscopy of the
very bright quasar HS~1946+7658 by \cite{kt97}.
}
\label{fig:fN_kt97}       % Give a unique label
\end{figure}

\subsection{Models for \fnhi}

The binned evaluations of \fnhi\ describe the shape of 
distribution function and have motivated empirical
models. These models reduce the binned evaluations to a few, 
well-constrained parameters.  In turn, one can
examine redshift evolution in the amplitude and shape
of \fnhi\ and use the model to evaluate \fnhi\ at 
any \nhi\ for calculations on the \lya\ forest (see below).
We emphasize, however, that we have no {\it a priori}
physical model for \fnhi\ and therefore there is the
freedom to introduce any functional form.
We also note that the results will depend
on the \nhi\ range considered. 

As the observational constrains have improved, the
functional forms adopted for \fnhi\ have evolved from
a single power-law \cite{tytler87},
$\mfnhi = B N^\beta$
to broken power-laws \cite{petitjean93,pro+10},
to broken, disjoint power-laws \cite{rudie13} 
to the sum of several functional forms \cite{inoue14},
to spline evaluations \cite{pro+14}.

In the early days, simple $\chi^2$ analysis
was sufficient but these results
are sensitive to the choice of bins and did
not account for evolution within the bin.
A more robust approach is to apply Maximum
Likelihood techniques \cite{storrie96;cooksey2010}.

Consider a sightline sample with a survey total path $\Delta z$ 
covering the interval $\delta z$ at redshift $z$ (refer
back to Equation~\ref{eqn:survey_path}).
We wish to model \fnhi\ at this redshift over a
range of \nhi.  
Divide the distribution space into $M$ cells, each with 
`volume': $\delta v = \delta z \, \delta \mnhi$.
Here, we have divided the range in \nhi\ into small intervals
and have assumed that \fnhi\ does not evolve in the small
redshift interval $\delta z$ considered.
The expected number of lines $\mu_i$ in $i$th cell
follows from our definition of \fnhi, 
\begin{equation}
\mu_i = f(\mnhi)_i \, \Delta z \, \delta\mnhi
\end{equation}
where we consider the full survey path $\Delta z$.
Therefore, the probability of detecting $m$ absorbers within 
cell $i$ is given by Poisson statistics,
\begin{equation}
P(m; \mu_i) = {\rm e}^{-\mu_i} \frac{\mu_i^m}{m!}
\end{equation}

We construct a Likelihood function as the 
simple product over all cells
\begin{equation}
\mathcal{L} = \prodl_i^M P(m; \mu_i) \;\; .
\end{equation}
Now perform what appears to be a swindle.
Reduce the volume of the cell so that each one contains
at most one absorber.  This is achieved by letting
$\delta\mnhi \to 0$.
In this case, $P(m;\mu_i)$ reduces to ${\rm e}^{-\mu_i}$ for an
empty cell and ${\rm e}^{-\mu_i} \mu_i$ for a cell with 1 system.
We may then evaluate the likelihood
by summing over all $M$ cells.  Explicitly,  
let there be $p$ lines detected giving $g=M-p$ empty cells.
It follows that 

\begin{align}
\mathcal{L} &= \prodl_{i=1}^g {\rm e}^{-\mu_i} \;
\prodl_{j=1}^p {\rm e}^{-\mu_j} \, \mu_j \\
&= \prodl_{i=1}^M {\rm e}^{-\mu_i} \; \prodl_{j=1}^p \mu_j
\end{align}
and the log-Likelihood is expressed as

\begin{align}
\ln \mathcal{L} &= \smm_i^M -\mu_i \, + \, \smm_i^p \ln \mu_i \\
&= \smm_i^M -f(\mnhi)_i \Delta z \, \delta\mnhi +
\smm_j^p \ln f(\mnhi)_j \Delta z + p \ln [\delta\mnhi]
\end{align}
In our limit with $\delta\mnhi \to 0$,
we ignore the last term (a constant) and take the integral form
of the first term to derive

\begin{equation}
\ln \mathcal{L} = - \intl_{N_{min}}^{N_{max}} f(\mnhi) \Delta z \, d\mnhi 
\, + \, \smm_{j=1}^p \ln f(\mnhi)_j \Delta z
\end{equation}
If $\Delta z$ is independent of \nhi\ it may be ignored,
but see \cite{cooksey10} for a treatment where
$\Delta z$ is dependent on \nhi.

Lastly, we maximize $\mathcal{L}$ for the parameterization of \fnhi\
to derive the ``best'' values for the parameters.
We emphasize that the resultant model need not 
provide a good description of the data.  It is simply the 
best model for the functional form imposed.
The assessment of goodness of fit requires another statistical
test (see the following section).

With modern data and the assumption of a simple
power-law, \cite{rudie13} report
$\mfnhi \propto \mnhi^{-1.65 \pm 0.02}$ for
$\mnhi = 10^{13.5} - 10^{17} \cm{-2}$ at $z \approx 2.5$.
With a similar but distinct dataset 
\cite{kim13} derive
$\mfnhi \propto \mnhi^{-1.52 \pm 0.02}$
for $\mnhi = 10^{12.75} - 10^{18} \cm{-2}$
at $z \approx 2.8$.
Formally, these are statistically incompatible.
Progress with the \fnhi\ approach
requires resolving systematic errors, including
human biases (ugh).

\subsection{$b$-value Distribution}

The results from  line-profile fitting also yield 
a distribution of measured $b$-values.
Data from \cite{kt97} are shown in 
Figure~\ref{fig:kt97_bN_scatter} where we identify
no strong dependence on \HI\ column density aside from
a possible increase in the minimum $b$-value with increasing \nhi.
The distribution of $b$-values is dominated by lines with
$b \approx 20-30$\,km/s.

% For figures use
%
\begin{figure}[b]
\sidecaption
% Use the relevant command for your figure-insertion program
% to insert the figure file.
% For example, with the graphicx style use
\includegraphics[scale=0.4]{Figures/kirkman97_fig5.pdf}
%
% If no graphics program available, insert a blank space i.e. use
%\picplace{5cm}{2cm} % Givethe correct figure height and width in cm
%
\caption{Scatter plot of $b$ vs.\ \nhi\ measurements for
the absorption lines analysed along the sightline
to quasar HS~1946+7658 by \cite{kt97}.
}
\label{fig:kt97_bN_scatter}       % Give a unique label
\end{figure}

\cite{hui99} derived a functional form for the $b$-value
distribution based on theoretical expectations (fluctuations
in the IGM optical depth):
\begin{equation}
g(b) = \frac{4 b_\sigma^4}{b^5} \, \exp \ltp - \frac{b_\sigma^4}{b^4} \rtp
\end{equation}
This model is described by a single parameter $b_\sigma$.
Fitting to the results in Figure~\ref{fig:kt97_bN_scatter}
they achieved a 
good description of the observations with $b_\sigma = 26.3$ km/s
(Figure~\ref{fig:hui_fig1}).
This yields an average $b$-value
\begin{equation}
<b> = \frac{\int b g(b) \, db}{\int g(b) \, db} = b_\sigma \Gamma(3/4)
\approx 32 \, {\rm km/s}
\end{equation}

% For figures use
%
\begin{figure}[b]
\sidecaption
% Use the relevant command for your figure-insertion program
% to insert the figure file.
% For example, with the graphicx style use
\includegraphics[scale=0.4]{Figures/hui99_fig1.pdf}
%
% If no graphics program available, insert a blank space i.e. use
%\picplace{5cm}{2cm} % Givethe correct figure height and width in cm
%
\caption{Fit by \cite{hui99} to the observed $b$-value distribution of
lines in the \lya\ forest \cite{kt97}.  This one-parameter
model has an average $b$-value of 32\,km/s.
}
\label{fig:hui99_fig1}       % Give a unique label
\end{figure}

It is evident in both of the above figures that the
\lya\ forest lines exhibit a lower limit to the 
measured $b$-values.  This is  $b_{\rm min} \approx 18$\,km/s.
What sets this apparent lower bound to the $b$-value (which is 
well above the spectral resolution of the instrument)?
Recalling that $b = \sqrt{\frac{2kT}{m_A} + \xi^2}$
with $\xi^2$ a characteristic turbulence of the gas,
one is tempted to associate $b_{\rm min}$ to the IGM temperature.
For purely thermal broadening of Hydrogen, we have
\begin{equation}
T = 10^4 \, {\rm K} \; \ltp \frac{b}{13 \, \rm km/s} \rtp^2
\end{equation}
and may estimate $T \approx 20,000$\,K.
We comment on modern estimations in the final section.

One also notes an upper bound to the distribution
of $b_{\rm max} \approx 100$\,km/s.
It is difficult to estalish whether this represents
a sensitivity limit (recall $\tau_0 \propto b^{-1}$)
or a physical limit.

\subsection{Line Density (the incidence of lines in the \lya\ forest)}
Referring back to Figure~\ref{fig:lyf_z}, the average 
opacity of the \lya\ forest clearly evolves with redshift.
In our description of the IGM as a series of lines, this
implies the evolution in the line density (or incidence).
Confusingly, the literature is abound with notation for
this quantity: $dN/dz, n(z), N(z)$.
I have adopted my own:  $\ell(z) dz $ defined as the
number of lines detected on average in the interval 
$z, z+dz$ over an interval of 
column density $\mnhi = [N_{\rm min}, N_{\rm max}]$.
This is the zeroth moment of our frequency distribution,

\begin{equation}
\ell(z) dz = \intl_{N_{\rm min}}^{N_{\rm max}} f(\mnhi,z) \, d\mnhi \, dz
\label{eqn:loz}
\end{equation}
It is akin to the number density of galaxies derived from a 
luminosity function.

Taking the results for \fnhi\ from \cite{kim13} at $z \approx 2.8$,
and integrating from $N_{\rm min} = 10^{12.75} \cm{-2}$
to $N_{\rm max} = 10^{17} \cm{-2}$, we estimate
\begin{equation}
\ell(z \approx 2.8) = \frac{10^{9.1}}{0.52} \mnhi^{-0.52} 
|_{N_{\rm min}}^{N_{\rm max}} \approx 560
\end{equation}
This integral is dominated, of course, by the lowest 
\nhi\ systems.  If we consider a 5\AA\ patch of spectrum at 4600\AA,
we estimate $\delta z = (1+z) (\delta\lambda/\lambda) = 0.004$
and predict $\mathcal{N} = \ell(z) \, \delta z = 2.4$ lines
on average.

% For figures use
%
\begin{figure}[b]
\sidecaption
% Use the relevant command for your figure-insertion program
% to insert the figure file.
% For example, with the graphicx style use
\includegraphics[width=4.0in]{Figures/kim13_fig9.pdf}
%
% If no graphics program available, insert a blank space i.e. use
%\picplace{5cm}{2cm} % Givethe correct figure height and width in cm
%
\caption{Evolution in the incidence of \lya\ forest lines
with redshift for two intervals of \nhi.  As inferred from a visual
inspection of the data (Figure~\ref{fig:lyf_z}, 
the incidence decreases steeply with decreasing redshift.
}
\label{fig:kim13_loz}       % Give a unique label
\end{figure}


Binned evaluations of $\ell(z)$ from \cite{kim13}
are shown in Figure~\ref{fig:kim13_loz}.
where one observes strong redshift evolution, 
as evident in individual spectra.
Their analysis also indicates differing behavior
for lines of differing \nhi\ implying a likely
evolution in the shape of \fnhi.

As with the shape of \fnhi\ at a given $z$,
we can model the redshift evolution in $\ell(z)$,
which is also the redshift evolution in the
normalization of \fnhi.  Here we have some
physical guidance on the functional form
(see also \cite{meiksin09}).
Imagine a population of absorbers with physical (proper)
number density $n_p(z)$ and proper cross-section $A_p(z)$,
e.g.\ a population of spherical absorbing cows.
On average, we expect to intersect
$\ell(r) = n_p(z) A_p(z)$ absorbers per proper path length $d r_p$.
In our Cosmology,

\begin{equation}
\frac{dr_p}{dz} = \frac{c}{H(z) (1+z)} \;\; .
\label{eqn:drdz}
\end{equation}
Recognizing $\ell(r) dr = \ell(z) dz$, we have

\begin{equation}
\frac{dN}{dz} = n_p(z) A_p(z) \frac{c}{H(z) (1+z)}
\end{equation}
The Hubble Parameter in a flat $\Lambda$CDM universe
is given by

\begin{equation}
H(z) = H_0 \sqrt{\Omega_m (1+z)^3 + \Omega_\Lambda}
\end{equation}
and at $z>2$, the universe is matter dominated and 
$\Omega_\Lambda$ may be ignored.  This gives
$H(z) \approx H_0 \Omega_m^{1/2} (1+z)^{3/2}$

Now ansatz that the absorbing gas (cows) is
a constant comoving population, 
$n_p(z) = n_c(z) (1+z)^{3}$, with a constant
physical size.  This gives

\begin{equation}
\ell(z) \propto n_c(z) A_p(z) (1+z)^{\ohf}
\end{equation}
which (on its own) implies a relatively shallow
redshift dependence (indeed, much weaker than observed).
However, if the gas is undergoing Hubble expansion, then 
$\bar\rho \propto (1+z)^3$ and one predicts
an increasing neutral fraction with $z$ 
provided the emissivity from ionizing sources does
not rise more steeply than $(1+z)^3$.
In any case, the number density of ionizing sources
should also scale as $(1+z)^\gamma$
and altogether, we have a physical motivation for a $(1+z)^\gamma$
evolution in $\ell(z)$.

The results at $z \sim 2-3$ shown in Figure~\ref{fig:kim13_loz}
yield
\begin{equation}
\ell(z) = 100 (1+z)^{1.12 \pm 0.24}
\end{equation}
for $\mnhi < 10^{14} \cm{-2}$ and
\begin{equation}
\ell(z) = 0.4 \, (1+z)^{4.14 \pm 0.6}
\end{equation}
for $\mnhi = [10^{14}, 10^{17}] \cm{-2}$.

%\item Current results including measurements at $z<1$ (Kim et al.\ 2003, Figure 5)
%			\begin{itemize}
%			\item Power-law extrapolation of $\ell(z)$ to $z=0$ fails
%			\item Observed incidence of absorption is much *higher*
%				\begin{itemize}
%				\item Caution: sample size is modest
%				\item Caution: Comparison at fixed \nhi\ implies varying gas density (see Schaye01 formalism to come)
%%				\end{itemize}
%			\item Nevertheless, we infer a greatly reduced EUVB
%				\begin{itemize}
%				\item Consistent with declining quasar population
%				\item Although not exactly (see Kollmeier et al.\ 2014)
%				\end{itemize}
%			\end{itemize}

\subsection{Mock Spectra of the \lya\ Forest}
Provided the \fnhi\ distribution, one may generate mock
spectra of the \lya\ forest (with or without an underlying
source continuum).  This produces an empirical description
that basically ignores any underlying physical model
and its implications (e.g. clustering of gas in the IGM).
However, as it is derived directly from the data it may
have a greater realism in many respects.  

An example is provided in the {\it fN} Notebook and the
code is available in the {\it pyigm} package.  We provide
here the basic recipe:

\begin{enumerate}
		\item Define a redshift interval $\delta z$ for the mock Forest
		\item Define \nhi\ bounds for \fnhi\ (usually the full dynamic range)
		\item Calculate the average number of lines 
		($N_{lines} = \ell(z) \delta z$)
		\item Random draw (Poisson) from $N_{lines}$
		\item Draw \nhi\ values from \fnhi
		\item Draw $b$ from $g(b)$
		\item Draw $z$ from $\ell(z)$
		\item Calculate $\tau_\lambda$, including all 
		Lyman series lines (as applicable)
			\begin{itemize}
			\item On a wavelength grid fine enough to 
			capture the line profile
			\item i.e.\ a perfect spectrometer
			\end{itemize}
		\item Calculate $F_\lambda = \exp[-\tau_\lambda]$
		\item Add in spectral characteristics
			\begin{itemize}
			\item Convolve with instrument LSF
			\item Rebin to final wavelength array
			\item Add in Noise
			\end{itemize}
		\item Add in source SED
\end{enumerate}

\subsection{Effective \lya\ Opacity: $\tau_{\rm eff}^{\rm Ly\alpha}$}
As emphasized above, the opacity of the \lya\ forest
is highly stochastic, both within a given sightline and
from sightline to sightline.
This implies an underlying, stochastic density field
associated with the underling large-scale structure of the
universe.  While there is great scientific value in the
undulations \cite[e.g.]{lee1X,natalie}, one may also derive
value constraints and inferences from the mean opacity across
cosmic time.  

Let \fnorm\ be the average, normalized flux in
the \lya\ forest as observed across many sightlines. 
See the {\it characterizing\_lya\_forest} Slides for 
a visualization.
One defines an effective \lya\ opacity
\begin{equation}
\tau_{\rm eff, \alpha} \equiv - \ln \mfnorm
\end{equation}
as the average opacity of the IGM from \lya\ absorption alone
(and generally limited to lower density gas, i.e.\ $\mnhi < 10^{17} \cm{-2}$).
An alternative description used is
the \lya\ decrement
\begin{equation}
D_A = 1 \, - \mfnorm
\end{equation}
Cosmologicially (as described below), this
average opacity is a balance between the baryon density 
$\Omega_b$ and the ambient radiation field.
With an accurate measurement of $\tau_{\rm eff,\alpha}$
one can infer either $\Omega_b$ or constrain the radiation
field. 

In principle, one can estimate $\tau_{\rm eff,\alpha}$
from \fnhi\ as described by
\cite{mj90,press93}. %[Moller \& Jakobsen 1990; Press et al.\ 1993]
We describe the technique here for completeness but note
that $\tau_{\rm eff,\alpha}$ is measured directly from observations.
As such, it can be used as a constraint on \fnhi.
Consider the number of lines $\mathcal{N}$ per unit rest wavelength 
per redshift interval $dz$
with
\begin{equation}
d\lambda_{\rm rest} = \lambda_{\rm rest} dz / (1+z) \;\; .
\end{equation}
We can calculate the number of lines per $dz$
from  our \HI\ frequency distribution
\begin{equation}
\ell(z) = \int f(\mnhi,b,z) dN db
\end{equation}
Translating to a rest wavelength interval,
\begin{equation}
\mathcal{N} = \frac{1+z}{\lambda_{\rm rest}} \int f(\mnhi,b,z) dN db
\end{equation}
For these lines, the
mean equivalent width is

\begin{equation}
\bar{W_\lambda} = \frac{1+z}{\mathcal{N} \lambda_{\rm rest}} 
\int f(\mnhi,b,z) W_\lambda(N,b) dN db 
\end{equation}
If the lines are randomly distributed (i.e.\ no clustering),
then the mean transmission ($1-D_A$) is\footnote{
This equation appears intuitive yet a proper derivation 
is remarkably complex! (see above references)}

\begin{equation}
1-D_A = \exp (-\mathcal{N} \bar W_\lambda)
\end{equation}
An expression for the effective opacity follows, 
\begin{equation}
\tau_{\rm eff, \alpha} = \int f(\mnhi,b,z) 
W^{\rm Ly\alpha}_\lambda(N,b) \, dN db  \;\; .
\label{eqn:teff}
\end{equation}
This equation is relatively straightforward, but
not analytic.
A typical cheat is to assume the average $b$-value instead
of a distribution which
gives a one-to-one correspondence between \nhi\ and $W_\lambda$.
The {\it fN} Notebook shows example calculations using
the {\it pyigm} software package.

% For figures use
%
\begin{figure}[b]
\sidecaption
% Use the relevant command for your figure-insertion program
% to insert the figure file.
% For example, with the graphicx style use
\includegraphics[width=4.5in]{Figures/dteff.pdf}
%
% If no graphics program available, insert a blank space i.e. use
%\picplace{5cm}{2cm} % Givethe correct figure height and width in cm
%
\caption{Differential contribution to 
$\tau_{\rm eff, \alpha}$ for $z=2.5$ using the
\fnhi\ function from \cite{p+14}.
}
\label{fig:dteff}       % Give a unique label
\end{figure}

Using \fnhi\ from \cite{p+14}, %Prochaska+14
we estimate $\tau_{\rm eff, \alpha} = 0.24$ at $z=2.5$.
Figure~\ref{fig:dteff} shows the
differential contribution with $\log \mnhi$.
It is evident that $\tau_{\rm eff,\alpha}$
is dominated by lines with $\tau_0 \approx 1$ and
that the result
depends on the choice of $N_{\rm min}$. 

% For figures use
%
\begin{figure}[b]
\sidecaption
% Use the relevant command for your figure-insertion program
% to insert the figure file.
% For example, with the graphicx style use
%\includegraphics[width=4.5in]{Figures/dteff.pdf}
\includegraphics[width=4.5in]{Figures/kirkman05_fig4.pdf}
%
% If no graphics program available, insert a blank space i.e. use
%\picplace{5cm}{2cm} % Givethe correct figure height and width in cm
%
\caption{Measurements of the \lya\ decrement
$D_A$ at $z \approx 2-3$ from a modest sample of
quasar spectra, by \cite{k05}.
}
\label{fig:DA}       % Give a unique label
\end{figure}

Returning to actual measurements of $\tau_{\rm eff,\alpha}$,
the first results were derived from
24 quasar spectra observed at Lick Observatory
\cite{kirkman05}.  The authors employed an
`army' of undergrad students to fit continua,
normalize the data, and assess
systematic uncertainty.
After masking metal absorption and \HI\
lines with $\mnhi > 10^{17} \cm{-2}$, they
reported $D_A$ measurements for $z \approx 2-3$
(Figure~\ref{fig:DA}).
Modeling $D_A$ as $(1+z)^\gamma$, they reported 
\begin{equation}
D_A = 0.0062 \, (1+z)^{2.75}
\end{equation}
which gives $\tau_{\rm eff, \alpha} = 0.22$ 
at $z=2.5$.

Analyzing a larger dataset ($\mathcal{N} \sim 100$)
at higher spectral
resolution, \cite{fg08} extended the results
to $z=4$ and improved the precision of the
measurements.  These authors
%			\item Analyzed nearly 100 echelle spectra at $z>2$
corrected statistically for metal absorption,
assessed continuum bias on mock spectra, and
reported on a disturbing `wiggle' in their measurements
(which has appeared to disappear in later datasets).
Their best-fit power law
is $\tau_{\rm eff, \alpha} = 0.0018 \, (1+z)^{3.92}$.

% For figures use
%
\begin{figure}[b]
\sidecaption
% Use the relevant command for your figure-insertion program
% to insert the figure file.
% For example, with the graphicx style use
%\includegraphics[width=4.5in]{Figures/dteff.pdf}
\includegraphics[width=4.5in]{Figures/becker13_fig4.pdf}
%\includegraphics[width=4.5in]{Paper_figs/kirkman05_fig4.pdf}
%
% If no graphics program available, insert a blank space i.e. use
%\picplace{5cm}{2cm} % Givethe correct figure height and width in cm
%
\caption{Relative measurements of the mean flux in the 
\lya\ forest as measured in composite quasar spectra
by \cite{becker+13}.
}
\label{fig:rel_teff}       % Give a unique label
\end{figure}

Most recently, \cite{becker+13} have considered the full dataset
of SDSS and leveraged the nearly constant mean continuum of
quasar spectra (see Figure~\ref{fig:becker13_fig2}) to
measure $\tau_{\rm eff,\alpha}$ in stacked, composite spectra.
Their analysis yields the relative evolution in $\tau_{\rm eff,\alpha}$
with redshift (Figure~\ref{fig:rel_teff}).
Tying their relative measurement to the absolute value at
$z=2.5$ in \cite{fg08}, they report
\begin{equation}
\tau_{\rm eff, \alpha}(z) = 0.751 \ltp \frac{1+z}{1+3.5} 
\rtp^{2.9}  - 0.132
\end{equation}
which includes
statistical corrections for $\mnhi > 10^{17} \cm{-2}$ absorbers 
and metal absorption.
These measurements provide a blunt yet powerful test
for any cosmological model of the IGM.

\subsection{Fluctuating Gunn-Peterson Approximation (FGPA)}
The modern paradigm of the \lya\ forest, or IGM,
is that the observed opacity results from undulations
in the underlying dark matter density.  The baryons
trace the dark matter overdensities and are predominantly
ionized with ionization state dictated by photoionization
from the extragalactic UV background (EUVB).

This paradigm lends to the formalism now known as the
fluctuating Gunn-Peterson approximation (FPGA).  The
standard formalism is derived as follows.
We begin with our result from quantum mechanics
that the  optical depth of \lya\ at line center is
\begin{equation}
\tau_0 = \frac{\pi e^2}{m_e c} \frac{N_{\rm HI} 
f_{\rm Ly\alpha}}{\Delta \nu_D}
\end{equation}
Expressing the  \nhi\ column density as
\begin{equation}
\mnhi = n_{\rm HI} \Delta r
\end{equation}
with $\Delta r$ an interval in space and $n_{\rm HI}$
the gas density.
From Cosmology (see Equation~\ref{eqn:drdz} for $dr/dz$),
we can relate distance to redshift
\begin{equation}
\Delta r = \frac{c \Delta z}{H(z) (1+z)} \;\; ,
\end{equation}
and we have 
\begin{equation}
\tau_0 = \frac{\pi e^2}{m_e} \frac{n_{\rm HI} \, \Delta z
f_{\rm Ly\alpha}}{\Delta \nu_D H(z) (1+z)}
\end{equation}
Because the absorption occurs over a small redshift interval,
we can express
$\Delta z/(1+z) = \Delta\nu_{\rm Ly\alpha} / \nu_{\rm Ly\alpha}$
and identify $\Delta \nu_D$ with $\Delta \nu_{\rm Ly\alpha}$
(i.e.\ the line width is given by cosmic expansion).
Ionization balance for $n_{\rm HI}$ with ionization rate $\Gamma$
and recombination rate $\alpha(T)$ gives

\begin{equation}
n_{\rm HI} \Gamma = \alpha(T) n_{\rm HII} n_e
\label{eqn:ion_balance}
\end{equation}
Altogether now,
\begin{equation}
\tau = \frac{\pi e^2 f_{\rm Ly\alpha}}{m_e \nu_{\rm Ly\alpha}}
\frac{1}{H(z)} \frac{\alpha(T) n_{\rm HII} n_e}{\Gamma}
\end{equation}

To link the gas density $n$ with the dark
matter density, we introduce the over-density $\delta$
where
\begin{equation}
\rho = \bar\rho (1+\delta)
\label{eqn:overdensity}
\end{equation}
and we express our number densities in terms of the over-density.
Furthermore, we define
ionization and mass fractions  -- $X$ is the Hydrogen mass fraction 
and $x$ is the ionized Hydrogen fraction, to express 

\begin{equation}
n_{\rm HII} = \frac{\rho_{\rm crit} \Omega_b}{m_p} \, X x \; (1+\delta) (1+z)^3
\end{equation}
A similar (uglier) expression for $n_e$ includes Helium 
(with $Y$, $y_{II}, y_{III}$ defined similarly):
\begin{equation}
n_e = \frac{\rho_{\rm crit} \Omega_b}{m_p} [Xx + 0.25Y(y_{II} + 2y_{III})] \; (1+\delta) (1+z)^3
\end{equation}
For gas with $T \sim 10^4$K (as expected for a photoionized medium),
$\alpha(T)$ is a power-law of the form,
\begin{equation}
\alpha(T) \approx \alpha_0 T^{-0.7} \;\; .
\end{equation}
Lastly, one assumes the IGM gas follows a power-law temperature-density
relation, (as derived from hydrodynamic analysis by \cite{hui97})
\begin{equation}
T = T_0 (1+\delta)^\beta
\label{eqn:rhoT}
\end{equation}

The FGPA expression becomes
\begin{equation}
\tau = A(z) \, (1+\delta)^{2 - 0.7 \beta}
\end{equation}
where the opacity ``fluctuates'' with the local over-density.
The $A(z)$ term is proportional to  $(1+z)^6 / [H(z) \Gamma]$.
or, in its glory,
\begin{equation}
A(z) \equiv \frac{\pi e^2 f_{\rm Ly\alpha}}{m_e \nu_{\rm Ly\alpha}}
\ltp \frac{\rho_{\rm crit} \Omega_b}{m_p} \rtp^2 
\frac{1}{H(z)} Xx [Xx + 0.25Y(y_{II} + 2y_{III})] \frac{\alpha_0 T_0^{-0.7}}{\Gamma}
(1+z)^6
\end{equation}
Given a probability density function for the overdensity
$P(\Delta)$ with $\Delta \equiv 1 + \delta$, we 
may calculate the mean flux (and opacity)
\begin{equation}
<F>(z) = \intl_0^\infty P(\Delta; z) \exp(-\tau) d\Delta \;\; .
\label{eqn:fgpa}
\end{equation}
See \cite{miralda00} for models of $P(\Delta)$
based on numerical simulations.
This FGPA provides an analytic expression, calibrated
against cosmological simulations, for the IGM opacity.

Returning to $\tau_{\rm eff,\alpha}$ measurements,
we can relate Equation~\ref{eqn:fgpa} to the data
with the only significant unknown\footnote{There is a
weak dependence on $T$ which leads to degeneracy.} 
being the photoionization rate $\Gamma$.
Figure~\ref{fig:Gamma} from \cite{fg08c}
shows estimates for $\Gamma$
ased on the $\tau_{\rm eff,\alpha}$ measurements of 
\cite{fg08a}.  The figure also shows their  
estimate for $\Gamma$ from quasars alone and the
authors argued that star-forming galaxies
must dominate the radiation field by $z \sim 3$.
Recent work, however, has raised the possibility
that faint quasars play as strong a role
\cite{giallongo15,hm15} an issue that awaits
deeper, high-$z$ surveys.

\subsection{Effective Lyman Series Opacity}
We finish this section by extending our calculations
for the effective opacity of \lya\ to the full
Lyman series.  Each line up the energy ladder of \HI\
contributes additional opacity at shorter
wavelengths with sequentially smaller opacity
($\tau_0 \propto \lambda f N$).
A complex `blending' of the Lyman series
from various redshifts occurs.  

For example
consider the quasar light emitted at 
at $\lambda_{\rm rest} = 920$\AA) 
from a quasar at $z_{\rm em} = 4$
(we observe these photons at $\lambda_{\rm obs} = 4600$\AA).
The photons emitted by the quasar will redshift into
resonance with 
\HI\ Ly$\alpha$ at $z_{\rm Ly\alpha} = 2.78$,
Ly$\beta$ at $z_{\rm Ly\beta} = 3.48$,
Ly$\gamma$ at $z_{\rm Ly\gamma} = 3.73$,
Ly$\delta$ at $z_{\rm Ly\delta} = 3.84$, etc.

% For figures use
%
\begin{figure}[b]
\sidecaption
% Use the relevant command for your figure-insertion program
% to insert the figure file.
% For example, with the graphicx style use
%\includegraphics[width=4.5in]{Figures/dteff.pdf}
%\includegraphics[width=4.5in]{Figures/fg08c_fig1.pdf}
\includegraphics[width=5.0in]{Figures/sawtooth.pdf}
%\includegraphics[width=4.5in]{Paper_figs/kirkman05_fig4.pdf}
%
% If no graphics program available, insert a blank space i.e. use
%\picplace{5cm}{2cm} % Givethe correct figure height and width in cm
%
\caption{So-called `sawtooth' transmission function of the
\HI\ Lyman series of the IGM for a source at $z=5$.
Redward of \HI\ \lya\ there is no attenuation
($\mathcal{T} = 1$).  From $\lambda_{\rm obs} \approx 5100-6000$\AA,
only \HI\ \lya\ contributes and the increasing transmission
with decreasing redshift is due to evolution in the \lya\ line density.
At shorter wavelengths, additional \HI\ Lyman series lines
contribute and reduce the transmission accordingly.
}
\label{fig:sawtooth}       % Give a unique label
\end{figure}

The effective opacity from each Lyman series is independent
and sum simply.  An example calculation for the
predicted transmission $\mathcal{T}$ through the IGM
with $\mathcal{T} = \exp(-\tau_{\rm eff}$ is
shown in Figure~\ref{fig:sawtooth}.
This `sawtooth' curve has teeth at each of the additional
Lyman series lines.  The slope of increasing transmission
with decreasing wavelength is due to the decreasing 
incidence of lines with decreasing redshift.
The {\it fN} Notebook shows an example calculation using
software from {\it pyigm}.

In Figure~\ref{fig:stack_saw}, we show the average 
relative flux (normalized to unity at $\lambda_{\rm rest} = 1450$\AA)
of $\approx 150$, $z \sim 4$ quasars from the SDSS.
Overlayed on the data is a the quasar SED from 
\cite{telfer02} attenuated by the effective opacity
shown in Figure~\ref{fig:sawtooth}.
The good agreement down to $\lambda \approx 4700$\AA\
is truly remarkable and demonstrates the quality of our
characterization of the IGM.

This IGM attenuation 
is imprinted in the spectra of {\it all} distant sources.
This includes galaxies.  
For a $z=4$ galaxy, the $g$-band flux suffers a decrement
of $\Delta m_g \approx 1$\,mag.
This insight  led to the discovery
of the first high-$z$ star-forming galaxies 
\cite{madau95,steidel96,koo}.
This attenuation will also affect our \lya\ emitters (LAEs).
At $z=5$, $\tau_{\rm eff, \alpha} \approx 1.5$ 
which leads to an asymmetry in \lya\ emission. 


% For figures use
%
\begin{figure}[b]
\sidecaption
% Use the relevant command for your figure-insertion program
% to insert the figure file.
% For example, with the graphicx style use
%\includegraphics[width=4.5in]{Figures/dteff.pdf}
%\includegraphics[width=4.5in]{Figures/fg08c_fig1.pdf}
\includegraphics[width=5.0in]{Figures/obs_sawtooth.pdf}
%\includegraphics[width=5.0in]{Figures/sawtooth.pdf}
%\includegraphics[width=4.5in]{Paper_figs/kirkman05_fig4.pdf}
%
% If no graphics program available, insert a blank space i.e. use
%\picplace{5cm}{2cm} % Givethe correct figure height and width in cm
%
\caption{Composite spectrum of 150 quasars at $z \sim 4$
from the SDSS (black).  These are shown in the observer
frame for $z=4$ (but were stacked in the rest-frame)
and were normalized to unity at $\lambda_{\rm rest} = 1450$\AA.
Overlayed on the data is a model using the quasar SED
from \cite{telfer+02} and our estimate of the IGM transmission
from the full \HI\ Lyman series.  The excellent agreement
(down to at least 4700\AA) is remarkable.
}
\label{fig:stack_saw}       % Give a unique label
\end{figure}

%

%
\begin{acknowledgement}
Acknowledge the NSF.  Wolfe.  Blumenthal
\end{acknowledgement}
%
%\section*{Appendix}
%\addcontentsline{toc}{section}{Appendix}
%
%
%When placed at the end of a chapter or contribution (as opposed to at the end of the book), the numbering of tables, figures, and equations in the appendix section continues on from that in the main text. Hence please \textit{do not} use the \verb|appendix| command when writing an appendix at the end of your chapter or contribution. If there is only one the appendix is designated ``Appendix'', or ``Appendix 1'', or ``Appendix 2'', etc. if there is more than one.

%\input{referenc}
% \bibliographystyle{}
% \bibliography{}
\end{document}